\chapter*{Podsumowanie}\label{chap:podsumowanie}
\addcontentsline{toc}{chapter}{Podsumowanie}

Celem pracy niniejszej pracy było zaprojektowanie i zaimplementowanie systemu multilateracyjnego działającego w domenie fal dźwiękowych. W ramach pracy dokonano wyboru metody multilateracji najlepiej nadającej się do tego zastosowania, jak również przeprowadzono szereg testów wykazujących nieprawidłowości działania, a następnie podjęto próby eliminacji tych błędów. Podczas eksperymentów podstawową potrzebę dokładnej synchronizacji zegarów urządzeń w systemie na efekty jego działania. Przeanalizowano też znaczenie wpływu zastosowanej metody synchronizacji zegarów, a także interesujące skalowanie estymowanej odległości wynikającej z fizycznych mechanizmów rozchodzenia się fali dźwiękowej.

Wyniki przeprowadzonych badań wskazują, że system multilateracyjny oparty o szeroko dostępne moduły mikrofonowe jest zdolny śledzić pozycję nadajnika sygnałów dźwiękowych w podzbiorze przestrzeni dwuwymiarowej obejmowanym ograniczonym zasięgiem tychże mikrofonów w węzłach odbiorczych. Znaczącą wadą systemu jest duża podatność na warunki akustyczne otoczenia, których mała zmiana może znacząco negatywnie wpłynąć na skuteczność lokalizacji. Dodatkowo, zasięg mikrofonów stanowił ograniczenie, przez co obszar, w którym możliwa była lokalizacja miał mniej niż $0.5 m^2$. Możliwość działania na większych odległościach między odbiornikami jest niepewna {-} zastosowanie lepszej jakości mikrofonów, o niższym poziomie szumu mogłoby znacząco poprawić ich zasięg. Biorąc jednak pod uwagę fakt, że system nie jest kosztowny obliczeniowo, co pozwoliło na zastosowanie prostych, a co za tym idzie, tanich komponentów (koszt odbiornika wyniósł mniej niż 5\$, a serwer mógłby być hostowany na prostym komputerze SBC (\textit{ang.\ single-board computer}) typu Raspberry Pi) zdolność do lokalizacji nadajnika przez zbudowany system powinna być zadowalająca.

Dalszymi krokami rozwoju systemu, których nie podjęto ze względu na ograniczony budżet czasowy mogłyby być:

\begin{itemize}
    \item rozszerzenie do i ewaluacja działania w przestrzeni trójwymiarowej,
    \item bardziej wnikliwe testy i badania roli urządzeń przetwarzających sygnał dźwiękowy na cyfrowy; inwestygacja wpływu typu rodzaju mikrofonu, wzmacniacza czy przetwornika ADC (\textit{ang.\ analog to digital}) na estymację odległości,
    \item stworzenie lepszego algorytmu korekcji estymowanej odległości względem rzeczywistej,
    \item ewentualne dostosowywanie algorytmów synchronizacji zegarów do przyszłych zastosowań,
    \item próba minimalizacji wpływu charakterystyki dźwiękowej otoczenia na dokładność wyników.
\end{itemize}

Podsumowując, zaprojektowany i zaimplementowany system multilateracyjny działający w domenie fal dźwiękowych udowodnił swoją skuteczność w śledzeniu pozycji nadajnika w dwuwymiarowej przestrzeni, otwierając tym samym drogę do dalszych badań i udoskonaleń, które mogą znacząco poszerzyć jego zastosowanie w różnych dziedzinach.