\chapter{Multilateriacja}\label{chap:multilateration}

\section{Przygotowanie danych wejściowych}

Aby otrzymać poprawne wyniki algorytmu multilateracji dane wejściowe, w naszym przypadku odległości między węzłami odbiorczymi a nadajnikiem powinny być jak najbliższe rzeczywistym odległościom z możliwie małymi odchyleniami. Będziemy kontynuować usprawnianie metod uzyskiwania poprawnych wyników zapoczątkowane w rozdziale poprzednim.

\subsection{Korekcja odległości}

W wynikach eksperymentów porównawczych metod synchronizacji czasu węzłów przeprowadzonych w rozdziale~\ref{chap:time_sync}.\ zaobserwowaliśmy skalowanie wyników na pierwszy rzut oka zachowujące się liniowo. Przyjrzyjmy się teraz dokładnie temu zjawisku. W tym i kolejnych przypadkach będziemy używać już jedynie synchronizacji sprzętowej z użyciem mikrofonów ze względu na brak konieczności dodatkowej kalibracji przesunięcia punktu 0.

\section{Ewaluacja działania systemu}

\section{Wyniki}

\subsection{Interpretacja}

\subsection{Wnioski}