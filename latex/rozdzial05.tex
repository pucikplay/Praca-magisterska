\chapter{Multilateriacja}\label{chap:multilateration}

\section{Przygotowanie danych wejściowych}

Aby otrzymać poprawne wyniki algorytmu multilateracji dane wejściowe, w naszym przypadku odległości między węzłami odbiorczymi a nadajnikiem powinny być jak najbliższe rzeczywistym odległościom z możliwie małymi odchyleniami. Będziemy kontynuować usprawnianie metod uzyskiwania poprawnych wyników zapoczątkowane w rozdziale poprzednim.

\subsection{Korekcja odległości}

W wynikach eksperymentów porównawczych metod synchronizacji czasu węzłów przeprowadzonych w rozdziale~\ref{chap:time_sync}.\ zaobserwowaliśmy skalowanie wyników na pierwszy rzut oka zachowujące się liniowo. Przyjrzyjmy się teraz dokładnie temu zjawisku. W tym i kolejnych przypadkach będziemy używać już jedynie synchronizacji sprzętowej z użyciem mikrofonów ze względu na brak konieczności dodatkowej kalibracji przesunięcia punktu 0.

Prawdopodobnym powodem skalowania obliczanych odległości może być odczyt zmiany sygnału mikrofonowego odczytywanego przez mikrokontroler. Poniższe wykresy są wynikiem czterech kolejnych eksperymentów, w których jedyną zmienną była czułość zintegrowanego wzmacniacza mikrofonu. Wzmacniacz ten nie pozwala na precyzyjną regulację, a jedynie na zmianę rezystancji wbudowanego potencjometru. Ponieważ przedział czułości odpowiadający wykrywaniu sygnału węzła nadającego przy jednoczesnym zminimalizowaniu fałszywych aktywacji jest niewielki (około $\frac{1}{8}$ obrotu potencjometru) cztery zbadane przypadki nie dzielą równo badanego zakresu. Rozpoczynając od największej możliwej czułości przy każdym kolejnym eksperymencie zmniejszano ją póki pozwalała wciąż na wykrywanie sygnału z badanych odległości.

\begin{figure}[h]
\centering
    \includegraphics[width=.49\textwidth]{pics/mic_sync_dist/dists_long_0.png}
    \includegraphics[width=.49\textwidth]{pics/mic_sync_dist/dists_close_long_0.png}
\caption{Pomiar obliczanych odległości 1.}
\label{pic:slope_test_0}
\end{figure}

\begin{figure}[h]
\centering
    \includegraphics[width=.49\textwidth]{pics/mic_sync_dist/dists_long_1.png}
    \includegraphics[width=.49\textwidth]{pics/mic_sync_dist/dists_close_long_1.png}
\caption{Pomiar obliczanych odległości 2.}
\label{pic:slope_test_1}
\end{figure}

\begin{figure}[h]
\centering
    \includegraphics[width=.49\textwidth]{pics/mic_sync_dist/dists_long_2.png}
    \includegraphics[width=.49\textwidth]{pics/mic_sync_dist/dists_close_long_2.png}
\caption{Pomiar obliczanych odległości 3.}
\label{pic:slope_test_2}
\end{figure}

\begin{figure}[h]
\centering
    \includegraphics[width=.49\textwidth]{pics/mic_sync_dist/dists_long_3.png}
    \includegraphics[width=.49\textwidth]{pics/mic_sync_dist/dists_close_long_3.png}
\caption{Pomiar obliczanych odległości 4.}
\label{pic:slope_test_3}
\end{figure}

Aby lepiej odczytać informacje z wykresu uśrednijmy pomiary dla każdej z badanych odległości i dodajmy do nich funkcje liniowe o współczynniku otrzymanym przy pomocy regresji liniowej z tychże uśrednionych punktów.

\begin{figure}[h]
\centering
    \includegraphics[width=.49\textwidth]{pics/mic_sync_dist/dists_close_long_0_mean.png}
    \includegraphics[width=.49\textwidth]{pics/mic_sync_dist/dists_close_long_1_mean.png}
    \includegraphics[width=.49\textwidth]{pics/mic_sync_dist/dists_close_long_2_mean.png}
    \includegraphics[width=.49\textwidth]{pics/mic_sync_dist/dists_close_long_3_mean.png}
\caption{Średnie obliczonych odległości}
\label{pic:slope_test_mean}
\end{figure}

Na wykresach można zauważyć trendy związane ze zmniejszającą się czułością wzmacniacza mikrofonu:

\begin{itemize}
    \item zmniejszanie się odległości punktu, w którym czułość jest zbyt mała by niezawodnie wyrywać nadawane sygnały,
    \item współczynnik prostej aproksymującej skalowane odległości rośnie.
\end{itemize}

Podobne efekty zaobserwowano kiedy brzęczyk nie był kierowany bezpośrednio w kierunku odbiornika.

\section{Ewaluacja działania systemu}

W celu ewaluacji działania przeprowadzono szereg eksperymentów testujących działanie systemu multilateracyjnego w zaproponowanych scenariuszach. Bazując na uprzednich obserwacjach rozszerzono program serwera obliczeniowego tak, aby bezpośrednio przed multilateracją przeprowadzone były synchronizacja czasu oraz korekcja odległości. Zasada działania obu typów węzłów pozostała niezmieniona.

\subsection{Wyniki}

Wyniki eksperymentów przedstawiono na wykresach. Mniejsze punkty o wyblakłym kolorze reprezentują badane punkty, w których umieszczony był węzeł nadawczy, natomiast większe, nasyconym kolorze odpowiadające im punkty będące uśrednionym wynikiem 25 powtórzeń polecenia lokalizacji w systemie. Wykresy należące do tej samej grupy zachowują skalę w celu łatwiejszego porównania.

\subsubsection{Eksperymeny jednowymiarowe}

Ewaluację rozpoczęto, tak jak podczas eksperymentu zerowego, od przypadku jednowymiarowego, aby sprawdzić czy dokładność i stabilność pomiarów w systemie została poprawiona. Tak samo jak pierwotnie nadajnik umiejscowiono w punkcie $(0)$, natomiast dwa odbiorniki w punktach odpowiednio $(-0.5)$ i $(0.5)$, wszystkie węzły były stacjonarne. Nadajnik co $0.5$ $s$ nadawał sygnał o długości $10$ $ms$, a serwer co $0.5$ $s$ zwracał wynik zagadnienia multilateracji na podstawie ostatnio otrzymanych danych. Następnie powtórzono eksperyment z użyciem dwóch dodatkowych węzłów umiejscowionych w punktach $(-0.25)$ i $(0.25)$. Wyniki przedstawiono na wykresach~\ref{pic:1d_mult}.

\begin{figure}[H]
\centering
\begin{subfigure}{.5\textwidth}
    \centering
    \includegraphics[width=\linewidth]{pics/mult_lat_1d/positions_2_mean.png}
\caption{2 węzły}
\label{pic:1d_2_mult}
\end{subfigure}%
\begin{subfigure}{.5\textwidth}
    \centering
    \includegraphics[width=\linewidth]{pics/mult_lat_1d/positions_4_mean.png}
\caption{4 węzły}
\label{pic:1d_4_mult}
\end{subfigure}
\caption{Średnie obliczonych pozycji, wariant 1D}
\label{pic:1d_mult}
\end{figure}

\subsubsection{Eksperymeny dwuwymiarowe}

Następnie sprawdzono system w wariancie dwuwymiarowym. Zby zachować podobną maksymalną odległość pomiędzy odbiornikami zdecydowano na umiejscowienie czterech węzłów na wierzchołkach kwadratu o boku $0.6$ symetrycznego względem puntu $(0,0)$, a więc na punktach o współrzędnych $(\pm0.3, \pm0.3)$. Zbadano dokładność lokalizacji w trzech grupach punktów:
\begin{itemize}
    \item $\{(0,0), (0.3,0.3), (-0.3,0.3), (-0.3,-0.3), (0.3,-0.3)\}$
    \item $\{(0,0), (0,0.3), (0,-0.3), (-0.3,0), (0.3,0)\}$
    \item $\{(0,0), (0.15,0.15), (-0.15,0.15), (-0.15,-0.15), (0.15,-0.15)\}$
\end{itemize}
Wyniki przedstawiono na wykresach~\ref{pic:2d_mult}.

\begin{figure}[H]
\centering
\begin{subfigure}{.5\textwidth}
    \centering
    \includegraphics[width=\linewidth]{pics/mult_lat_2d/positions_1_mean.png}
\caption{zestaw punktów 1.}
\label{pic:2d_1_mult}
\end{subfigure}%
\begin{subfigure}{.5\textwidth}
    \centering
    \includegraphics[width=\linewidth]{pics/mult_lat_2d/positions_2_mean.png}
\caption{zestaw punktów 2.}
\label{pic:2d_2_mult}
\end{subfigure}
\end{figure}
\begin{figure}[H]
\ContinuedFloat\centering
\begin{subfigure}{.5\textwidth}
    \centering
    \includegraphics[width=\linewidth]{pics/mult_lat_2d/positions_3_mean.png}
\caption{zestaw punktów 3.}
\label{pic:2d_3_mult}
\end{subfigure}
\caption{Średnie obliczonych pozycji, wariant 2D}
\label{pic:2d_mult}
\end{figure}

Następnie w celu ewaluacji wpływu charakterystyki otoczenia na działanie systemu przeprowadzono trzy następujące bezpośrednio po sobie eksperymenty zachowując ustaloną na początku synchronizację zegarów i stałe korekcji odległości. Sprawdzono dokładność estymacji tego samego zbioru punktów wyjściowo oraz po rotacji całego układu o $45^{\circ}$ i $90^{\circ}$ zgodnie z ruchem wskazówek zegara. Wyniki przedstawiono na wykresach~\ref{pic:2d_angle_mult}. 

\begin{figure}[H]
\centering
\begin{subfigure}{.5\textwidth}
    \centering
    \includegraphics[width=\linewidth]{pics/mult_lat_2d_angle/positions_0_mean.png}
\caption{rotacja $0^{\circ}$}
\label{pic:2d_0_angle_mult}
\end{subfigure}%
\begin{subfigure}{.5\textwidth}
    \centering
    \includegraphics[width=\linewidth]{pics/mult_lat_2d_angle/positions_45_mean.png}
\caption{rotacja $45^{\circ}$}
\label{pic:2d_45_angle_mult}
\end{subfigure}
\end{figure}
\begin{figure}[H]
\ContinuedFloat\centering
\begin{subfigure}{.5\textwidth}
    \centering
    \includegraphics[width=\linewidth]{pics/mult_lat_2d_angle/positions_90_mean.png}
\caption{rotacja $90^{\circ}$}
\label{pic:2d_90_angle_mult}
\end{subfigure}
\caption{Średnie obliczonych pozycji, wariant 2D z rotacją}
\label{pic:2d_angle_mult}
\end{figure}

Ostatecznie sprawdzono zachowanie systemu wraz ze zmianą liczby węzłów odbiorczych od minimalnej liczby trzech aż do ośmiu. Węzły umiejscowiono na punktach ze zbioru $\{(0.3,0.3), (-0.3,0.3), (-0.3,-0.3), (0.3,-0.3), (0, 0.425), (-0.425, 0), (0, -0.425), (0.425, 0)\}$ oraz dodawano je zgodnie z tą kolejnością. Wyniki przedstawiono na wykresach~\ref{pic:2d_num_mult}. 

\begin{figure}[H]
\centering
\begin{subfigure}{.5\textwidth}
    \centering
    \includegraphics[width=\linewidth]{pics/mult_lat_2d_num/positions_3_mean.png}
\caption{3 węzły}
\label{pic:2d_3_num_mult}
\end{subfigure}%
\begin{subfigure}{.5\textwidth}
    \centering
    \includegraphics[width=\linewidth]{pics/mult_lat_2d_num/positions_4_mean.png}
\caption{4 węzły}
\label{pic:2d_4_num_mult}
\end{subfigure}
\end{figure}
\begin{figure}[H]
\ContinuedFloat\centering
\begin{subfigure}{.5\textwidth}
    \centering
    \includegraphics[width=\linewidth]{pics/mult_lat_2d_num/positions_5_mean.png}
\caption{5 węzłów}
\label{pic:2d_5_num_mult}
\end{subfigure}%
\begin{subfigure}{.5\textwidth}
    \centering
    \includegraphics[width=\linewidth]{pics/mult_lat_2d_num/positions_6_mean.png}
\caption{6 węzłów}
\label{pic:2d_6_num_mult}
\end{subfigure}
\end{figure}
\begin{figure}[H]
\ContinuedFloat\centering
\begin{subfigure}{.5\textwidth}
    \centering
    \includegraphics[width=\linewidth]{pics/mult_lat_2d_num/positions_7_mean.png}
\caption{7 węzłów}
\label{pic:2d_7_num_mult}
\end{subfigure}%
\begin{subfigure}{.5\textwidth}
    \centering
    \includegraphics[width=\linewidth]{pics/mult_lat_2d_num/positions_8_mean.png}
\caption{8 węzłów}
\label{pic:2d_8_num_mult}
\end{subfigure}
\caption{Średnie obliczonych pozycji, wariant 2D ze zmienną liczbą węzłów}
\label{pic:2d_num_mult}
\end{figure}

\section{Interpretacja i wnioski}

