\chapter{Multilateriacja}\label{chap:multilateration}

\section{Przygotowanie danych wejściowych}

Aby otrzymać poprawne wyniki algorytmu multilateracji dane wejściowe, w naszym przypadku odległości między węzłami odbiorczymi a nadajnikiem powinny być jak najbliższe rzeczywistym odległościom z możliwie małymi odchyleniami. Będziemy kontynuować usprawnianie metod uzyskiwania poprawnych wyników zapoczątkowane w rozdziale poprzednim.

\subsection{Korekcja odległości}

W wynikach eksperymentów porównawczych metod synchronizacji czasu węzłów przeprowadzonych w rozdziale~\ref{chap:time_sync}.\ zaobserwowaliśmy skalowanie wyników na pierwszy rzut oka zachowujące się liniowo. Przyjrzyjmy się teraz dokładnie temu zjawisku. W tym i kolejnych przypadkach będziemy używać już jedynie synchronizacji sprzętowej z użyciem mikrofonów ze względu na brak konieczności dodatkowej kalibracji przesunięcia punktu 0.

Prawdopodobnym powodem skalowania obliczanych odległości może być odczyt zmiany sygnału mikrofonowego odczytywanego przez mikrokontroler. Poniższe wykresy są wynikiem czterech kolejnych eksperymentów, w których jedyną zmienną była czułość zintegrowanego wzmacniacza mikrofonu. Wzmacniacz ten nie pozwala na precyzyjną regulację, a jedynie na zmianę rezystancji wbudowanego potencjometru. Ponieważ przedział czułości odpowiadający wykrywaniu sygnału węzła nadającego przy jednoczesnym zminimalizowaniu fałszywych aktywacji jest niewielki (około $\frac{1}{8}$ obrotu potencjometru) cztery zbadane przypadki nie dzielą równo badanego zakresu. Rozpoczynając od największej możliwej czułości przy każdym kolejnym eksperymencie zmniejszano ją, póki pozwalała wciąż na wykrywanie sygnału z badanych odległości.

\begin{figure}[H]
    \centering
    \begin{subfigure}{\textwidth}
        \centering
        \includegraphics[width=\textwidth]{pics/mic_sync_dist/dists_long_0.png}
        \caption{pomiar 1.}
        \label{pic:slope_test_0}
    \end{subfigure}
\end{figure}
\begin{figure}[H]
    \ContinuedFloat\centering
    \begin{subfigure}{\textwidth}
        \centering
        \includegraphics[width=\textwidth]{pics/mic_sync_dist/dists_long_1.png}
        \caption{pomiar 2.}
        \label{pic:slope_test_1}
    \end{subfigure}
\end{figure}
\begin{figure}[H]
    \ContinuedFloat\centering
    \begin{subfigure}{\textwidth}
        \centering
        \includegraphics[width=\textwidth]{pics/mic_sync_dist/dists_long_2.png}
        \caption{pomiar 3.}
        \label{pic:slope_test_2}
    \end{subfigure}
\end{figure}
\begin{figure}[H]
    \ContinuedFloat\centering
    \begin{subfigure}{\textwidth}
        \centering
        \includegraphics[width=\textwidth]{pics/mic_sync_dist/dists_long_3.png}
        \caption{pomiar 4.}
        \label{pic:slope_test_3}
    \end{subfigure}
    \caption{Pomiar obliczanych odległości.   Diagram po prawej stanowi powiększenie zaznaczonego obszaru diagramu po lewej.}
    \label{fig:slope_test}
\end{figure}

Aby lepiej odczytać informacje z wykresu uśrednijmy pomiary dla każdej z badanych odległości i dodajmy do nich funkcje liniowe o współczynniku otrzymanym przy pomocy regresji liniowej z tychże uśrednionych punktów.

\begin{figure}[H]
    \centering
    \begin{subfigure}{0.5\textwidth}
        \centering
        \includegraphics[width=\textwidth]{pics/mic_sync_dist/dists_close_long_0_mean.png}
        \caption{pomiar 1.}
        \label{pic:slope_test_mean_0}
    \end{subfigure}%
    \begin{subfigure}{0.5\textwidth}
        \centering
        \includegraphics[width=\textwidth]{pics/mic_sync_dist/dists_close_long_1_mean.png}
        \caption{pomiar 2.}
        \label{pic:slope_test_mean_1}
    \end{subfigure}
\end{figure}
\begin{figure}[H]
    \ContinuedFloat\centering
    \begin{subfigure}{0.5\textwidth}
        \centering
        \includegraphics[width=\textwidth]{pics/mic_sync_dist/dists_close_long_2_mean.png}
        \caption{pomiar 3.}
        \label{pic:slope_test_mean_2}
    \end{subfigure}%
    \begin{subfigure}{0.5\textwidth}
        \centering
        \includegraphics[width=\textwidth]{pics/mic_sync_dist/dists_close_long_3_mean.png}
        \caption{pomiar 4.}
        \label{pic:slope_test_mean_3}
    \end{subfigure}
    \caption{Średnie obliczonych odległości}
    \label{fig:slope_test_mean}
\end{figure}

Na wykresach można zauważyć trendy związane ze zmniejszającą się czułością wzmacniacza mikrofonu:

\begin{itemize}
    \item zmniejszanie się odległości punktu, w którym czułość jest zbyt mała by niezawodnie wyrywać nadawane sygnały,
    \item współczynnik prostej aproksymującej skalowane odległości rośnie.
\end{itemize}

Podobne efekty zaobserwowano kiedy brzęczyk nie był kierowany bezpośrednio w kierunku odbiornika.

\section{Ocena działania systemu}

W celu oceny działania tak skonstruowanego systemu multilateracyjnego przeprowadzono szereg eksperymentów testujących jego działanie   w zaproponowanych scenariuszach. Bazując na uprzednich obserwacjach, rozszerzono program serwera obliczeniowego tak, aby bezpośrednio przed multilateracją przeprowadzone były synchronizacja czasu oraz korekcja odległości. Zasada działania obu typów węzłów pozostała niezmieniona.

\subsection{Wyniki}

Wyniki eksperymentów przedstawiono na  poniższych wykresach. Mniejsze punkty o wyblakłym kolorze reprezentują badane punkty, w których umieszczony był węzeł nadawczy, natomiast większe, o nasyconym kolorze to odpowiadające im punkty będące uśrednionym wynikiem 25 powtórzeń polecenia lokalizacji w systemie. Wykresy należące do tej samej grupy zachowują skalę w celu łatwiejszego porównania. Dodatkowo, dla przypadku jednowymiarowego, dołączono wybrane wykresy zawierające osobne pomiary każdego z punktów, które składały się na wynik uśredniony.

\subsubsection{Eksperymeny jednowymiarowe}\label{sec:1d}

Ewaluację rozpoczęto, tak jak podczas eksperymentu zerowego, od przypadku jednowymiarowego, aby sprawdzić czy dokładność i stabilność pomiarów w systemie została poprawiona. Tak samo jak pierwotnie nadajnik umiejscowiono w punkcie $(0)$, natomiast dwa odbiorniki w punktach odpowiednio $(-0,5)$ i $(0,5)$, wszystkie węzły były stacjonarne. Nadajnik co $0,5$ $s$ nadawał sygnał o długości $10$ $ms$, a serwer co $0,5$ $s$ zwracał wynik zagadnienia multilateracji na podstawie ostatnio otrzymanych danych. Następnie zbadano dokładność aproksymacji dla punktów ze zbioru $\{(-0,5; -0,25; 0,25; 0,5)\}$, jak również powtórzono wszystkie pomiary z użyciem dwóch dodatkowych węzłów umiejscowionych w punktach $(-0,25)$ i $(0,25)$. Wyniki przedstawiono na wykresach~\ref{fig:1d_mult_separate} oraz~\ref{fig:1d_mult}.

\begin{figure}[H]
    \centering
    \begin{subfigure}{\textwidth}
        \centering
        \includegraphics[width=\linewidth]{pics/mult_lat_1d/position_[0]_2.png}
        \caption{Punkt w pozycji (0), 2 węzły}
        \label{pic:1d_mult_[0]_2}
    \end{subfigure}
\end{figure}
\begin{figure}[H]
    \ContinuedFloat\centering
    \begin{subfigure}{\textwidth}
        \centering
        \includegraphics[width=\linewidth]{pics/mult_lat_1d/position_[0]_4.png}
        \caption{Punkt w pozycji (0), 4 węzły}
        \label{pic:1d_mult_[0]_4}
    \end{subfigure}
\end{figure}
\begin{figure}[H]
    \ContinuedFloat\centering
    \begin{subfigure}{\textwidth}
        \centering
        \includegraphics[width=\linewidth]{pics/mult_lat_1d/position_[0.25]_2.png}
        \caption{Punkt w pozycji (0,25), 2 węzły}
        \label{pic:1d_mult_[0.25]_2}
    \end{subfigure}
\end{figure}
\begin{figure}[H]
    \ContinuedFloat\centering
    \begin{subfigure}{\textwidth}
        \centering
        \includegraphics[width=\linewidth]{pics/mult_lat_1d/position_[0.25]_4.png}
        \caption{Punkt w pozycji  (0,25), 4 węzły}
        \label{pic:1d_mult_[0.25]_4}
    \end{subfigure}
\end{figure}
\begin{figure}[H]
    \ContinuedFloat\centering
    \begin{subfigure}{\textwidth}
        \centering
        \includegraphics[width=\linewidth]{pics/mult_lat_1d/position_[-0.5]_2.png}
        \caption{Punkt w pozycji  (-0,5), 2 węzły}
        \label{pic:1d_mult_[-0.5]_2}
    \end{subfigure}
\end{figure}
\begin{figure}[H]
    \ContinuedFloat\centering
    \begin{subfigure}{\textwidth}
        \centering
        \includegraphics[width=\linewidth]{pics/mult_lat_1d/position_[-0.5]_4.png}
        \caption{Punkt w pozycji  (-0,5), 4 węzły}
        \label{pic:1d_mult_[-0.5]_4}
    \end{subfigure}
    \caption{Wykres obliczonej pozycji odbiornika w zależności od czasu}
    \label{fig:1d_mult_separate}
\end{figure}

\begin{figure}[H]
    \centering
    \begin{subfigure}{.5\textwidth}
        \centering
        \includegraphics[width=\linewidth]{pics/mult_lat_1d/positions_2_mean.png}
        \caption{2 węzły}
        \label{pic:1d_2_mult}
    \end{subfigure}%
    \begin{subfigure}{.5\textwidth}
        \centering
        \includegraphics[width=\linewidth]{pics/mult_lat_1d/positions_4_mean.png}
        \caption{4 węzły}
        \label{pic:1d_4_mult}
    \end{subfigure}
    \caption{Średnie obliczonych pozycji, wariant 1D}
    \label{fig:1d_mult}
\end{figure}

\subsubsection{Eksperymenty dwuwymiarowe}

Następnie sprawdzono system w wariancie dwuwymiarowym. Aby zachować podobną maksymalną odległość pomiędzy odbiornikami, zdecydowano na umiejscowienie czterech węzłów na wierzchołkach kwadratu o boku $0,6$ symetrycznie względem puntu $(0;0)$, a więc na punktach o współrzędnych $(\pm0,3; \pm0,3)$. Zbadano dokładność lokalizacji w trzech grupach punktów:

\begin{itemize}
    \item $\{(0;0), (0,3;0,3), (-0,3;0,3), (-0,3;-0,3), (0,3;-0,3)\}$
    \item $\{(0;0), (0;0,3), (0;-0,3), (-0,3;0), (0,3;0)\}$
    \item $\{(0;0), (0,15;0,15), (-0,15;0,15), (-0,15;-0,15), (0,15;-0,15)\}$
\end{itemize}
Wyniki przedstawiono na wykresach~\ref{fig:2d_mult}.

\begin{figure}[H]
    \centering
    \begin{subfigure}{.5\textwidth}
        \centering
        \includegraphics[width=\linewidth]{pics/mult_lat_2d/positions_1_mean.png}
        \caption{zestaw punktów 1.}
        \label{pic:2d_1_mult}
    \end{subfigure}%
    \begin{subfigure}{.5\textwidth}
        \centering
        \includegraphics[width=\linewidth]{pics/mult_lat_2d/positions_2_mean.png}
        \caption{zestaw punktów 2.}
        \label{pic:2d_2_mult}
    \end{subfigure}
\end{figure}
\begin{figure}[H]
    \ContinuedFloat\centering
    \begin{subfigure}{.5\textwidth}
        \centering
        \includegraphics[width=\linewidth]{pics/mult_lat_2d/positions_3_mean.png}
        \caption{zestaw punktów 3.}
        \label{pic:2d_3_mult}
    \end{subfigure}
    \caption{Średnie obliczonych pozycji, wariant 2D}
    \label{fig:2d_mult}
\end{figure}

Następnie w celu ewaluacji wpływu charakterystyki otoczenia na działanie systemu przeprowadzono trzy następujące bezpośrednio po sobie eksperymenty zachowując ustaloną na początku synchronizację zegarów i stałe korekcji odległości. Sprawdzono dokładność estymacji tego samego zbioru punktów wyjściowo oraz po rotacji całego układu o $45^{\circ}$ i $90^{\circ}$ zgodnie z ruchem wskazówek zegara. Wyniki przedstawiono na wykresach~\ref{fig:2d_angle_mult}. 

\begin{figure}[H]
    \centering
    \begin{subfigure}{.5\textwidth}
        \centering
        \includegraphics[width=\linewidth]{pics/mult_lat_2d_angle/positions_0_mean.png}
        \caption{rotacja $0^{\circ}$}
        \label{pic:2d_0_angle_mult}
    \end{subfigure}%
    \begin{subfigure}{.5\textwidth}
        \centering
        \includegraphics[width=\linewidth]{pics/mult_lat_2d_angle/positions_45_mean.png}
        \caption{rotacja $45^{\circ}$}
        \label{pic:2d_45_angle_mult}
    \end{subfigure}
\end{figure}
\begin{figure}[H]
    \ContinuedFloat\centering
    \begin{subfigure}{.5\textwidth}
        \centering
        \includegraphics[width=\linewidth]{pics/mult_lat_2d_angle/positions_90_mean.png}
        \caption{rotacja $90^{\circ}$}
        \label{pic:2d_90_angle_mult}
    \end{subfigure}
    \caption{Średnie obliczonych pozycji, wariant 2D z rotacją}
    \label{fig:2d_angle_mult}
\end{figure}

Ostatecznie sprawdzono zachowanie systemu wraz ze zmianą liczby węzłów odbiorczych od minimalnej liczby trzech aż do ośmiu. Węzły umiejscowiono na punktach ze zbioru $\{(0,3;0,3), (-0,3;0,3), (-0,3;-0,3), (0,3;-0,3), (0; 0,425), (-0,425; 0), (0; -0,425), (0,425; 0)\}$ oraz dodawano je zgodnie z tą kolejnością. Wyniki przedstawiono na wykresach~\ref{fig:2d_num_mult}. 

\begin{figure}[H]
    \centering
    \begin{subfigure}{.5\textwidth}
        \centering
        \includegraphics[width=\linewidth]{pics/mult_lat_2d_num/positions_3_mean.png}
        \caption{3 węzły}
        \label{pic:2d_3_num_mult}
    \end{subfigure}%
    \begin{subfigure}{.5\textwidth}
        \centering
        \includegraphics[width=\linewidth]{pics/mult_lat_2d_num/positions_4_mean.png}
        \caption{4 węzły}
        \label{pic:2d_4_num_mult}
    \end{subfigure}
\end{figure}
\begin{figure}[H]
    \ContinuedFloat\centering
    \begin{subfigure}{.5\textwidth}
        \centering
        \includegraphics[width=\linewidth]{pics/mult_lat_2d_num/positions_5_mean.png}
        \caption{5 węzłów}
        \label{pic:2d_5_num_mult}
    \end{subfigure}%
    \begin{subfigure}{.5\textwidth}
        \centering
        \includegraphics[width=\linewidth]{pics/mult_lat_2d_num/positions_6_mean.png}
        \caption{6 węzłów}
        \label{pic:2d_6_num_mult}
    \end{subfigure}
\end{figure}
\begin{figure}[H]
    \ContinuedFloat\centering
    \begin{subfigure}{.5\textwidth}
        \centering
        \includegraphics[width=\linewidth]{pics/mult_lat_2d_num/positions_7_mean.png}
        \caption{7 węzłów}
        \label{pic:2d_7_num_mult}
    \end{subfigure}%
    \begin{subfigure}{.5\textwidth}
        \centering
        \includegraphics[width=\linewidth]{pics/mult_lat_2d_num/positions_8_mean.png}
        \caption{8 węzłów}
        \label{pic:2d_8_num_mult}
    \end{subfigure}
    \caption{Średnie obliczonych pozycji, wariant 2D ze zmienną liczbą węzłów}
    \label{fig:2d_num_mult}
\end{figure}

\section{Interpretacja i wnioski}

Rozpoczynając od porównania wyników eksperymentu zerowego~\ref{chap:experiment_zero} z wynikami przypadku jednowymiarowego~\ref{sec:1d} można od razu zauważyć poprawę zarówno dokładności jak i stabilności pomiarów. W pierwszej kolejności zwróćmy uwagę na znacznie zmniejszenie przesunięcia obliczanych lokalizacji nadajnika względem jego prawdziwego położenia, które wynika z zastosowania synchronizacji sprzętowej~\ref{sec:mic_sync} biorącej pod uwagę wszelkie opóźnienia pomijane w przypadku synchronizacji sprzętowej~\ref{sec:prog_sync}. Ponadto dokładność wyników multilateracji nie ulega gwałtownemu spadkowi wraz z upływem czasu (wyjątkiem w poprawionych eksperymentach jest~\ref{pic:1d_mult_[-0.5]_4}, w którym zauważalna jest nagła zmiana średniej lokalizowanych punktów, czego przyczyną mogło być prawdopodobnie zmiana warunków otoczenia eksperymentu). Ostateczne wyniki multilateracji jednowymiarowej przedstawione na rysunku~\ref{fig:1d_mult} wskazują na zadowalającą (w zależności od zastosowań) dokładność pomiaru położenia nadajnika.

Przypadek jednowymiarowy jest jednak na tyle zdegenerowany, że do jego rozwiązania nie jest konieczne użycie metod multilateracyjnych, dlatego decydującym o użyteczności systemu było testowanie w wariancie dwuwymiarowym. Wyniki pierwszych testów~\ref{fig:2d_mult} nie były obiecujące {-} układ punktów względem siebie był w przybliżeniu zachowany, jednak punkty zwracane przez system znajdują się czasami w znacznej odległości od ich rzeczywistych położeń. Kolejne dwie serie testów, przeprowadzone w celu sprawdzenia jakie aspekty wpływają pozytywnie, a jakie negatywnie na dokładność lokalizacji wykazały dwie zależności:
\begin{itemize}
    \item podatność systemu na zmiany charakterystyki sonicznej otocznia~\ref{fig:2d_angle_mult};
    \item wzrost dokładności lokalizacji wraz ze wzrostem liczby węzłów odbiorczych~\ref{fig:2d_num_mult}.
\end{itemize}
Pierwsza z zależności jest wyraźnie widoczna, ponieważ wraz z obrotem błąd lokalizacji rośnie, jak również położenie punktów względem siebie ulega zmianie do tego stopnia, że trudno odgadnąć jaki kształt tworzyły w rzeczywistości. Druga zależność nie jest tak wydatna jak poprzednia, ponieważ błąd lokalizacji każdego z punktów nie jest ściśle malejący, czasem większa liczba węzłów nadawczych nie pomaga, a może nawet przeszkadzać (kolejny dodany węzeł mógł mieć niepoprawnie ustawioną czułość wzmacniacza, co znacznie wpłynęłoby na wyniki przy małej liczbie węzłów). Po zwiększeniu liczby do 7 i 8 widać zalety skali w takim systemie {-} nawet jeśli jakiś któraś z raportowanych odległości jest obarczona znaczącym błędem rozwiązanie aproksymacyjne minimalizujące sumę kwadratów błędów będzie bliższe punktowi, na który wskazują odległości podane przez pozostałe węzły.
