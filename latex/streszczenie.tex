\pdfbookmark[0]{Streszczenie}{streszczenie.1}
%\phantomsection
%\addcontentsline{toc}{chapter}{Streszczenie}
%%% Poniższe zostało niewykorzystane (tj. zrezygnowano z utworzenia nienumerowanego rozdziału na abstrakt)
%%%\begingroup
%%%\setlength\beforechapskip{48pt} % z jakiegoś powodu była maleńka różnica w położeniu nagłówka rozdziału numerowanego i nienumerowanego
%%%\chapter*{\centering Abstrakt}
%%%\endgroup
%%%\label{sec:abstrakt}
%%%Lorem ipsum dolor sit amet eleifend et, congue arcu. Morbi tellus sit amet, massa. Vivamus est id risus. Sed sit amet, libero. Aenean ac ipsum. Mauris vel lectus. 
%%%
%%%Nam id nulla a adipiscing tortor, dictum ut, lobortis urna. Donec non dui. Cras tempus orci ipsum, molestie quis, lacinia varius nunc, rhoncus purus, consectetuer congue risus. 
%\mbox{}\vspace{2cm} % można przesunąć, w zależności od długości streszczenia
\begin{abstract}
    Problem pozycjonowania w przestrzeni na podstawie emitowanego dźwięku obiektu pozycjonowanego wiąże się z wykorzystaniem możliwie zsynchronizowanych w czasie węzłów (mikrofonów) i pomiarze różnic czasu odbioru dźwięku przez czujniki. W pracy zostanie zbudowana sieć (co najmniej 4 sztuki) sensorów audio połączonych bezprzewodowo między sobą i ze stacją główną. Zadaniem sieci będzie wskazanie lokalizacji w przestrzeni punkowego przedmiotu emitującego dźwięk. Oprócz wyboru i implementacji algorytmu multilateracji zaproponowane zostanie rozwiązanie problemu synchronizacji czasu między sensorami, minimalizacji opóźnienia w komunikacji oraz kalibracji systemu.
\end{abstract}
\mykeywords{}

{
\selectlanguage{english}
\begin{abstract}
    The problem of positioning in space based on the emitted sound of the positioned object involves the use of as closely synchronized nodes (microphones) as possible in time and measuring the differences in the time of sound reception by sensors. In the work, a network (of at least 4 units) of audio sensors connected wirelessly to each other and to the main station will be built. The network's task will be to indicate the location in space of a point-like object emitting sound. In addition to selecting and implementing the multilateration algorithm, a solution to the problem of time synchronization between sensors, minimizing communication delay, and system calibration will be proposed.
\end{abstract}
\mykeywords{}
}
