\pdfbookmark[0]{Skróty}{skroty.1}% 
%%\phantomsection
%%\addcontentsline{toc}{chapter}{Skróty}
\chapter*{Skróty}\label{sec:skroty}
\noindent\vspace{-\topsep-\partopsep-\parsep} % Jeśli zaczyna się od otoczenia description, to otoczenie to ląduje lekko niżej niż wylądowałby zwykły tekst, dlatego wstawiano przesunięcie w pionie
\begin{description}[labelwidth=*]
  \item [WASN] (ang.\ \emph{Wireless Audio Sensor Networks}) %-- bezprzewodowe sieci sensorów audio
  \item [TOA] (ang.\ \emph{time of arrival})
  \item [AOA] (ang.\ \emhp{angle of arrival})
  \item [DOA] (ang.\ \emph{direction of arrival})
  \item [FDOA] (ang.\ \emph{frequency difference of arrival})
  \item [RSS] (ang.\ \emph{received signal strength})
  \item 
\end{description}
