\chapter{Eksperyment zerowy}

Po przygotowaniu komponentów systemu wstępnie zaimplementowano program rozwiązujący problem multilateracji, aby zbadać, czy problem nie jest zbyt trywialny, aby opisać go w pracy, lub przeciwnym razie, na podstawie wyników eksperymentu zastanowić się jakie przeszkody stoją na drodze do rozwiązania o zadowalającej precyzji.

\section{Opis działania}

Program zaimplementowano na podstawie rozwiązania aproksymacyjnego równania \ref{eq:matrix} postaci

\begin{equation}
    \hat{\boldsymbol{x}} = {\left(A^T A\right)}^{-1} A^T \boldsymbol{b}
\end{equation}
zaczerpniętego z artykułu~\cite{norrdine2012algebraic}.

\subsection{Program węzła}

\begin{algorithm}
\caption{Program nadajnika}\label{alg:source}
\begin{algorithmic}[1]
    \State\ $buzz \gets False$
    \State\ $buzzTime \gets 0$
    \State\ $lastBuzzTime \gets 0$

    \If{$beep\ \&\&\ micros() - lastBuzzTime > 500000$}
        \State\ $buzzTime \gets micros()$
        \State\ $publish(buzzTime)$
        \State\ $lastBuzzTime \gets buzzTime$
        \State\ $buzzer()$
    \EndIf
\end{algorithmic}
\end{algorithm}

\subsection{Program serwera}

\subsection{Opis algorytmu}

\section{Ewaluacja działania systemu}

\section{Interpretacja wyników i wnioski}