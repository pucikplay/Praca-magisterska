\chapter*{Wstęp}\label{chap:introduction}
\addcontentsline{toc}{chapter}{Wstęp}

Multilateracja jest metodą lokalizacji opartą na odległościach od punktów o znanych współrzędnych do szukanego punktu. Zazwyczaj odległości te uzyskuje się na podstawie czasu, w którym sygnał dotarł do odbiornika (\textit{ang.\ time of arrival, TOA}). Problem wyliczania pozycji punktu w przestrzeni jest kluczowy w wielu wpółczesnych serwisach lokalizacyjnych, takich jak GPS (\textit{ang.\ Global Positioning System}). Był on badany przynajmniej od lat 60., ze znacznymi postępami poczynionymi w następnej dekadzie podczas opracowywania systemu GPS, a w latach 90. w związku z błyskawicznym wzrostem liczby telefonów komórkowych i potrzebą lokalizacji osób komunikujących się przy ich użyciu z służbami ratunkowymi~\cite{govinfo}. W ostatnich latach coraz większym zainteresowaniem cieszy się zastosowanie metod multilateracji w sieciach urządzeń IoT~\cite{9184896} w celu autonomicznego uzyskiwania informacji lokalizacji przestrzennej bez potrzeby użycia dedykowanej do tego celu aparatury.

Na przestrzeni lat rozwijano wiele wariantów rozwiązań problemu lokalizacji na podstawie czasu otrzymania sygnału. Pierwszym były metody iteracyjne, dzięki którym w prosty sposób otrzymywano rozwiązanie równań nieliniowych, ale nie gwarantowały ani zbiegania do poprawnego rozwiązania, ani krótkiego czasu jego otrzymania. W latach 80. wprowadzono metody manipulacji układami równań tak, by otrzymać równanie kwadratowe jednej niewiadomej dające dwa rozwiązania, z których jedno było poprawne. Wreszcie w połowie lat 90. zaczęto używać układów równań liniowych, których rozwiązania otrzymywano dzięki programowaniu liniowemu. Zaletą tej metody jest jej wszechstronność, ponieważ daje możliwość łatwego rozszerzenia problemu o więcej pomiarów. Te dodatkowe informacje mogą być dodatkowymi  pomiarami TOA lub każdymi innymi związanymi z problemem przy pomocy równań liniowych. Dzięki temu mogły rozwinąć się algorytmy lokalizacji wykorzystujące kąt przybycia (\textit{ang.\ angle of arrival, AOA}), różnicę częstotliwości (\textit{ang.\ frequency difference of arrival, FDOA}), moc odebranego sygnału (\textit{ang.\ received signal strength,\ RSS}) oraz kierunek przybycia\footnote{Różnicą między DOA a AOA jest odległość, z której oodbierany jest sygnał. DOA zakłada, że źródło sygnału jest na tyle daleko, że AOA jest wspólny dla wszystkich odbiorników. W AOA różnice pomiarów są na tyle duże, aby móc użyć ich do triangulacji.} (\textit{ang.\ direction of arrival, DOA}). W poniższej pracy skupimy się na podstawowym wariancie multilateracji, czyli TOA.

Założeniem pracy jest zaprojektowanie i zaimplementowanie systemu multilateracyjnego działającego w domenie fal dźwiękowych, opartego o szeroko dostępne mikrokontrolery z rodziny \texttt{Arduino}. System będzie składał się z zsynchronizowanych ze sobą węzłów odbiorczych (mikrofonów) oraz węzła nadawczego (brzęczyka). W skład urządzeń systemowych będzie wchodził również centralny serwer obliczeniowy, który będzie wykonywał obliczenia multilateracyjne na podstawie odległości estymowanych przy pomocy różnicy czasu nadania i odebrania sygnału dźwiękowego przez każdy z węzłów. Zanim system będzie zdolny zwracać choćby w przybliżeniu poprawne lokalizacje, potrzebne będzie rozwiązanie problemów wynikających z nieadekwatnej synchronizacji zegarów urządzeń systemowych i nieoczekiwanych wpływów wzmacniacza operacyjnego, wbudowanego w zastosowane mikrofony. Każdy z tych podproblemów, jak również ostateczna multlilateracja, opatrzone będą odpowiednimi eksperymentami, których wyniki przestawione będą na wykresach.