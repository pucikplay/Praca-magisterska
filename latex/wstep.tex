\chapter*{Wstęp}\label{chap:introduction}

Multilateracja jest metodą lokalizacji opartą na odległościach od punktów o znanych współrzędnych do szukanego punktu aby określić jego położenie. Zazwyczaj odległości te uzyskuje się na podstawie czasu, w którym sygnał dotarł do odbiornika (\textit{ang.\ time of arrival, TOA}). Problem tego typu jest podstawowym w wielu wpółczesnych serwisach lokalizacyjnych, takich jak GPS (\textit{ang.\ Global Positioning System}) i był od badany przynajmniej od lat '60, ze znacznymi postępami poczynionymi w latach '70 podczas opracowywania systemu GPS, a w latach '90 w związku z błyskawicznym wzrostem liczby telefonów komórkowych i potrzebą lokalizacji osób komunikujących się przy ich użyciu z służbami ratunkowymi~\cite{govinfo}. W ostatnich latach coraz większym zainteresowaniem cieszy są zastosowanie metod multilateracji w sieciach urządzeń IoT~\cite{9184896} w celu autonomicznego uzyskiwania informacji lokalizacji przestrzennej bez potrzeby użycia dedykowanej do tego celu aparatury.

Na przestrzeni lat rozwijano wiele wariantów rozwiązań problemu lokalizacji na podstawie czasu otrzymania sygnału. Pierwszym były metody iteracyjne, dzięki którym w prosty sposób otrzymywano rozwiązanie równań nieliniowych, ale nie gwarantowały ani zbiegania do poprawnego rozwiązania, ani krótkiego czasu otrzymania go. W latach '80 wprowadzono metody manipulacji układami równań tak by otrzymać równanie kwadratowe jednej niewiadomej dające dwa rozwiązanie, z których jedno było poprawne. Wreszcie w połowie lat '90 zaczęto używać układów równań liniowych, których rozwiązanie otrzymywano dzięki programowaniu liniowemu. Zaletą tej metody jest jej wszechstronność, ponieważ daje możliwość łatwego rozszerzenia problemu o więcej pomiarów. Te dodatkowe informacje mogą być dodatkowymi  pomiarami TOA lub każdymi innymi związanymi z problemem przy pomocy równań liniowych. Dzięki temu mogły rozwinąć się algorytmy lokalizacji wykorzystujące kąt przybycia (\textit{ang.\ angle of arrival, AOA}), różnicę częstotliwości (\textit{ang.\ frequency difference of arrival, FDOA}), moc odebranego sygnału (\textit{ang.\ received signal strength,\ RSS}) oraz kierunek przybycia\footnote{Różnicą między DOA a AOA jest odległość, z której oodbierany jest sygnał. DOA zakłada, że źródło sygnału jest na tyle daleko, że AOA jest wspólny dla wszystkich odbiorników. W AOA różnice pomiarów są na tyle duże, aby móc użyć ich do triangulacji.} (\textit{ang.\ direction of arrival, DOA}). W poniższej pracy skupimy się na podstawowym wariancie multilateracji, czyli TOA.