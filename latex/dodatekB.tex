\chapter{Opis załączonej płyty CD/DVD}
\label{chap:opis-plyty}
Tutaj jest miejsce na zamieszczenie opisu zawartości załączonej płyty. Opis ten jest redagowany przed załadowaniem pracy do systemy APD USOS, a więc w chwili, gdy nieznana jest jeszcze nazwa, jaką system ten wygeneruje dla załadowanego pliku. Dlatego też redagując treść tego dodatku dobrze jest stosować ogólniki typu: ,,Na płycie zamieszczono dokument \texttt{pdf} z niniejszej tekstem pracy'' -- bez wskazywania nazwy tego pliku. 

Dawniej obowiązywała reguła, by nazywać dokumenty według wzorca \texttt{W04\_[nr albumu]\_[rok kalendarzowy]\_[rodzaj pracy]}, gdzie \texttt{rok kalendarzowy} odnosił się do roku realizacji kursu ,,Praca dyplomowa'', a nie roku obrony. Przykładowo wzorzec nazwy dla pracy dyplomowej inżynierskiej w konkretnym przypadku wyglądał tak: \texttt{W04\_123456\_2015\_praca inżynierska.pdf},  Takie nazwy utrwalane były w systemie składania prac dyplomowych. Obecnie działa to już inaczej.