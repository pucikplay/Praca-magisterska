\chapter{Sprzęt systemowy}\label{chap:hardware}

Znaczna większość prac adresujacych temat multilateracji opiera się na systemach urządzeń działających w zakresie fal elektromagnetycznych (\cite{murphy1995determination},\cite{norrdine2012algebraic},\cite{9254075}). W tej pracy poświęcimy uwagę systemowi działającemu w domenie dźwięku, jak ten aspekt wpływa na skuteczność i dokładność rozwiązania problemu multilateracji.

\begin{figure}[!h]
    \centering
    \begin{tikzpicture}[circlenode/.style={circle, draw=black}, node distance=3cm, thick]
        \node[circlenode] (S) {
            \begin{tikzpicture}[circlenode/.style={circle, draw=black}, node distance=5mm, thick]
                \node[circlenode] (B) {Broker};
                \node[circlenode] (P) [left=of B] {Serwer obliczeniowy};
                \node (T) [above=of P] {\textbf{Host}};
                \draw[<->,thick] (B) to (P);
            \end{tikzpicture}
        };
        \node[circlenode] (N) [right=of S] {Nadajnik};
        \node[circlenode] (O1) [above=of N] {Odbiornik 1.};
        \node[circlenode] (O2) [below=of N] {Odbiornik 2.};

        \draw[<->,thick] (N) to (S);
        \draw[<->,thick] (O1) to (S);
        \draw[<->,thick] (O2) to (S);
    \end{tikzpicture}
    \caption{Topologia systemu}
\label{fig:topology}
\end{figure}

\section{Serwer MQTT}

Urządzenia systemowe porozumiewają się przy uzyciu protokołu MQTT, każdy z węzłów oraz serwer łączą się z centralnym brokerem (?), który przekierowuje wiadomości do klientów, które zasubskrybowały dany temat.

\section{Węzeł}

Każdy z węzłów oparty jest o mikrokontroler ESP8266 zaprogramowany przy użyciu Arduino IDE. W systemie występują dwa rodzaje węzłów:

\begin{itemize}
    \item nadajnik,
    \item odbiornik.
\end{itemize}

Nadajnik jest wyposażony jest w przełącznik cewkowy sterowany przez mikrokontroler, który służy do kontrolowania brzęczyka zasilanego napięciem $12V$. Wybrano brzęczyk o głośności $90dB$ w celu zmaksymalizowania zasięgu działania systemu.

Odbiornik natomiast wyposażony jest w mikrofon elektretowy dającego binarny sygnał wyjściowy. Czułość mikrofonu dostrajana jest ręcznie poprzez potencjometr.

\section{Serwer obliczeniowy}

Centralnym urządzeniem systemu jest serwer obliczeniowy kumulujący dane otrzymane z sensorów do rozwiązania problemu multilateracji. Serwer hostowany jest na tej samej maszynie co broker MQTT.