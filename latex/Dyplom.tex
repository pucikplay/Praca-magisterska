\documentclass[a4paper,onecolumn,oneside,11pt,extrafontsizes]{memoir}
\usepackage[utf8]{inputenc}
\usepackage[T1]{fontenc}
\usepackage[english,polish]{babel}
\usepackage{polski}
\usepackage{setspace}
\usepackage{color,calc}
\usepackage{ebgaramond}
\usepackage{tgtermes}   
\renewcommand*\ttdefault{txtt}


%%%%%% pakiety
\usepackage{tabularx}
\usepackage{multicol}
\usepackage{printlen}
\usepackage{enumitem}
\usepackage{amssymb}
\usepackage{algorithm}
\usepackage{algpseudocode}
\makeatletter
\renewcommand{\ALG@name}{Program}
\makeatother
\usepackage{mathtools}
\uselengthunit{pt}
\makeatletter
\newcommand{\showFontSize}{\f@size{} pt} % makro wypisujące wielkość bieżącej czcionki
\makeatother
\usepackage{amsfonts}
\DeclareMathOperator{\rank}{rank}
\usepackage[pdftex]{graphicx}
\usepackage{tikz}
\usetikzlibrary{shapes.geometric}
\usetikzlibrary{positioning}
\usepackage{caption}
\usepackage{subcaption}

%%%%%%%%%%%%%%%%%%%%%%%%%%%%%%%%%%%%%%%%%%%%%%%%%%%%%%%%%%%%%%%%%%%%%%%%%%%%%%%%
%% Ustawienia odpowiedzialne za sposób łamania dokumentu
%% i ułożenie elementów pływających
%%%%%%%%%%%%%%%%%%%%%%%%%%%%%%%%%%%%%%%%%%%%%%%%%%%%%%%%%%%%%%%%%%%%%%%%%%%%%%%%
\clubpenalty=10000      % kara za sierotki
\widowpenalty=10000     % nie pozostawiaj wdów
\righthyphenmin=3			  % dziel minimum 3 litery
\renewcommand{\topfraction}{0.95}
\renewcommand{\bottomfraction}{0.95}
\renewcommand{\textfraction}{0.05}
\renewcommand{\floatpagefraction}{0.35}
%%%%%%%%%%%%%%%%%%%%%%%%%%%%%%%%%%%%%%%%%%%%%%%%%%%%%%%%%%%%%%%%%%%%%%%%%%%%%%%%
%%  Ustawienia rozmiarów: tekstu, nagłówka i stopki, marginesów
%%  dla dokumentów klasy memoir 
%%%%%%%%%%%%%%%%%%%%%%%%%%%%%%%%%%%%%%%%%%%%%%%%%%%%%%%%%%%%%%%%%%%%%%%%%%%%%%%%
\setlength{\headsep}{10pt} 
\setlength{\headheight}{15pt} % wartość baselineskip dla czcionki 11pt tj. \small wynosi 13.6pt
\setlength{\footskip}{\headsep+\headheight}
\setlength{\uppermargin}{\headheight+\headsep+1cm}
\setlength{\textheight}{\paperheight-\uppermargin-\footskip-1.5cm}
\setlength{\textwidth}{\paperwidth-5cm}
\setlength{\spinemargin}{2.5cm}
\setlength{\foremargin}{2.5cm}
\setlength{\marginparsep}{2mm}
\setlength{\marginparwidth}{2.3mm}
%\settrimmedsize{297mm}{210mm}{*}
%\settrims{0mm}{0mm}	
\checkandfixthelayout[fixed] % konieczne, aby się dobrze wszystko poustawiało
%%%%%%%%%%%%%%%%%%%%%%%%%%%%%%%%%%%%%%%%%%%%%%%%%%%%%%%%%%%%%%%%%%%%%%%%%%%%%%%%
%%  Ustawienia odległości linii, wcięć, odstępów
%%%%%%%%%%%%%%%%%%%%%%%%%%%%%%%%%%%%%%%%%%%%%%%%%%%%%%%%%%%%%%%%%%%%%%%%%%%%%%%%
\linespread{1}
%\linespread{1.241}
\setlength{\parindent}{14.5pt}
%%%%%%%%%%%%%%%%%%%%%%%%%%%%%%%%%%%%%%%%%%%%%%%%%%%%%%%%%%%%%%%%%%%%%%%%%%%%%%%%
%% Pakiet do wstawiania fragmentów kodu
%%%%%%%%%%%%%%%%%%%%%%%%%%%%%%%%%%%%%%%%%%%%%%%%%%%%%%%%%%%%%%%%%%%%%%%%%%%%%%%%
\usepackage{listings} 
\usepackage{xpatch}
\makeatletter
\xpatchcmd\l@lstlisting{1.5em}{0em}{}{}
\makeatother

\lstset{
  basicstyle=\small\ttfamily, % lub basicstyle=\footnotesize\ttfamily
  breaklines=true,
  postbreak=\mbox{\textcolor{red}{$\hookrightarrow$}\space}, 
	belowskip=.5\baselineskip,
	literate={\_}{{\_\allowbreak}}1 % ta deklaracja przydaje się, jeśli na listingu mają być łamane nazwy zawierające podkreślniki
}

\lstset{literate=%-
{ą}{{\k{a}}}1 {ć}{{\'c}}1 {ę}{{\k{e}}}1 {ł}{{\l{}}}1 {ń}{{\'n}}1 {ó}{{\'o}}1 {ś}{{\'s}}1 {ż}{{\.z}}1 {ź}{{\'z}}1 {Ą}{{\k{A}}}1 {Ć}{{\'C}}1 {Ę}{{\k{E}}}1 {Ł}{{\L{}}}1 {Ń}{{\'N}}1 {Ó}{{\'O}}1 {Ś}{{\'S}}1 {Ż}{{\.Z}}1 {Ź}{{\'Z}}1 
    {Ö}{{\"O}}1
    {Ä}{{\"A}}1
    {Ü}{{\"U}}1
    {ß}{{\ss}}1
    {ü}{{\"u}}1
    {ä}{{\"a}}1
    {ö}{{\"o}}1
    {~}{{\textasciitilde}}1
		{—}{{{\textemdash} }}1
}%{\ \ }{{\ }}1}

\definecolor{pblue}{rgb}{0.13,0.13,1}
\definecolor{pgreen}{rgb}{0,0.5,0}
\definecolor{pred}{rgb}{0.9,0,0}
\definecolor{pgrey}{rgb}{0.46,0.45,0.48}
\definecolor{dark-grey}{rgb}{0.4,0.4,0.4}
% styl json
\newcommand\JSONnumbervaluestyle{\color{blue}}
\newcommand\JSONstringvaluestyle{\color{red}}

\newif\ifcolonfoundonthisline{}

\makeatletter

\lstdefinestyle{json-style}  
{
	showstringspaces    = false,
	keywords            = {false,true},
	alsoletter          = 0123456789.,
	morestring          = [s]{''}{''},
	stringstyle         = \ifcolonfoundonthisline\JSONstringvaluestyle\fi,
	MoreSelectCharTable =%
	\lst@DefSaveDef{`:}\colon@json{\processColon@json},
	basicstyle          = \footnotesize\ttfamily,
	keywordstyle        = \ttfamily\bfseries,
	numbers				= left, % zakomentować, jeśli numeracja linii jest niepotrzebna
	numberstyle={\footnotesize\ttfamily\color{dark-grey}},
	xleftmargin			= 2em % zakomentować, jeśli numeracja linii jest niepotrzebna
}

\newcommand\processColon@json{%
	\colon@json%
	\ifnum\lst@mode=\lst@Pmode%
	\global\colonfoundonthislinetrue%
	\fi
}

\lst@AddToHook{Output}{%
	\ifcolonfoundonthisline%
	\ifnum\lst@mode=\lst@Pmode%
	\def\lst@thestyle{\JSONnumbervaluestyle}%
	\fi
	\fi
	\lsthk@DetectKeywords% 
}

\lst@AddToHook{EOL}%
{\global\colonfoundonthislinefalse}

\makeatother
%%%%%%%%%%%%%%%%%%%%%%%%%%%%%%%%%%%%%%%%%%%%%%%%%%%%%%%%%%%%%%%%%%%%%%%%%%%%%%%%
%%  Formatowanie list wyliczeniowych, wypunktowań i własnych otoczeń
%%%%%%%%%%%%%%%%%%%%%%%%%%%%%%%%%%%%%%%%%%%%%%%%%%%%%%%%%%%%%%%%%%%%%%%%%%%%%%%%
\usepackage{enumitem} % pakiet pozwalający zarządzać formatowaniem list wyliczeniowych
\setlist{noitemsep,topsep=4pt,parsep=0pt,partopsep=4pt,leftmargin=*} % zadeklarowane parametry pozwalają uzyskać 'zwartą' postać wypunktowania bądź wyliczenia
\setenumerate{labelindent=0pt,itemindent=0pt,leftmargin=!,label=\arabic*.} % można zmienić \arabic na \alph, jeśli wyliczenia mają być z literkami
\setlistdepth{4} % definiujemy głębokość zagnieżdżenia list wyliczeniowych do 4 poziomów
\setlist[itemize,1]{label=$\bullet$}  % definiujemy, jaki symbol ma być użyty w wyliczeniu na danym poziomie
\setlist[itemize,2]{label=\normalfont\bfseries\textendash}
\setlist[itemize,3]{label=$\ast$}
\setlist[itemize,4]{label=$\cdot$}
\renewlist{itemize}{itemize}{4}
\makeatletter
\renewenvironment{quote}{
	\begin{list}{}
	{
	\setlength{\leftmargin}{1em}
	\setlength{\topsep}{0pt}%
	\setlength{\partopsep}{0pt}%
	\setlength{\parskip}{0pt}%
	\setlength{\parsep}{0pt}%
	\setlength{\itemsep}{0pt}
	}
	}{
	\end{list}}
\makeatother

%%%%%%%%%%%%%%%%%%%%%%%%%%%%%%%%%%%%%%%%%%%%%%%%%%%%%%%%%%%%%%%%%%%%%%%%%%%%%%%%
%%  Pakiet i komendy do generowania indeksu 
%% (ważne, by pojawiły się przed pakietem hyperref)
%%%%%%%%%%%%%%%%%%%%%%%%%%%%%%%%%%%%%%%%%%%%%%%%%%%%%%%%%%%%%%%%%%%%%%%%%%%%%%%%
% pdftex jest w stanie wygenerować indeks (czyli spis haseł z referencjami do stron, na których te hasła się pojawiły).
% Generalnie z indeksem jest sporo problemów, zwłaszcza, gdy pojawiają się polskie literki.
% Trzeba wtedy korzystać z xindy.
% Zwykle w pracach dyplomowych indeksy nie są wykorzystywane. Dlatego są zamarkowane.
%\DisemulatePackage{imakeidx}
%\usepackage[makeindex,noautomatic]{imakeidx} % tutaj mówimy, żeby indeks nie generował się automatycznie, 
%\makeindex
%
%\makeatletter
%%%%\renewenvironment{theindex}
							 %%%%{\vskip 10pt\@makeschapterhead{\indexname}\vskip -3pt%
								%%%%\@mkboth{\MakeUppercase\indexname}%
												%%%%{\MakeUppercase\indexname}%
								%%%%\vspace{-3.2mm}\parindent\z@%
								%%%%\renewcommand\subitem{\par\hangindent 16\p@ \hspace*{0\p@}}%%
								%%%%\phantomsection%
								%%%%\begin{multicols}{2}
								%%%%%\thispagestyle{plain}
								%%%%\parindent\z@                
								%%%%%\parskip\z@ \@plus .3\p@\relax
								%%%%\let\item\@idxitem}
							 %%%%{\end{multicols}\clearpage}
%%%%
%\makeatother




%%%%%%%%%%%%%%%%%%%%%%%%%%%%%%%%%%%%%%%%%%%%%%%%%%%%%%%%%%%%%%%%%%%%%%%%%%%%%%%%
%%  Sprawy metadanych w wynikowym pdf, hyperlinków itp.
%%%%%%%%%%%%%%%%%%%%%%%%%%%%%%%%%%%%%%%%%%%%%%%%%%%%%%%%%%%%%%%%%%%%%%%%%%%%%%%%
\usepackage{ifpdf}
\ifpdf{}
 \usepackage{datetime2} % INFO: pakiet potrzeby do uzyskania i sformatowania daty 
 \usepackage[pdftex,bookmarks,breaklinks,unicode]{hyperref}
 \usepackage{hyperxmp}
 \DeclareGraphicsExtensions{.pdf,.jpg,.mps,.png} % po zadeklarowaniu rozszerzeń można będzie wstawiać pliki z grafiką bez konieczności podawania tych rozszerzeń w ich nazwach
\pdfcompresslevel=9
\pdfoutput=1

\makeatletter
\AtBeginDocument{  
  \hypersetup{
	pdfinfo={
    Title = {\@title},
    Author = {\@author},
    Subject={Praca dyplomowa \ifMaster{} magisterska\else inżynierska\fi},  
    Keywords={\@kvpl}, 
		Producer={}, 
	  CreationDate= {}, % należy wstawiać zgodnie ze składnią: {D:yyyymmddhhmmss}, np. D:20210208175600
    ModDate={\pdfcreationdate},   % data modyfikacji będzie datą kompilacji
		Creator={pdftex},
	}}
}
\pdftrailerid{} %Remove ID
\pdfsuppressptexinfo15 %Suppress PTEX.Fullbanner and info of imported PDFs
\makeatother
\else             % jeśli kompilacja jest inna niż pdflatex
\usepackage{graphicx}
\usepackage{tikz}
\usetikzlibrary{positioning}
\DeclareGraphicsExtensions{.eps,.ps,.jpg,.mps,.png}
\fi
\sloppy

% INFO: dodane by lepiej łamać urle 
\def\UrlBreaks{\do\/\do-\do_}


%%%%%%%%%%%%%%%%%%%%%%%%%%%%%%%%%%%%%%%%%%%%%%%%%%%%%%%%%%%%%%%%%%%%%%%%%%%%%%%%
%%  Formatowanie dokumentu
%%%%%%%%%%%%%%%%%%%%%%%%%%%%%%%%%%%%%%%%%%%%%%%%%%%%%%%%%%%%%%%%%%%%%%%%%%%%%%%%
% INFO: Deklaracja głębokościu numeracji
\setcounter{secnumdepth}{2}
\setcounter{tocdepth}{2}
\setsecnumdepth{subsection} 
% INFO: Dodanie kropek po numerach sekcji
\makeatletter
\def\@seccntformat#1{\csname the#1\endcsname.\quad}
\def\numberline#1{\hb@xt@\@tempdima{#1\if&#1&\else.\fi\hfil}}
\makeatother
% INFO: Numeracja rozdziałów i separatory
\renewcommand{\chapternumberline}[1]{#1.\quad}
\renewcommand{\cftchapterdotsep}{\cftdotsep}

\makeatletter % odstępy w spisie pomiędzy rozdziałami
\renewcommand*{\insertchapterspace}{%
  \addtocontents{lof}{\protect\addvspace{3pt}}%
  \addtocontents{lot}{\protect\addvspace{3pt}}%
	\addtocontents{toc}{\protect\addvspace{3pt}} %
  \addtocontents{lol}{\protect\addvspace{3pt}}}
\makeatother 


\setlength{\cftbeforechapterskip}{0pt} % odstępy w spisie treści przed rozdziałem, działa w korelacji z:
\renewcommand{\aftertoctitle}{\afterchaptertitle\vspace{-4pt}} % 
\captionnamefont{\small}
\captiontitlefont{\small}


% INFO: Sformatowanie podpisu nad dwukolumnowym listingiem
\newcommand{\listingcaption}[1]
{%
\vspace*{\abovecaptionskip}\small 
\refstepcounter{lstlisting}\hfill%
Listing \thelstlisting: #1\hfill%\hfill%
\addcontentsline{lol}{lstlisting}{\protect\numberline{\thelstlisting}#1}
}%



% INFO: Pomocnicze marko do wyróżniania tekstu w języku angielskim
\newcommand{\eng}[1]{(ang.~\emph{#1})}
% IFNO: Pomocnicze makro do dołączania podpisów do rysunków ze wskazaniem źródła (bez wypisywania tego źródła w spisie rysunków)
\newcommand*{\captionsource}[2]{%
  \caption[{#1}]{%
    #1 \emph{Źródło:} #2%
  }%
}

\addto\captionspolish{\renewcommand{\tablename}{Tab.}} 
\addto\captionspolish{\renewcommand{\figurename}{Rys.}}
\addto\captionspolish{\renewcommand{\lstlistlistingname}{Spis listingów}}
\newlistof{lstlistoflistings}{lol}{\lstlistlistingname}
\addto\captionspolish{\renewcommand{\bibname}{Literatura}}
\addto\captionspolish{\renewcommand{\listfigurename}{Spis rysunków}}
\addto\captionspolish{\renewcommand{\listtablename}{Spis tabel}}
\addto\captionspolish{\renewcommand\indexname{Indeks rzeczowy}}

\renewcommand{\abstractnamefont}{\normalfont\Large\bfseries}
\renewcommand{\abstracttextfont}{\normalfont}


%%%%%%%%%%%%%%%%%%%%%%%%%%%%%%%%%%%%%%%%%%%%%%%%%%%%%%%%%%%%%%%%%%%%%%%%%%%%%%%%
%% Definicje stopek i nagłówków
%%%%%%%%%%%%%%%%%%%%%%%%%%%%%%%%%%%%%%%%%%%%%%%%%%%%%%%%%%%%%%%%%%%%%%%%%%%%%%%%
\addtopsmarks{headings}{%
\nouppercaseheads{} % added at the beginning
}{%
\createmark{chapter}{both}{shownumber}{}{.\ \space}
%\createmark{chapter}{left}{shownumber}{}{.\ \space}
\createmark{section}{right}{shownumber}{}{.\ \space}
}%use the new settings

\makeatletter
\copypagestyle{outer}{headings}
\makeoddhead{outer}{}{}{\small\itshape\rightmark\/}
\makeevenhead{outer}{\small\itshape\leftmark\/}{}{}
\makeoddfoot{outer}{\small\@author:~\@titleShort}{}{\small\thepage}
\makeevenfoot{outer}{\small\thepage}{}{\small\@author:~\@title}
\makeheadrule{outer}{\linewidth}{\normalrulethickness}
\makefootrule{outer}{\linewidth}{\normalrulethickness}{2pt}
\makeatother

% fix plain
\copypagestyle{plain}{headings} % overwrite plain with outer
\makeoddhead{plain}{}{}{} % remove right header
\makeevenhead{plain}{}{}{} % remove left header
\makeevenfoot{plain}{}{}{}
\makeoddfoot{plain}{}{}{}

\copypagestyle{empty}{headings} % overwrite plain with outer
\makeoddhead{empty}{}{}{} % remove right header
\makeevenhead{empty}{}{}{} % remove left header
\makeevenfoot{empty}{}{}{}
\makeoddfoot{empty}{}{}{}

% INFO: deklaracja zmiennej logicznej wykorzystywanej do rozróżnienia pracy inżynierskiej i magisterskiej
\newif\ifMaster{}
\Mastertrue{}

%%%%%%%%%%%%%%%%%%%%%%%%%%%%%%%%%%%%%%%%%%%%%%%%%%%%%%%%%%%%%%%%%%%%%%%%%%%%%%%%
%% Definicja strony tytułowej 
%%%%%%%%%%%%%%%%%%%%%%%%%%%%%%%%%%%%%%%%%%%%%%%%%%%%%%%%%%%%%%%%%%%%%%%%%%%%%%%%
\makeatletter
%Uczelnia
\newcommand\uczelnia[1]{\renewcommand\@uczelnia{#1}}
\newcommand\@uczelnia{}
%Wydział
\newcommand\wydzial[1]{\renewcommand\@wydzial{#1}}
\newcommand\@wydzial{}
%Kierunek
\newcommand\kierunek[1]{\renewcommand\@kierunek{#1}}
\newcommand\@kierunek{}
%Specjalność
\newcommand\specjalnosc[1]{\renewcommand\@specjalnosc{#1}}
\newcommand\@specjalnosc{}
%Tytuł po angielsku
\newcommand\titleEN[1]{\renewcommand\@titleEN{#1}}
\newcommand\@titleEN{}
%Tytuł krótki
\newcommand\titleShort[1]{\renewcommand\@titleShort{#1}}
\newcommand\@titleShort{}
%Promotor
\newcommand\promotor[1]{\renewcommand\@promotor{#1}}
\newcommand\@promotor{}
%Słowa kluczowe
\newcommand\kvpl[1]{\renewcommand\@kvpl{#1}}
\newcommand\@kvpl{}
\newcommand\kven[1]{\renewcommand\@kven{#1}}
\newcommand\@kven{}
%Komenda wykorzystywana w streszczeniu
\newcommand\mykeywords{\hspace{\absleftindent}%
\parbox{\linewidth-2.0\absleftindent}{
       \iflanguage{polish}{\textbf{Słowa kluczowe:} \@kvpl}{%
			 \iflanguage{english}{\textbf{Keywords:} \@kven}}{}}
				}

\def\maketitle{%
  \pagestyle{empty}%
%%\garamond 
	\fontfamily{\ebgaramond@family}\selectfont % na stronie tytułowej czcionka garamond
%%%%%%%%%%%%%%%%%%%%%%%%%%%%%%%%%%%%%%%%%%%%%%%%%%%%%%%%%%%%%%%%%%%%%%%%%%%%%%	
%% Poniżej, w otoczniu picture, wstawiono tytuł i autora. 
%% Tytuł (z autorem) musi znaleźć się w obszarze 
%% odpowiadającym okienku 110mmx75mm, którego lewy górny róg 
%% jest w położeniu 77mm od lewej i 111mm od górnej  krawędzi strony 
%% (tak wynika z wycięcia na okładce). 
%% Poniższy kod musi być użyty dokładnie w miejscu gdzie jest.
%% Jeśli tytuł nie mieści się w okienku, to należy tak pozmieniać 
%% parametry użytych komend, aby ten przydługi tytuł jednak 
%% upakować do okienka.
%%
%% Sama okładka (kolorowa strona z wycięciem, kiedyś była do pobrania z dydaktyki) 
%% powinna być przycięta o 3mm od każdej z krawędzi.
%% Te 3mm pewnie zostawiono na ewentualne spady czy też specjalną oprawę.
%%%%%%%%%%%%%%%%%%%%%%%%%%%%%%%%%%%%%%%%%%%%%%%%%%%%%%%%%%%%%%%%%%%%%%%%%%%%%%
\newlength{\tmpfboxrule}
\setlength{\tmpfboxrule}{\fboxrule}
\setlength{\fboxsep}{2mm}
\setlength{\fboxrule}{0mm} 
%\setlength{\fboxrule}{0.1mm} %% INFO: Jeśli chcemy zobaczyć ramkę, wystarczy odmarkować tę linijkę
\setlength{\unitlength}{1mm}
\begin{picture}(0,0)
%\put(26,-124){\fbox{% ustawienie do "wyciętego okienka"
\put(20,-124){\fbox{% ustawienie na środku
\parbox[c][71mm][c]{104mm}{\centering%\lineskip=34pt 
{\fontsize{18pt}{20pt}\bfseries\selectfont \@title}\\[5mm]
{\fontsize{18pt}{20pt}\bfseries\selectfont \@titleEN}\\[10mm] % INFO: wstawiono tytuł w języku angielskim, choć w obecnych oficjalnych zaleceniach tego nie ma
%\fontsize{16pt}{18pt}\selectfont AUTOR:\\[2mm]
{\fontsize{16pt}{18pt}\selectfont \@author}}
}
}
\end{picture}
\setlength{\fboxrule}{\tmpfboxrule} 
%%%%%%%%%%%%%%%%%%%%%%%%%%%%%%%%%%%%%%%%%%%%%%%%%%%%%%%%%%%%%%%%%%%%%%%%%%%%%%
%% Reszta strony z nazwą uczelni, wydziału, kierunkiem, specjalnością
%% promotorem, oceną pracy (zakomentowane), miastem i rokiem
	{\vskip 9pt\centering
		{\fontsize{20pt}{22pt}\bfseries\selectfont \@uczelnia}\\[5pt]
		{\fontsize{16pt}{18pt}\bfseries\selectfont \@wydzial}\\[1pt]
		  \hrule
	}
{\vskip 24pt\raggedright\fontsize{14pt}{16pt}\selectfont%
\begin{tabular}{@{}ll}
Kierunek: & {\bfseries \@kierunek}\\
%Specjalność: & {\bfseries \@specjalnosc}\\
\end{tabular}\\[1.3cm]
}
{\vskip 29pt\centering{\fontsize{24pt}{26pt}\selectfont%
{\fontsize{26pt}{28pt}\selectfont P}RACA {\fontsize{26pt}{24pt}\selectfont D}YPLOMOWA\\[7pt]
\ifMaster\selectfont{\fontsize{26pt}{28pt}\selectfont M}AGISTERSKA\\[2.5cm]%
\else \selectfont{\fontsize{26pt}{28pt}\selectfont I}NŻYNIERSKA\\[2.5cm]\fi
}}
	\vfill
{\centering
		{\fontsize{14pt}{16pt}\selectfont Opiekun pracy}\\[2mm] 
		{\fontsize{14pt}{16pt}\bfseries\selectfont \@promotor}\\[10mm]%INFO: tutaj wstawiane ejst nazwisko promotora
%		&{\fontsize{16pt}{18pt}\selectfont OCENA PRACY:}\\[20mm] 
% INFO: linię powyższą zakomentowano, gdyż od czasu pandemii COVID-19 prace mogą być dostarczane bez podpisu promotora
}
\vspace{4cm}\noindent
{\fontsize{12pt}{14pt}\selectfont Słowa kluczowe: \@kvpl}% INFO: na stronę tytułową trafiają tylko słowa kluczowe w języku polskim (w jakim napisana jest praca)
\vspace{1.3cm}
\hrule\vspace*{0.3cm}
{\centering
{\fontsize{14pt}{16pt}\selectfont \@date}\\[0cm]
}
%\ungaramond
\normalfont{}
 \cleardoublepage{}
}
\makeatother

%%%%%%%%%%%%%%%%%%%%%%%%%%%%%%%%%%%%%%%%%%%%%%%%%%%%%%%%%%%%%%%%%%%%%%%%%%%%%%%%%%
\Mastertrue
\title{Mechanizm multilateracji w rozproszonej sieci sensorów audio} % INFO: tytuł pracy w języku polskim 
\titleShort{Metody multilateracji $\ldots$}  % INFO: krótki tytuł pracy (do zamieszczenia w stopce, sklejony z imieniem i nazwiskiem autora nie powinien zająć więcej niż jedną linijkę)
\titleEN{Multilateration mechanism in distributed net of audio sensors} % INFO: tytuł pracy w języku angielskim
\author{Gabriel Budziński}  % INFO: imię i nazwisko autora
\uczelnia{Politechnika Wrocławska} % INFO: nazwa uczelni
\wydzial{Wydział Informatyki i Telekomunikacji} % INFO: nazwa wydziału
\kierunek{Informatyka algorytmiczna (INA)} % INFO: nazwa kierunku
\specjalnosc{Informatyka algorytmiczna (INA)} % INFO: nazwa specjalności
\promotor{dr inż. Przemysław Błaśkiewicz} % INFO: dane promotora 
\kvpl{multilateracja, sensory audio, synchronizacja czasu} % INFO: słowa kluczowe po polsku
\kven{multilateration, WASN, clock synchronization} % INFO: słowa kluczowe po angielsku
\date{WROCŁAW, 2024} % INFO: miejscowość, rok złożenia pracy dyplomowej

%%%%%%%%%%%%%%%%%%%%%%%%%%%%%%%%%%%%%%%%%%%%%%%%%%%%%%%%%%%%%%%%%%%%%%%%%%%%%%%%%%
%%
%%  Struktura dokumentu
%%  - tutaj należy wstawić własne rozdziały
%%
%%%%%%%%%%%%%%%%%%%%%%%%%%%%%%%%%%%%%%%%%%%%%%%%%%%%%%%%%%%%%%%%%%%%%%%%%%%%%%%%%%

%%%%%%%%%%%%%%%%%%%%%%%%%%%%%%%%%%%%%%%%%%%%%%%%%%%%%%%%%%%%%%%%%%%%%%%%%%%%%%%%%%
% INFO: Za pomocą polecenia \includeonly{} można dokonać selekcji  
%       tych części (plików z latexowym kodem), które mają być kompilowane. 
%       Przydaje się to szczególnie podczas pracy nad dużymi dokumentami. 
%       Bo im mniej części zostanie wyselekcjonowanych, tym szybsza będzie kompilacja.
%       Proszę nie mylić tej komendy z poleceniem \include{}, którą używa się 
%       do zadeklarowania pełnej struktury dokumentu (plików z latexowym kodem).
%\includeonly{skroty,rozdzial01}  

\begin{document}

\pdfbookmark[0]{Tytuł}{Tytul.1}
\maketitle
\clearpage

\pdfbookmark[0]{Streszczenie}{streszczenie.1}
%\phantomsection
%\addcontentsline{toc}{chapter}{Streszczenie}
%%% Poniższe zostało niewykorzystane (tj. zrezygnowano z utworzenia nienumerowanego rozdziału na abstrakt)
%%%\begingroup
%%%\setlength\beforechapskip{48pt} % z jakiegoś powodu była maleńka różnica w położeniu nagłówka rozdziału numerowanego i nienumerowanego
%%%\chapter*{\centering Abstrakt}
%%%\endgroup
%%%\label{sec:abstrakt}
%%%Lorem ipsum dolor sit amet eleifend et, congue arcu. Morbi tellus sit amet, massa. Vivamus est id risus. Sed sit amet, libero. Aenean ac ipsum. Mauris vel lectus. 
%%%
%%%Nam id nulla a adipiscing tortor, dictum ut, lobortis urna. Donec non dui. Cras tempus orci ipsum, molestie quis, lacinia varius nunc, rhoncus purus, consectetuer congue risus. 
%\mbox{}\vspace{2cm} % można przesunąć, w zależności od długości streszczenia
\begin{abstract}
    Problem pozycjonowania w przestrzeni na podstawie emitowanego dźwięku obiektu pozycjonowanego wiąże się z wykorzystaniem możliwie zsynchronizowanych w czasie węzłów (mikrofonów) i pomiarze różnic czasu odbioru dźwięku przez czujniki. W pracy zostanie zbudowana sieć (co najmniej 4 sztuki) sensorów audio połączonych bezprzewodowo między sobą i ze stacją główną. Zadaniem sieci będzie wskazanie lokalizacji w przestrzeni punkowego przedmiotu emitującego dźwięk. Oprócz wyboru i implementacji algorytmu multilateracji zaproponowane zostanie rozwiązanie problemu synchronizacji czasu między sensorami, minimalizacji opóźnienia w komunikacji oraz kalibracji systemu.
\end{abstract}
\mykeywords

{
\selectlanguage{english}
\begin{abstract}

\end{abstract}
\mykeywords
}

\pagestyle{outer}
\clearpage
% SPIS TREŚCI (zostanie wygenerowany automatycznie)
\pdfbookmark[0]{Spis treści}{spisTresci.1}%
%%\phantomsection
%%\addcontentsline{toc}{chapter}{Spis treści}
\setsecnumdepth{section}
\settocdepth{section}
\tableofcontents* 
\clearpage
% SPIS RYSUNKÓW (zostanie wygenerowany automatycznie)
% \pdfbookmark[0]{Spis rysunków}{spisRysunkow.1} % jeśli chcemy mieć w spisie treści, to zamarkować tę linię, a odmarkować linie poniższe
%%\phantomsection
%%\addcontentsline{toc}{chapter}{Spis rysunków}
% \listoffigures*
% \clearpage
% SPIS TABEL (zostanie wygenerowany automatycznie)
% \pdfbookmark[0]{Spis tabel}{spisTabel.1} %
%%\phantomsection
%%\addcontentsline{toc}{chapter}{Spis tabel}
% \listoftables*
% \clearpage
% SPIS LISTINGÓW (zostanie wygenerowany automatycznie)
% \pdfbookmark[0]{Spis listingów}{spisListingow.1} %
%%\phantomsection
%%\addcontentsline{toc}{chapter}{Spis listingów}
% \lstlistoflistings*
% \clearpage
% SKRÓTY (to opcjonalna część pracy)
% \pdfbookmark[0]{Skróty}{skroty.1}% 
%%\phantomsection
%%\addcontentsline{toc}{chapter}{Skróty}
\chapter*{Skróty}\label{sec:skroty}
\noindent\vspace{-\topsep-\partopsep-\parsep} % Jeśli zaczyna się od otoczenia description, to otoczenie to ląduje lekko niżej niż wylądowałby zwykły tekst, dlatego wstawiano przesunięcie w pionie
\begin{description}[labelwidth=*]
  \item [WASN] (ang.\ \emph{Wireless Audio Sensor Networks}) %-- bezprzewodowe sieci sensorów audio
  \item [TOA] (ang.\ \emph{time of arrival})
  \item [AOA] (ang.\ \emhp{angle of arrival})
  \item [DOA] (ang.\ \emph{direction of arrival})
  \item [FDOA] (ang.\ \emph{frequency difference of arrival})
  \item [RSS] (ang.\ \emph{received signal strength})
  \item
\end{description}
 
% ROZDZIAŁY (kolejne rozdziały dołączane są z kolejnych plików)
\chapterstyle{default}
\chapter*{Wstęp}\label{chap:introduction}

Multilateracja jest metodą lokalizacji

Bajeczka o GPS, namierzaniu telefonów~\cite{govinfo} i IoT~\cite{9184896}.
\chapter{Przedstawienie problemu}

Multilateracja jest techniką lokalizacji pozwalającą obliczyć nieznane koordynaty punktu na podstawie odległości od innych, znanych punktów. Weźmy dwuwymiarowy egzemplarz naszego problemu (Rys.~\ref{fig:example}), gdzie $N$ - nadajnik, $O_i$ - odbiorniki, $d_i$ - odległości

\begin{figure}[!h]
    \centering
    \begin{tikzpicture}
        \coordinate (O1) at (0,0);
        \coordinate (O2) at (3,5);
        \coordinate (O3) at (8,1);
        \coordinate (N) at (4,3);
        \filldraw[black] (O1) circle (2pt) node[left]{$O_1$};
        \filldraw[black] (O2) circle (2pt) node[right]{$O_2$};
        \filldraw[black] (O3) circle (2pt) node[right]{$O_3$};
        \filldraw[black] (N) circle (2pt) node[above right]{$N$};
        \draw[black, thin] (O1) -- node[above left]{$d_1$} (N);
        \draw[black, thin] (O2) -- node[left]{$d_2$} (N);
        \draw[black, thin] (O3) -- node[above]{$d_3$} (N);
    \end{tikzpicture}
    \caption[short]{Egzemplarz problemu multilateracji}
\label{fig:example}
\end{figure}

Znalezienie koordynatów $(x,y)$ punktu N jest równoważne z rowiązaniem układu równań,

\begin{equation}
    \begin{dcases}
        {(x - x_1)}^2 + {(y - y_1)}^2 = {d_1}^2\\
        {(x - x_2)}^2 + {(y - y_2)}^2 = {d_2}^2\\
        {(x - x_3)}^2 + {(y - y_3)}^2 = {d_3}^2\\
    \end{dcases}
\end{equation}

który może zostać przekształcony do postaci

\begin{equation}
    \begin{dcases}
        (x^2 + y^2) - 2x_1x - 2y_1y = d_1^2 - x_1^2 - y_1^2\\
        (x^2 + y^2) - 2x_2x - 2y_2y = d_2^2 - x_2^2 - y_2^2\\
        (x^2 + y^2) - 2x_3x - 2y_3y = d_3^2 - x_3^2 - y_3^2\\
    \end{dcases}
\end{equation}

lub w reprezenacji macierzowej,

\begin{equation}
    \left[
        \begin{matrix}
            1 & -2x_1 & -2y_1\\
            1 & -2x_2 & -2y_2\\
            1 & -2x_3 & -2y_3\\
        \end{matrix}
    \right]
    \left[
        \begin{matrix}
            x^2 + y^2\\
            x\\
            y\\
        \end{matrix}
    \right]
    =
    \left[
        \begin{matrix}
            d_1^2 - x_1^2 - y_1^2\\
            d_2^2 - x_2^2 - y_2^2\\
            d_3^2 - x_3^2 - y_3^2\\
        \end{matrix}
    \right]
\end{equation}

którą można przedstawić jako

\begin{equation}
    \boldsymbol{A} \cdot \boldsymbol{x} = \boldsymbol{b}
\end{equation}

Uogólniona forma równania macierzowego problemu multilateracji dla przestrzeni $n$-wymiarowej i $m$ odbiorników:

\begin{equation}
    \left[
        \begin{matrix}
            1 & -2\boldsymbol{x^{(1)}}\\
            1 & -2\boldsymbol{x^{(2)}}\\
            \vdots & \vdots\\
            1 & -2\boldsymbol{x^{(m)}}\\
        \end{matrix}
    \right]
    \left[
        \begin{matrix}
            \sum_{i=1}^{n}{x_i}^2\\
            x_1\\
            \vdots\\
            x_n
        \end{matrix}
    \right]
    =
    \left[
        \begin{matrix}
            d_1^2 - \sum_{i=1}^{n}{x_i^{(1)}}^2\\
            d_2^2 - \sum_{i=1}^{n}{x_i^{(2)}}^2\\
            \vdots\\
            d_m^2 - \sum_{i=1}^{n}{x_i^{(m)}}^2\\
        \end{matrix}
    \right]
\end{equation}


\section{State of the art}

Napisać coś o~\cite{murphy1995determination},\cite{norrdine2012algebraic}

\chapter{Sprzęt systemowy}

\section{Węzeł}

\section{Serwer MQTT}

\section{Serwer obliczniowy}

\chapter{Eksperyment zerowy}

\section{Opis działania}

\subsection{Program węzła}

\subsection{Program serwera}

\subsection{Opis algorytmu}

\section{Ewaluacja działania systemu}

\section{Interpretacja wyników i wnioski}
\chapter{Synchronizacja czasu}\label{chap:time_sync}

Synchronizacja czasu jest jednym z podstawowych problemów systemów opartych o wiele urządzeń posiadających własne zegary. Wiedza o dokładnym czasie zachodzących w sieci zdarzeń jest niezbędna do szybkiego przesyłu danych, koordynacji procesów czy aktualizacji systemu plików. W przypadku niniejszej pracy dokładne określenie czasu zachodzących zdarzeń jest kluczowe do wiarygodnego określenia odległości pomiędzy węzłami na podstawie interwału czasowego pomiędzy nadaniem a odbiorem sygnału dźwiękowego.

\section{Synchronizacja programowa}

Pierwszym podejściem do rozwiązania problemu synchronizacji zegarów było zastosowanie synchronizacji programowej~\cite{6066334}, w której urządzenie-host utrzymuje wysokiej rozdzielczości licznik czasowy i wysyła sygnały o jego wartości pozostałym węzłom w sieci. Na podstawie tych informacji każdy z węzłów oblicza różnicę w zegarach i wprowadza odpowiednie przesunięcie własnego licznika.

\subsection{Algorytm synchronizacji NTP}\label{sec:ntp_sync}

Jednym z najbardziej rozpowszechnionych protokołów synchronizacji programowej jest \textit{Network Time Protocol} opisany w pracy~\cite{103043}. Zachowując oznaczenia w niej zastosowane opiszmy w skrócie zasadę działania tego protokołu.

Na rysunku~\ref{fig:ntp} przestawiono schemat działania protokołu NTP. Urządzenia \textit{A} i \textit{B} wymieniają wiadomości zawierające sygnatury czasowe. Niech $T_{i},\ T_{i-1},\ T_{i-2},\ T_{i-3}$ będą czterema ostatnimi wiadomościami oraz niech $a = T_{i-2} - T_{i-3}$ oraz $b = T_{i-1} - T_i$. Wtedy całkowity czas transmisji $\delta_i$ i przesunięcie zegara $\theta_i$ urządzenia \textit{B} względem urządzenia \textit{A} to

\[\delta_i = a - b\quad \text{oraz}\quad \theta_i = \frac{a+b}{2}\]

Dodatkowym wnioskiem przedstawionym w powyższej pracy jest własność prawdziwego przesunięcia względem aktualnie obliczonego:

\[\theta_i - \frac{\delta_i}{2} \leqslant \theta \leqslant \theta_i + \frac{\delta_i}{2}\]

Jak łatwo zauważyć im krótszy jest czas propagacji tym lepsze przybliżenie dostajemy.

\begin{figure}[H]
\centering
\resizebox{1\textwidth}{!}{%
\begin{tikzpicture}
    \tikzstyle{every node}=[font=\Huge]
    \draw [line width=1pt] (10,24.25) -- (36.25,24.25);
    \draw [line width=1pt] (10,15.5) -- (36.25,15.5);
    \draw [line width=1pt, ->] (12.5,15.5) -- (18.75,24.25);
    \draw [line width=1pt, ->] (27.5,24.25) -- (33.75,15.5);
    \draw [line width=1pt] (20,15.5) -- (20,21.75);
    \draw [line width=1pt] (26.25,24.25) -- (26.25,18);
    \draw [line width=1pt, <->] (20,19.75) -- (26.25,19.75)node[pos=0.5, fill=white]{$\theta$};
    \node [font=\Huge] at (12.5,14.75) {$T_{i-3}$};
    \node [font=\Huge] at (18.75,25) {$T_{i-2}$};
    \node [font=\Huge] at (27.5,24.75) {$T_{i-1}$};
    \node [font=\Huge] at (34,14.75) {$T_i$};
    \node [font=\Huge] at (23,24.75) {$B$};
    \node [font=\Huge] at (23,14.75) {$A$};
\end{tikzpicture}
}%
\caption{Pomiar opóźnienia transmisji i przesunięcia zegara}
\label{fig:ntp}
\end{figure}

Na postawie tych informacji podjęto próbę synchronizacji programowej zegarów węzłów w systemie multilateracyjnym. Przy użyciu czterech węzłów przeprowadzono eksperymenty mające na celu próbę ustalenia przesunięć zegarów każdego z urządzeń względem centralnego serwera. Opisany wyżej schemat wymiany czterech wiadomości powtarzano $n$ razy wspólnie dla wszystkich węzłów zapisując obliczone przesunięcia oraz czasy propagacji.

W celu poprawy czytelności wyniki przesunięć zostały znormalizowane poprzez przesunięcie o wartość średnią dla każdego z węzłów.

\begin{figure}[H]
\centering
    \includegraphics[width=\textwidth]{pics/ntp_sync/offsets.png}
\caption{Wyniki pomiarów przesunięć zegarów}
\label{pic:offsets_ntp}
\end{figure}

\begin{figure}[H]
\centering
    \includegraphics[width=\textwidth]{pics/ntp_sync/prop_times.png}
\caption{Wyniki pomiarów czasów propagacji}
\label{pic:prop_times}
\end{figure}

\begin{figure}[H]
\centering
    \includegraphics[width=.49\textwidth]{pics/ntp_sync/stddev_offsets.png}
    \includegraphics[width=.49\textwidth]{pics/ntp_sync/stddev_prop.png}
\caption{Odchylenia standardowe pomiarów}
\label{pic:stddev_ntp}
\end{figure}

Na podstawie otrzymanych wykresów łatwo zauważyć, że czas propagacji wiadomości w systemie jest nieprzewidywalny i zmienia sie znacząco z pomiaru na pomiar. Interesującą nas statystyką są jednak zaobserwowane przesunięcia zegarów, bez których nie jesteśmy w stanie poprawnie ocenić odległości od źródła dźwięku. Tutaj wartości wyglądają na bardziej skoncentrowane, jednak nie widać wyraźnych tendencji koncentracji wokół wartości średnich dla żadnego z węzłów. Ponadto zaobserwowane odchylenia standardowe $\sigma_i,\ i \in \{0,1,2,3\}$ wielkości $\approx 7000 \mu s$ są nieakceptowalne, ponieważ w takim czasie dźwięk w powietrzu pokonuje $\frac{7000}{1000000}s \cdot 343\frac{m}{s} = 2.401m$. Możliwe jest, że wielokrotne powtarzanie pomiarów da zadowalającą wartość średnią, pozwalającą na centymetrową precyzję obliczanych odległości. W porównaniu z pozostałymi sposobami brane będą pod uwagę wartości uśrednione.

\subsection{Bezpośredni pomiar przesunięć zegarów}\label{sec:time_deltas_sync}

Biorąc pod uwagę zauważoną nieprzewidywalność i rozrzut czasów propagacji (spowodowanych najprawdopodobniej użyciem protokołu MQTT do przesyłu wiadomości pomiędzy urządzeniami) następnym pomysłem schematu synchronizacji jest bezpośrednie badanie względnego przesunięcia zegarów. Węzeł wysyła $n$ wiadomości zawierających aktualną wartość zegara, która po odebraniu przez serwer jest porównywana z zegarem w nim dostępnym.

\begin{figure}[H]
\centering
    \includegraphics[width=\textwidth]{pics/time_deltas/time_deltas.png}
\caption{Wyniki pomiarów przesunięć zegarów}
\label{pic:offsets_deltas}
\end{figure}

\begin{figure}[H]
\centering
    \includegraphics[width=.49\textwidth]{pics/time_deltas/stddev.png}
\caption{Odchylenia standardowe pomiarów}
\label{pic:stddev_deltas}
\end{figure}

Wykresy przesunięć wygenerowane na podstawie tych testów na pierwszy rzut oka są skoncentrowane podobnie jak poprzednie, jednakże odchylenie standardowe są prawie dwukrotnie mniejsze niż uprzednio, co daje nadzieje na bardziej wiarygodne wyniki. W porównaniu z pozostałymi sposobami brane będą pod uwagę wartości uśrednione.

\section{Synchronizacja sprzętowa}



\subsection{Synchronizacja z użyciem mikrofonów}\label{sec:mic_sync}

Innym pomysłem na synchronizację zegarów w węzłach było użycie mikrofonów, w które węzły odbiorcze są wyposażone. Potrzebujmy znać jedynie przesunięcie naszego zegara względem zegara w węźle nadawczym dlatego wystarczającym będzie porównanie czasu nadania i odebrania sygnału dźwiękowego. Ponadto ten rodzaj synchronizacji w przeciwieństwie do synchronizacji programowej, która uzgadniała ze sobą jedynie zegary na podstawie wymienianych wiadomości, wlicza w czas transmisji wszelkie nie wzięte wcześniej pod uwagę opóźnienia, takie jak:

\begin{itemize}
    \item Czas pomiędzy wysłaniem wiadomości o nadaniu sygnału a zamknięciem kontaktora i poruszeniem membraną brzęczyka,
    \item Czas pomiędzy odebraniem sygnału przez mikrofon a zmianą stanu zmiennej na to wskazującej.
\end{itemize}

Przeprowadzono testy tego typu synchronizacji, których wyniki przedstawiono na wykresach~\ref{pic:mic_sync}. W celu zwiększenia czytelności odrzucono pierwszy z pomiarów oraz przesunięto wyniki o wartość średnią.

\begin{figure}[H]
\centering
    \includegraphics[width=\textwidth]{pics/mic_sync/offsets.png}
\caption{Wyniki pomiarów przesunięć zegarów}
\label{pic:mic_sync}
\end{figure}

\begin{figure}[H]
\centering
    \includegraphics[width=.49\textwidth]{pics/mic_sync/stddev_offsets.png}
\caption{Odchylenia standardowe pomiarów}
\label{pic:stddev_mic}
\end{figure}

Łatwo zauważyć, ża udało się zmniejszyć odchylenia obliczonych przesunięć zegara aż o dwa rzędy wielkości. Taka dokładność daje znacznie lepsze przybliżenie rzeczywistej odległości między węzłami, ponieważ w czasie $40 \mu s$ dźwięk pokona jedynie $\frac{40}{10000000}s \cdot 343 \frac{m}{s} \approx 0,014m$. Mając tak dokładne odległości będziemy mogli wprowadzić je do modelu multilateracyjnego.

\subsection{Porównanie metod}

Porównajmy teraz dokładność obliczanych odległości na podstawie interwału czasowego pomiędzy nadaniem dźwięku a jego odbiorem w różnych wariantach synchronizacji zegarów. Wykresy uzyskane przy zastosowaniu synchronizacji czasowej zostały przedstawione przed i po znormalizowaniu poprzez przesunięcie tak, by średnia pomiarów rozpoczynała się od 0. Pomiary oparte o synchronizację z użyciem mikrofonów nie wymagały tego dodatkowego kroku.

\begin{figure}[H]
\centering
    \includegraphics[width=.49\textwidth]{pics/ntp_sync_dist/dists.png}
    \includegraphics[width=.49\textwidth]{pics/ntp_sync_dist/dists_close.png}
\caption{Wyniki pomiarów odległości z użyciem synchronizacji~\ref{sec:ntp_sync}}
\label{pic:ntp_sync_dist}
\end{figure}

\begin{figure}[H]
\centering
    \includegraphics[width=.49\textwidth]{pics/time_deltas_dist/dists.png}
    \includegraphics[width=.49\textwidth]{pics/time_deltas_dist/dists_close.png}
\caption{Wyniki pomiarów odległości z użyciem synchronizacji~\ref{sec:time_deltas_sync}}
\label{pic:time_deltas_dist}
\end{figure}

\begin{figure}[H]
\centering
    \includegraphics[width=.49\textwidth]{pics/mic_sync_dist/dists.png}
    \includegraphics[width=.49\textwidth]{pics/mic_sync_dist/dists_close.png}
\caption{Wyniki pomiarów odległości z użyciem synchronizacji~\ref{sec:mic_sync}}
\label{pic:mic_sync_dist}
\end{figure}

Na wykresach widać, że wszystkie trzy algorytmy synchronizacji po normalizacji dają wyniki o podobnej dokładności adekwatnej do użycia w danych wejściowych multilateracji. Ciekawą jest natomiast obserwacja skalowania obliczonych odległości. Widzimy, że w przypadku~\ref{pic:ntp_sync_dist} oraz~\ref{pic:mic_sync_dist} otrzymane wartości leżą blisko trzykrotności rzeczywistej badanej odległości, natomiast w~\ref{pic:time_deltas_dist} blisko dwukrotności. W kolejnym rozdziale pochylimy się nad możliwymi przyczynami i rozwiązaniem tego zachowania.
\chapter{Metody multilateriacji}

\section{Omówienie zastosowanych metod}

\subsection{Układ równań liniowych}

\subsection{Liniowa metoda najmniejszej sumy kwadratów}

\subsection{Nieliniowa metoda najmniejszej sumy kwadratów}

\subsection{Rozkład według wartości osobliwych (SVD)}

\section{Wyniki}

\subsection{Interpretacja}

\subsection{Wnioski}
\chapter*{Podsumowanie}\label{chap:podsumowanie}
\addcontentsline{toc}{chapter}{Podsumowanie}

Cel pracy niniejszej pracy było zaprojektowanie i zaimplementowanie systemu multilateracyjnego działającego w domenie fal dźwiękowych. W ramach pracy dokonano wyboru metody multilateracji najlepiej nadającej się do tego zastosowania, jak również przeprowadzono szereg testów wykazujących nieprawidłowości działania, które były rektyfikowane. Podczas eksperymentów ukazano znaczenie wpływu zastosowanej metody synchronizacji zegarów urządzeń w systemie, jak również interesujące skalowanie estymowanej na podstawie różnic zegarów odległości wynikającej z zastosowanych w mikrofonach wzmacniaczy operacyjnych.

Wnioski z przeprowadzonych badań wskazują, że system multilateracyjny oparty o szeroko dostępne moduły mikrofonowe jest zdolny śledzić pozycję nadajnika sygnałów dźwiękowych w podzbiorze przestrzeni dwuwymiarowej obejmowanym ograniczonym zasięgiem tychże mikrofonów w węzłach odbiorczych. Uzyskane wyniki są istotne z punktu widzenia systemów lokalizacji bazujących na falach dźwiękowych, szczególnie w kontekście ich zastosowań w technologii IoT. Wykazano, że dostępne komercyjnie moduły mikrofonowe mogą być wykorzystane do tworzenia efektywnych i stosunkowo tanich systemów lokalizacji.

Dalszymi krokami rozwoju systemu, których nie podjęto ze względu na ograniczony budżet czasowy mogłyby być:

\begin{itemize}
    \item rozszerzenie do i ewaluacja działania w przestrzeni trójwymiarowej,
    \item bardziej wnikliwe testy i badania roli urządzeń przetwarzających sygnał dźwiękowy na cyfrowy; inwestygacja wpływu typu rodzaju mikrofonu, wzmacniacza czy przetwornika ADC (\textit{ang.\ analog to digital}) na estymację odległości,
    \item stworzenie lepszego algorytmu korekcji estymowanej odległości względem rzeczywistej,
    \item ewentualne dostosowywanie algorytmów synchronizacji zegarów do przyszłych zastosowań,
    \item próba minimalizacji wpływu charakterystyki dźwiękowej otoczenia na dokładność wyników.
\end{itemize}

Podsumowując, zaprojektowany i zaimplementowany system multilateracyjny działający w domenie fal dźwiękowych udowodnił swoją skuteczność w śledzeniu pozycji nadajnika w dwuwymiarowej przestrzeni, otwierając tym samym drogę do dalszych badań i udoskonaleń, które mogą znacząco poszerzyć jego zastosowanie w różnych dziedzinach.

\bibliographystyle{plabbrv}

\setlength{\bibitemsep}{2pt}

\addtocontents{toc}{\addvspace{2pt}}

\bibliography{bibliografia}
\appendix

% Jeśli w pracy pojawiać się ma indeks, należy odkomentować poniższe linie
%%\chapterstyle{noNumbered}
%%\phantomsection % sets an anchor
%%\addcontentsline{toc}{chapter}{Indeks rzeczowy}
%%\printindex

\end{document}
