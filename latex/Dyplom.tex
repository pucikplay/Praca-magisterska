\documentclass[a4paper,onecolumn,oneside,12pt,extrafontsizes]{memoir}
%  W celu przygotowania wydruku do archiwum można:
%  a) przygotować pdf, w którym dwie strony zostaną wstawione na jedną fizyczną stronę i taki dokument wydrukować dwustronnie (podejście zalecane)
%
%   Taki dokument można przygotować poprzez
%   - wydruk z Adobe Acrobat Reader z opcją "Wiele" - sekcja "Rozmiar i obsługa stron"
%   - wykorzystanie narzędzi psutils
%
%      Windows (zakładając, że w dystrybucji MiKTeX jest pakiet miktex-psutils-bin-x64-2.9):
%        "c:\Program Files\MiKTeX 2.9\miktex\bin\x64\pdf2ps.exe" Dyplom.pdf Dyplom.ps
%        "c:\Program Files\MiKTeX 2.9\miktex\bin\x64\psnup.exe" -2 Dyplom.ps Dyplom2.ps
%        "c:\Program Files\MiKTeX 2.9\miktex\bin\x64\ps2pdf.exe" Dyplom2.ps Dyplom2.pdf
%        Del Dyplom2.ps Dyplom.ps
%
%     Linux:
%        pdf2ps Dyplom.pdf - | psnup -2 | ps2pdf - Dyplom2.pdf
%
%  b) przekomplilować dokument zmniejszając czcionkę (podejście niezalecane, bo zmienia formatowanie dokumentu)
%
%    Do tego wystarczy posłużyć się poniższymi komendami (zamiast documentclass z pierwszej linijki):
%   \documentclass[a4paper,onecolumn,twoside,10pt]{memoir} 
%   \renewcommand{\normalsize}{\fontsize{8pt}{10pt}\selectfont}

%\usepackage[cp1250]{inputenc} % Proszę zostawić, jeśli kodowanie edytowanych plików to cp1250 
\usepackage[utf8]{inputenc} % Proszę użyć zamiast powyższego, jeśli kodowanie edytowanych plików to UTF8
\usepackage[T1]{fontenc}
\usepackage[english,polish]{babel} % Tutaj ważna jest kolejność atrybutów (dla pracy po polsku polish powinno być na końcu)
%\DisemulatePackage{setspace}
\usepackage{setspace}
\usepackage{color,calc}
%\usepackage{soul} % pakiet z komendami do podkreślania, przekreślania, podświetlania tekstu (raczej niepotrzebny)
\usepackage{ebgaramond} % pakiet z czcionkami garamond, potrzebny tylko do strony tytułowej, musi wystąpić przed pakietem tgtermes

%% Aby uzyskać polskie literki w pdfie (a nie zlepki) korzystamy z pakietu czcionek tgterms. 
%% W pakiecie tym są zdefiniowane klony czcionek Times o kształtach: normalny, pogrubiony, italic, italic pogrubiony.
%% W pakiecie tym brakuje czcionki o kształcie: slanted (podobny do italic). 
%% Jeśli w dokumencie gdzieś zostanie zastosowana czcionka slanted (np. po użyciu komendy \textsl{}), to
%% latex dokona podstawienia na czcionkę standardową i zgłosi to w ostrzeżeniu (warningu).
%% Ponadto tgtermes to czcionka do tekstu. Wszelkie matematyczne wzory będą sformatowane domyślną czcionką do wzorów.
%% Jeśli wzory mają być sformatowane z wykorzystaniem innych czcionek, trzeba to jawnie zadeklarować.

%% Po zainstalowaniu pakietu tgtermes może będzie trzeba zauktualizować informacje 
%% o dostępnych fontach oraz mapy. Można to zrobić z konsoli (jako administrator)
%% initexmf --admin --update-fndb
%% initexmf --admin --mkmaps

\usepackage{tgtermes}   
\renewcommand*\ttdefault{txtt}


%%%%%%%%%%%%%%%%%%%%%%%%%%%%%%%%%%%%%%%%%%%%%%%%%%%%%%%%%%%%%%%%%%%%%%%%%%%%%%%%
%% Ustawienia odpowiedzialne za sposób łamania dokumentu
%% i ułożenie elementów pływających
%%%%%%%%%%%%%%%%%%%%%%%%%%%%%%%%%%%%%%%%%%%%%%%%%%%%%%%%%%%%%%%%%%%%%%%%%%%%%%%%
%\hyphenpenalty=10000		% nie dziel wyrazów zbyt często
\clubpenalty=10000      % kara za sierotki
\widowpenalty=10000     % nie pozostawiaj wdów
%\brokenpenalty=10000		% nie dziel wyrazów między stronami - trzeba było wyłączyć, bo nie łamały się linie w lstlisting
%\exhyphenpenalty=999999		% nie dziel słów z myślnikiem - trzeba było wyłączyć, bo nie łamały się linie w lstlisting
\righthyphenmin=3			  % dziel minimum 3 litery

%\tolerance=4500
%\pretolerance=250
%\hfuzz=1.5pt
%\hbadness=1450

\renewcommand{\topfraction}{0.95}
\renewcommand{\bottomfraction}{0.95}
\renewcommand{\textfraction}{0.05}
\renewcommand{\floatpagefraction}{0.35}

%%%%%%%%%%%%%%%%%%%%%%%%%%%%%%%%%%%%%%%%%%%%%%%%%%%%%%%%%%%%%%%%%%%%%%%%%%%%%%%%
%%  Ustawienia rozmiarów: tekstu, nagłówka i stopki, marginesów
%%  dla dokumentów klasy memoir 
%%%%%%%%%%%%%%%%%%%%%%%%%%%%%%%%%%%%%%%%%%%%%%%%%%%%%%%%%%%%%%%%%%%%%%%%%%%%%%%%
\setlength{\headsep}{10pt} 
\setlength{\headheight}{15pt} % wartość baselineskip dla czcionki 11pt tj. \small wynosi 13.6pt
\setlength{\footskip}{\headsep+\headheight}
\setlength{\uppermargin}{\headheight+\headsep+1cm}
\setlength{\textheight}{\paperheight-\uppermargin-\footskip-1.5cm}
\setlength{\textwidth}{\paperwidth-5cm}
\setlength{\spinemargin}{2.5cm}
\setlength{\foremargin}{2.5cm}
\setlength{\marginparsep}{2mm}
\setlength{\marginparwidth}{2.3mm}
%\settrimmedsize{297mm}{210mm}{*}
%\settrims{0mm}{0mm}	
\checkandfixthelayout[fixed] % konieczne, aby się dobrze wszystko poustawiało
%%%%%%%%%%%%%%%%%%%%%%%%%%%%%%%%%%%%%%%%%%%%%%%%%%%%%%%%%%%%%%%%%%%%%%%%%%%%%%%%
%%  Ustawienia odległości linii, wcięć, odstępów
%%%%%%%%%%%%%%%%%%%%%%%%%%%%%%%%%%%%%%%%%%%%%%%%%%%%%%%%%%%%%%%%%%%%%%%%%%%%%%%%
\linespread{1}
%\linespread{1.241}
\setlength{\parindent}{14.5pt}


\usepackage{multicol} % pakiet umożliwiający stworzenie wielokolumnowego tekstu
%%%%%%%%%%%%%%%%%%%%%%%%%%%%%%%%%%%%%%%%%%%%%%%%%%%%%%%%%%%%%%%%%%%%%%%%%%%%%%%%
%% Pakiety do formatowania tabel
%%%%%%%%%%%%%%%%%%%%%%%%%%%%%%%%%%%%%%%%%%%%%%%%%%%%%%%%%%%%%%%%%%%%%%%%%%%%%%%%
\usepackage{tabularx}
% Proszę używać tylko tabularx. Innych pakietów proszę nie stosować !!!
% Dokument na pewno da się zredagować bez ich użycia.
%\usepackage{longtable}
%\usepackage{ltxtable}
%\usepackage{tabulary}

%%%%%%%%%%%%%%%%%%%%%%%%%%%%%%%%%%%%%%%%%%%%%%%%%%%%%%%%%%%%%%%%%%%%%%%%%%%%%%%%
%% Pakiet do wstawiania fragmentów kodu
%%%%%%%%%%%%%%%%%%%%%%%%%%%%%%%%%%%%%%%%%%%%%%%%%%%%%%%%%%%%%%%%%%%%%%%%%%%%%%%%
\usepackage{listings} 
\usepackage{xpatch}
\makeatletter
\xpatchcmd\l@lstlisting{1.5em}{0em}{}{}
\makeatother
% Pakiet dostarcza otoczenia lstlisting. Jest ono wysoce konfigurowalne. 
% Konfigurować można indywidualnie każdy z listingów lub globalnie, w poleceniu \lstset{}.

% Zalecane jest, by kod źródłowy był wyprowadzany z użyciem czcionki maszynowej \ttfamily
% Ponieważ kod źródłowy, nawet po obcięciu do interesujących fragmentów, bywa obszerny, należy zmniejszyć czcionkę.
% Zalecane jest \small (dla krótkich fragmentów) oraz \footnotesize (dla dłuższych fragmentów).

% Ponadto podczas konfiguracji można zadeklarować sposób numerowania linii. Numerowanie linii zalecane jest jednak 
% tylko w przypadkach, gdy w redagowanym tekście znajdują się jakieś odwołania do konkretnych linii.
% Jeśli takich odwołań nie ma, numerowanie linii jest zbędne. Proszę wtedy go nie stosować.
% Przy włączaniu numerowania linii należy zwrócić uwagę na to, gdzie pojawią się te numery.
% Bez zmiany dodatkowych parametrów pojawiają się one na marginesie strony (co jest niepożądane).

\lstset{
  basicstyle=\small\ttfamily, % lub basicstyle=\footnotesize\ttfamily
  %%columns=fullflexible,
	%%showstringspaces=false,
	%%showspaces=false,
  breaklines=true,
  postbreak=\mbox{\textcolor{red}{$\hookrightarrow$}\space}, 
  %%numbers=left,  % ta i poniższe linie dotyczą ustawienia numerowania i sposobu jego wyprowadzania
  %%firstnumber=1, 
  %%numberfirstline=true, 
	%%xleftmargin=17pt,
  %%framexleftmargin=17pt,
  %%framexrightmargin=5pt,
  %%framexbottommargin=4pt,
	belowskip=.5\baselineskip,
	literate={\_}{{\_\allowbreak}}1 % ta deklaracja przydaje się, jeśli na listingu mają być łamane nazwy zawierające podkreślniki
}

% Jeśli edytowany plik nie jest w kodowaniu cp1250, to jest problem z polskimi znakami występującymi we wstawianym kodzie.
% Dlatego podczas pracy na plikach w kodowaniu UTF8 trzeba zadeklarować mapowanie jak niżej (wystarczy odmarkować).
% Niestety, jak się zastosuje to mapowanie mogą pojawić się problemy z podświetlaniem składni (patrz dalej).
\lstset{literate=%-
{ą}{{\k{a}}}1 {ć}{{\'c}}1 {ę}{{\k{e}}}1 {ł}{{\l{}}}1 {ń}{{\'n}}1 {ó}{{\'o}}1 {ś}{{\'s}}1 {ż}{{\.z}}1 {ź}{{\'z}}1 {Ą}{{\k{A}}}1 {Ć}{{\'C}}1 {Ę}{{\k{E}}}1 {Ł}{{\L{}}}1 {Ń}{{\'N}}1 {Ó}{{\'O}}1 {Ś}{{\'S}}1 {Ż}{{\.Z}}1 {Ź}{{\'Z}}1 
    {Ö}{{\"O}}1
    {Ä}{{\"A}}1
    {Ü}{{\"U}}1
    {ß}{{\ss}}1
    {ü}{{\"u}}1
    {ä}{{\"a}}1
    {ö}{{\"o}}1
    {~}{{\textasciitilde}}1
		{—}{{{\textemdash} }}1
}%{\ \ }{{\ }}1}


%% lstlisting pozwala na ostylowania podświetlania składni wybranych języków.
%% Działa to na zasadzie zdefiniowania słów kluczowych oraz sposobu ich wyświetlania.
%% Ponieważ jest to prosty mechanizm, czasem trudno osiągnąć takie efekty, jakie dają narzędzia IDE. 
%% Jednak w większości przypadku osiągane rezutlaty są zadowalające.


%% lstlisting obsługuje domyślnie kilka najpopularniejszych języków.
%%\lstloadlanguages{% Check Dokumentation for further languages ...
%%C,
%%C++,
%%csh,
%%Java
%%}
%% Inne języki muszą być dodefiniowane. Poniżej podano przykłady definicji języków i styli.

\definecolor{lightgray}{rgb}{.9,.9,.9}
\definecolor{darkgray}{rgb}{.4,.4,.4}
\definecolor{purple}{rgb}{0.65, 0.12, 0.82}
\definecolor{javared}{rgb}{0.6,0,0} % for strings
\definecolor{javagreen}{rgb}{0.25,0.5,0.35} % comments
\definecolor{javapurple}{rgb}{0.5,0,0.35} % keywords
\definecolor{javadocblue}{rgb}{0.25,0.35,0.75} % javadoc
 
\lstdefinelanguage{JavaScript}{ 
	keywords={typeof, new, true, false, catch, function, return, null, catch, switch, var, if, in, while, do, else, case, break},
	keywordstyle=\color{blue}\bfseries,
	ndkeywords={class, export, boolean, throw, implements, import, this},
	ndkeywordstyle=\color{darkgray}\bfseries,
	identifierstyle=\color{black},
	sensitive=false,
	comment=[l]{//},
	morecomment=[s]{/*}{*/},
	commentstyle=\color{purple}\ttfamily,
	stringstyle=\color{red}\ttfamily,
	morestring=[b]',
	morestring=[b]''
}
\lstdefinestyle{JavaScriptStyle}{
	language=JavaScript,
	commentstyle=\color{javagreen}, % niestety, jeśli w linii komentarza pojawią się słowa kluczowe, to zostaną pokolorowane
	backgroundcolor=,%\color{lightgray}, % można ustwić kolor tła, ale jest to niezalecane
	extendedchars=true,
	basicstyle=\footnotesize\ttfamily,
	showstringspaces=false,
	showspaces=false,
	numbers=none,%left,
	numberstyle=\footnotesize,
	numbersep=9pt,
	tabsize=2,
	breaklines=true,
	showtabs=false,
	captionpos=t
}

\lstdefinestyle{JavaStyle}{
basicstyle=\footnotesize\ttfamily,
keywordstyle=\color{javapurple}\bfseries,
stringstyle=\color{javared},
commentstyle=\color{javagreen},
morecomment=[s][\color{javadocblue}]{/**}{*/},
numbers=none,%left,
numberstyle=\tiny\color{black},
stepnumber=2,
numbersep=10pt,
tabsize=4,
showspaces=false,
showstringspaces=false,
captionpos=t
}

\definecolor{pblue}{rgb}{0.13,0.13,1}
\definecolor{pgreen}{rgb}{0,0.5,0}
\definecolor{pred}{rgb}{0.9,0,0}
\definecolor{pgrey}{rgb}{0.46,0.45,0.48}
\definecolor{dark-grey}{rgb}{0.4,0.4,0.4}
% styl json
\newcommand\JSONnumbervaluestyle{\color{blue}}
\newcommand\JSONstringvaluestyle{\color{red}}

\newif\ifcolonfoundonthisline{}

\makeatletter

\lstdefinestyle{json-style}  
{
	showstringspaces    = false,
	keywords            = {false,true},
	alsoletter          = 0123456789.,
	morestring          = [s]{''}{''},
	stringstyle         = \ifcolonfoundonthisline\JSONstringvaluestyle\fi,
	MoreSelectCharTable =%
	\lst@DefSaveDef{`:}\colon@json{\processColon@json},
	basicstyle          = \footnotesize\ttfamily,
	keywordstyle        = \ttfamily\bfseries,
	numbers				= left, % zakomentować, jeśli numeracja linii jest niepotrzebna
	numberstyle={\footnotesize\ttfamily\color{dark-grey}},
	xleftmargin			= 2em % zakomentować, jeśli numeracja linii jest niepotrzebna
}

\newcommand\processColon@json{%
	\colon@json%
	\ifnum\lst@mode=\lst@Pmode%
	\global\colonfoundonthislinetrue%
	\fi
}

\lst@AddToHook{Output}{%
	\ifcolonfoundonthisline%
	\ifnum\lst@mode=\lst@Pmode%
	\def\lst@thestyle{\JSONnumbervaluestyle}%
	\fi
	\fi
	\lsthk@DetectKeywords% 
}

\lst@AddToHook{EOL}%
{\global\colonfoundonthislinefalse}

\makeatother

%%\definecolor{red}{rgb}{0.6,0,0} % for strings
%%\definecolor{blue}{rgb}{0,0,0.6}
%%\definecolor{green}{rgb}{0,0.8,0}
%%\definecolor{cyan}{rgb}{0.0,0.6,0.6}
%%
%%\lstdefinestyle{sqlstyle}{
%%language=SQL,
%%basicstyle=\footnotesize\ttfamily, 
%%numbers=left, 
%%numberstyle=\tiny, 
%%numbersep=5pt, 
%%tabsize=2, 
%%extendedchars=true, 
%%breaklines=true, 
%%showspaces=false, 
%%showtabs=true, 
%%xleftmargin=17pt,
%%framexleftmargin=17pt,
%%framexrightmargin=5pt,
%%framexbottommargin=4pt,
%%keywordstyle=\color{blue}, 
%%commentstyle=\color{green}, 
%%stringstyle=\color{red}, 
%%}
%%
%%\lstdefinestyle{sharpcstyle}{
%%language=[Sharp]C,
%%basicstyle=\footnotesize\ttfamily, 
%%numbers=left, 
%%numberstyle=\tiny, 
%%numbersep=5pt, 
%%tabsize=2, 
%%extendedchars=true, 
%%breaklines=true, 
%%showspaces=false, 
%%showtabs=true, 
%%xleftmargin=17pt,
%%framexleftmargin=17pt,
%%framexrightmargin=5pt,
%%framexbottommargin=4pt,
%%morecomment=[l]{//}, %use comment-line-style!
%%morecomment=[s]{/*}{*/}, %for multiline comments
%%showstringspaces=false, 
%%morekeywords={  abstract, event, new, struct,
                %%as, explicit, null, switch,
                %%base, extern, object, this,
                %%bool, false, operator, throw,
                %%break, finally, out, true,
                %%byte, fixed, override, try,
                %%case, float, params, typeof,
                %%catch, for, private, uint,
                %%char, foreach, protected, ulong,
                %%checked, goto, public, unchecked,
                %%class, if, readonly, unsafe,
                %%const, implicit, ref, ushort,
                %%continue, in, return, using,
                %%decimal, int, sbyte, virtual,
                %%default, interface, sealed, volatile,
                %%delegate, internal, short, void,
                %%do, is, sizeof, while,
                %%double, lock, stackalloc,
                %%else, long, static,
                %%enum, namespace, string},
%%keywordstyle=\color{cyan},
%%identifierstyle=\color{red},
%%stringstyle=\color{blue}, 
%%commentstyle=\color{green},
%%}



%%%%%%%%%%%%%%%%%%%%%%%%%%%%%%%%%%%%%%%%%%%%%%%%%%%%%%%%%%%%%%%%%%%%%%%%%%%%%%%%
%%  Pakiety i komendy zastosowane tylko do zamieszczenia informacji o użytych komendach i fontach w tym szablonie.
%%  Normalnie nie są one potrzebne. Proszę poniższe deklaracje zamarkować podczas redakcji pracy !!!!
%%%%%%%%%%%%%%%%%%%%%%%%%%%%%%%%%%%%%%%%%%%%%%%%%%%%%%%%%%%%%%%%%%%%%%%%%%%%%%%%
% \usepackage{memlays}     % extra layout diagrams, zastosowane w szblonie do 'debuggowania', używa pakietu layouts
%\usepackage{layouts}
\usepackage{printlen} % pakiet do wyświetlania wartości zdefiniowanych długości, stosowany do 'debuggowania'
\usepackage{enumitem} % pakiet do numerowania 1.1 1.2 w sekcji enumrate
\uselengthunit{pt}
\makeatletter
\newcommand{\showFontSize}{\f@size{} pt} % makro wypisujące wielkość bieżącej czcionki
\makeatother
% do pokazania ramek można byłoby użyć:
%\usepackage{showframe} 

%%%%%%%%%%%%%%%%%%%%%%%%%%%%%%%%%%%%%%%%%%%%%%%%%%%%%%%%%%%%%%%%%%%%%%%%%%%%%%%%
%%  Formatowanie list wyliczeniowych, wypunktowań i własnych otoczeń
%%%%%%%%%%%%%%%%%%%%%%%%%%%%%%%%%%%%%%%%%%%%%%%%%%%%%%%%%%%%%%%%%%%%%%%%%%%%%%%%

% Domyślnie wypunktowania mają zadeklarowane znaki, które nie występują w tgtermes
% Aby latex nie podstawiał w ich miejsca znaków z czcionki standardowej można zrobić podstawienie:
%    \DeclareTextCommandDefault{\textbullet}{\ensuremath{\bullet}}
%    \DeclareTextCommandDefault{\textasteriskcentered}{\ensuremath{\ast}}
%    \DeclareTextCommandDefault{\textperiodcentered}{\ensuremath{\cdot}}
% Jednak jeszcze lepszym pomysłem jest zdefiniowanie otoczeń z wykorzystaniem enumitem
\usepackage{enumitem} % pakiet pozwalający zarządzać formatowaniem list wyliczeniowych
\setlist{noitemsep,topsep=4pt,parsep=0pt,partopsep=4pt,leftmargin=*} % zadeklarowane parametry pozwalają uzyskać 'zwartą' postać wypunktowania bądź wyliczenia
\setenumerate{labelindent=0pt,itemindent=0pt,leftmargin=!,label=\arabic*.} % można zmienić \arabic na \alph, jeśli wyliczenia mają być z literkami
\setlistdepth{4} % definiujemy głębokość zagnieżdżenia list wyliczeniowych do 4 poziomów
\setlist[itemize,1]{label=$\bullet$}  % definiujemy, jaki symbol ma być użyty w wyliczeniu na danym poziomie
\setlist[itemize,2]{label=\normalfont\bfseries\textendash}
\setlist[itemize,3]{label=$\ast$}
\setlist[itemize,4]{label=$\cdot$}
\renewlist{itemize}{itemize}{4}

%%%http://tex.stackexchange.com/questions/29322/how-to-make-enumerate-items-align-at-left-margin
%\renewenvironment{enumerate}
%{
%\begin{list}{\arabic{enumi}.}
%{
%\usecounter{enumi}
%%\setlength{\itemindent}{0pt}
%%\setlength{\leftmargin}{1.8em}%{2zw} % 
%%\setlength{\rightmargin}{0zw} %
%%\setlength{\labelsep}{1zw} %
%%\setlength{\labelwidth}{3zw} % 
%\setlength{\topsep}{6pt}%
%\setlength{\partopsep}{0pt}%
%\setlength{\parskip}{0pt}%
%\setlength{\parsep}{0em} % 
%\setlength{\itemsep}{0em} % 
%%\setlength{\listparindent}{1zw} % 
%}
%}{
%\end{list}
%}

\makeatletter
\renewenvironment{quote}{
	\begin{list}{}
	{
	\setlength{\leftmargin}{1em}
	\setlength{\topsep}{0pt}%
	\setlength{\partopsep}{0pt}%
	\setlength{\parskip}{0pt}%
	\setlength{\parsep}{0pt}%
	\setlength{\itemsep}{0pt}
	}
	}{
	\end{list}}
\makeatother

%%%%%%%%%%%%%%%%%%%%%%%%%%%%%%%%%%%%%%%%%%%%%%%%%%%%%%%%%%%%%%%%%%%%%%%%%%%%%%%%
%%  Pakiet i komendy do generowania indeksu 
%% (ważne, by pojawiły się przed pakietem hyperref)
%%%%%%%%%%%%%%%%%%%%%%%%%%%%%%%%%%%%%%%%%%%%%%%%%%%%%%%%%%%%%%%%%%%%%%%%%%%%%%%%
% pdftex jest w stanie wygenerować indeks (czyli spis haseł z referencjami do stron, na których te hasła się pojawiły).
% Generalnie z indeksem jest sporo problemów, zwłaszcza, gdy pojawiają się polskie literki.
% Trzeba wtedy korzystać z xindy.
% Zwykle w pracach dyplomowych indeksy nie są wykorzystywane. Dlatego są zamarkowane.
%\DisemulatePackage{imakeidx}
%\usepackage[makeindex,noautomatic]{imakeidx} % tutaj mówimy, żeby indeks nie generował się automatycznie, 
%\makeindex
%
%\makeatletter
%%%%\renewenvironment{theindex}
							 %%%%{\vskip 10pt\@makeschapterhead{\indexname}\vskip -3pt%
								%%%%\@mkboth{\MakeUppercase\indexname}%
												%%%%{\MakeUppercase\indexname}%
								%%%%\vspace{-3.2mm}\parindent\z@%
								%%%%\renewcommand\subitem{\par\hangindent 16\p@ \hspace*{0\p@}}%%
								%%%%\phantomsection%
								%%%%\begin{multicols}{2}
								%%%%%\thispagestyle{plain}
								%%%%\parindent\z@                
								%%%%%\parskip\z@ \@plus .3\p@\relax
								%%%%\let\item\@idxitem}
							 %%%%{\end{multicols}\clearpage}
%%%%
%\makeatother




%%%%%%%%%%%%%%%%%%%%%%%%%%%%%%%%%%%%%%%%%%%%%%%%%%%%%%%%%%%%%%%%%%%%%%%%%%%%%%%%
%%  Sprawy metadanych w wynikowym pdf, hyperlinków itp.
%%%%%%%%%%%%%%%%%%%%%%%%%%%%%%%%%%%%%%%%%%%%%%%%%%%%%%%%%%%%%%%%%%%%%%%%%%%%%%%%
% Szablon przygotowano głównie dla pdflatex. Specyficzne komendy dla pdf-owej kompilacj wstawiono 
% w instrukcję warunkową dostarczaną przez pakiet ifpdf 
% Jeśli metadane zawierają przecinki lub średniki, domyślnie metadane te otaczane są apostrofami.
% Piszą o tym na stronie: https://tex.stackexchange.com/questions/3708/hyperref-enquotes-metadata
% Aby pozbyć się tych apostrofów użyto pakietu hyperxmp (ładującego kilka innych pakietów)
\usepackage{ifpdf}
%\newif\ifpdf \ifx\pdfoutput\undefined
%\pdffalse % we are not running PDFLaTeX
%\else
%\pdfoutput=1 % we are running PDFLaTeX
%\pdftrue \fi
\ifpdf{}
 \usepackage{datetime2} % INFO: pakiet potrzeby do uzyskania i sformatowania daty 
 \usepackage[pdftex,bookmarks,breaklinks,unicode]{hyperref}
 \usepackage{hyperxmp}
 \usepackage[pdftex]{graphicx}
 \DeclareGraphicsExtensions{.pdf,.jpg,.mps,.png} % po zadeklarowaniu rozszerzeń można będzie wstawiać pliki z grafiką bez konieczności podawania tych rozszerzeń w ich nazwach
\pdfcompresslevel=9
\pdfoutput=1

% Dobrze przygotowany dokument pdf to taki, który zawiera metadane.
% Poniżej zadeklarowano pola metadanych, jakie będą włączone do dokumentu pdf.
% Można je zmodyfikować w zależności od potrzeb
\makeatletter
\AtBeginDocument{  
  \hypersetup{
	pdfinfo={
    Title = {\@title},
    Author = {\@author},
    Subject={Praca dyplomowa \ifMaster{} magisterska\else inżynierska\fi},  
    Keywords={\@kvpl}, 
		Producer={}, 
	  CreationDate= {}, % należy wstawiać zgodnie ze składnią: {D:yyyymmddhhmmss}, np. D:20210208175600
    ModDate={\pdfcreationdate},   % data modyfikacji będzie datą kompilacji
		Creator={pdftex},
	}}
}
\pdftrailerid{} %Remove ID
\pdfsuppressptexinfo15 %Suppress PTEX.Fullbanner and info of imported PDFs
\makeatother
\else             % jeśli kompilacja jest inna niż pdflatex
\usepackage{graphicx}
\DeclareGraphicsExtensions{.eps,.ps,.jpg,.mps,.png}
\fi
\sloppy

% INFO: dodane by lepiej łamać urle 
\def\UrlBreaks{\do\/\do-\do_}


%%%%%%%%%%%%%%%%%%%%%%%%%%%%%%%%%%%%%%%%%%%%%%%%%%%%%%%%%%%%%%%%%%%%%%%%%%%%%%%%
%%  Formatowanie dokumentu
%%%%%%%%%%%%%%%%%%%%%%%%%%%%%%%%%%%%%%%%%%%%%%%%%%%%%%%%%%%%%%%%%%%%%%%%%%%%%%%%
% INFO: Deklaracja głębokościu numeracji
\setcounter{secnumdepth}{2}
\setcounter{tocdepth}{2}
\setsecnumdepth{subsection} 
% INFO: Dodanie kropek po numerach sekcji
\makeatletter
\def\@seccntformat#1{\csname the#1\endcsname.\quad}
\def\numberline#1{\hb@xt@\@tempdima{#1\if&#1&\else.\fi\hfil}}
\makeatother
% INFO: Numeracja rozdziałów i separatory
\renewcommand{\chapternumberline}[1]{#1.\quad}
\renewcommand{\cftchapterdotsep}{\cftdotsep}


%\usepackage{etoolbox} % odstępy w spisie treści (jeden ze sposobów ustawiania)
%%\makeatletter
%%\pretocmd{\chapter}{\addtocontents{toc}{\protect\addvspace{-1\p@}}}{}{}
%%\pretocmd{\section}{\addtocontents{toc}{\protect\addvspace{-1\p@}}}{}{}
%%\pretocmd{\subsection}{\addtocontents{toc}{\protect\addvspace{-1\p@}}}{}{}
%%\makeatother

\makeatletter % odstępy w spisie pomiędzy rozdziałami
\renewcommand*{\insertchapterspace}{%
  \addtocontents{lof}{\protect\addvspace{3pt}}%
  \addtocontents{lot}{\protect\addvspace{3pt}}%
	\addtocontents{toc}{\protect\addvspace{3pt}} %
  \addtocontents{lol}{\protect\addvspace{3pt}}}
\makeatother 


\setlength{\cftbeforechapterskip}{0pt} % odstępy w spisie treści przed rozdziałem, działa w korelacji z:
\renewcommand{\aftertoctitle}{\afterchaptertitle\vspace{-4pt}} % 
% https://stackoverflow.com/questions/3029271/latex-make-listoffigures-look-like-listoftables-or-lstlistoflistings
%\renewcommand{\memchapinfo}[4]{%
%  \addtocontents{lol}{\protect\addvspace{10pt}}
%}

%\cftsetindents{section}{1.5em}{2.3em}

%\setbeforesecskip{10pt plus 0.5ex}%{-3.5ex \@plus -1ex \@minus -.2ex}
%\setaftersecskip{10pt plus 0.5ex}%\onelineskip}
%\setbeforesubsecskip{8pt plus 0.5ex}%{-3.5ex \@plus -1ex \@minus -.2ex}
%\setaftersubsecskip{8pt plus 0.5ex}%\onelineskip}
%\setlength\floatsep{6pt plus 2pt minus 2pt} 
%\setlength\intextsep{12pt plus 2pt minus 2pt} 
%\setlength\textfloatsep{12pt plus 2pt minus 2pt} 

% Ustawienie odstępu od góry w nienumerowanych rozdziałach oraz wykazach:
% Spis treści, Spis tabel, Spis rysunków, Indeks rzeczowy
%\newlength{\linespace}
%\setlength{\linespace}{-\beforechapskip-\topskip+\headheight+\topsep}
%%%\makechapterstyle{noNumbered}{%
%%%\renewcommand\chapterheadstart{\vspace*{\linespace}}
%%%}
%% powyższa komenda załatwia to, co robią komendy poniższe dla spisów
%\renewcommand*{\tocheadstart}{\vspace*{\linespace}}
%\renewcommand*{\lotheadstart}{\vspace*{\linespace}}
%\renewcommand*{\lofheadstart}{\vspace*{\linespace}}


% INFO: Czcionka do podpisów tabel, rysunków, listingów
\captionnamefont{\small}
\captiontitlefont{\small}


% INFO: Sformatowanie podpisu nad dwukolumnowym listingiem
\newcommand{\listingcaption}[1]
{%
\vspace*{\abovecaptionskip}\small 
\refstepcounter{lstlisting}\hfill%
Listing \thelstlisting: #1\hfill%\hfill%
\addcontentsline{lol}{lstlisting}{\protect\numberline{\thelstlisting}#1}
}%



% INFO: Pomocnicze marko do wyróżniania tekstu w języku angielskim
\newcommand{\eng}[1]{(ang.~\emph{#1})}
% IFNO: Pomocnicze makro do dołączania podpisów do rysunków ze wskazaniem źródła (bez wypisywania tego źródła w spisie rysunków)
\newcommand*{\captionsource}[2]{%
  \caption[{#1}]{%
    #1 \emph{Źródło:} #2%
  }%
}


% INFO: Makro pozwalające zmienić sposób wypisywania rozdziału (proszę z niego nie korzystać)
%\def\printchaptertitle##1{\fonttitle \space \thechapter.\space ##1} 

% INFO: definicje etykiet i tytułów spisów

%\AtBeginDocument{% 
        \addto\captionspolish{% 
        \renewcommand{\tablename}{Tab.}%% INFO: Przedefiniowanie etykiet w podpisach tabel 
}%} 

%\AtBeginDocument{% 
%        \addto\captionspolish{% 
%        \renewcommand{\chaptername}{Rozdział}% INFO: Przedefiniowanie nazwy rozdziału, niepotrzebne, bo przy polskich ustawieniach językowych jest 'Rozdział'
%}} 

% Przedefiniowanie etykiet oraz nazw wykazu literatury, spisów, indeksu
%\AtBeginDocument{% 
        \addto\captionspolish{% 
        \renewcommand{\figurename}{Rys.}%% INFO: Przedefiniowanie etykiet w podpisach rysunków 
}%}

%\AtBeginDocument{% 
        \addto\captionspolish{% 
        \renewcommand{\lstlistlistingname}{Spis listingów}%% INFO: Przedefiniowanie nazwy spisu listingów
}%} 
\newlistof{lstlistoflistings}{lol}{\lstlistlistingname}


%\AtBeginDocument{% 
        \addto\captionspolish{% 
        \renewcommand{\bibname}{Literatura}%% INFO: Przedefiniowanie nazwy wykazu literatury 
}%}

%\AtBeginDocument{% 
        \addto\captionspolish{% 
        \renewcommand{\listfigurename}{Spis rysunków}%% INFO: Przedefiniowanie nazwy spisu rysunków 
}%}

%\AtBeginDocument{% 
        \addto\captionspolish{% 
        \renewcommand{\listtablename}{Spis tabel}%% INFO: Przedefiniowanie nazwy spisu tabel 
}%}

%\AtBeginDocument{% 
        \addto\captionspolish{% 
\renewcommand\indexname{Indeks rzeczowy}%% INFO: Przedefiniowanie nazwy indeksu 
}%}

%\AtBeginDocument{% 
%    \addto\captionspolish{
%\renewcommand\abstractname{Streszczenie}%% INFO: Przedefiniowanie nazwy strzeszczenia, niepotrzebne, bo przy polskich ustawieniach językowych jest 'Streszczenie'
%}%}

%\AtBeginDocument{% 
%    \addto\captionsenglish{
%\renewcommand\abstractname{Abstract} 
%}%}

\renewcommand{\abstractnamefont}{\normalfont\Large\bfseries}
\renewcommand{\abstracttextfont}{\normalfont}


%%%%%%%%%%%%%%%%%%%%%%%%%%%%%%%%%%%%%%%%%%%%%%%%%%%%%%%%%%%%%%%%%%%%%%%%%%%%%%%%
%% Definicje stopek i nagłówków
%%%%%%%%%%%%%%%%%%%%%%%%%%%%%%%%%%%%%%%%%%%%%%%%%%%%%%%%%%%%%%%%%%%%%%%%%%%%%%%%
\addtopsmarks{headings}{%
\nouppercaseheads{} % added at the beginning
}{%
\createmark{chapter}{both}{shownumber}{}{.\ \space}
%\createmark{chapter}{left}{shownumber}{}{.\ \space}
\createmark{section}{right}{shownumber}{}{.\ \space}
}%use the new settings

\makeatletter
\copypagestyle{outer}{headings}
\makeoddhead{outer}{}{}{\small\itshape\rightmark\/}
\makeevenhead{outer}{\small\itshape\leftmark\/}{}{}
\makeoddfoot{outer}{\small\@author:~\@titleShort}{}{\small\thepage}
\makeevenfoot{outer}{\small\thepage}{}{\small\@author:~\@title}
\makeheadrule{outer}{\linewidth}{\normalrulethickness}
\makefootrule{outer}{\linewidth}{\normalrulethickness}{2pt}
\makeatother

% fix plain
\copypagestyle{plain}{headings} % overwrite plain with outer
\makeoddhead{plain}{}{}{} % remove right header
\makeevenhead{plain}{}{}{} % remove left header
\makeevenfoot{plain}{}{}{}
\makeoddfoot{plain}{}{}{}

\copypagestyle{empty}{headings} % overwrite plain with outer
\makeoddhead{empty}{}{}{} % remove right header
\makeevenhead{empty}{}{}{} % remove left header
\makeevenfoot{empty}{}{}{}
\makeoddfoot{empty}{}{}{}

% INFO: deklaracja zmiennej logicznej wykorzystywanej do rozróżnienia pracy inżynierskiej i magisterskiej
\newif\ifMaster{}
\Mastertrue{}

%%%%%%%%%%%%%%%%%%%%%%%%%%%%%%%%%%%%%%%%%%%%%%%%%%%%%%%%%%%%%%%%%%%%%%%%%%%%%%%%
%% Definicja strony tytułowej 
%%%%%%%%%%%%%%%%%%%%%%%%%%%%%%%%%%%%%%%%%%%%%%%%%%%%%%%%%%%%%%%%%%%%%%%%%%%%%%%%
\makeatletter
%Uczelnia
\newcommand\uczelnia[1]{\renewcommand\@uczelnia{#1}}
\newcommand\@uczelnia{}
%Wydział
\newcommand\wydzial[1]{\renewcommand\@wydzial{#1}}
\newcommand\@wydzial{}
%Kierunek
\newcommand\kierunek[1]{\renewcommand\@kierunek{#1}}
\newcommand\@kierunek{}
%Specjalność
\newcommand\specjalnosc[1]{\renewcommand\@specjalnosc{#1}}
\newcommand\@specjalnosc{}
%Tytuł po angielsku
\newcommand\titleEN[1]{\renewcommand\@titleEN{#1}}
\newcommand\@titleEN{}
%Tytuł krótki
\newcommand\titleShort[1]{\renewcommand\@titleShort{#1}}
\newcommand\@titleShort{}
%Promotor
\newcommand\promotor[1]{\renewcommand\@promotor{#1}}
\newcommand\@promotor{}
%Słowa kluczowe
\newcommand\kvpl[1]{\renewcommand\@kvpl{#1}}
\newcommand\@kvpl{}
\newcommand\kven[1]{\renewcommand\@kven{#1}}
\newcommand\@kven{}
%Komenda wykorzystywana w streszczeniu
\newcommand\mykeywords{\hspace{\absleftindent}%
\parbox{\linewidth-2.0\absleftindent}{
       \iflanguage{polish}{\textbf{Słowa kluczowe:} \@kvpl}{%
			 \iflanguage{english}{\textbf{Keywords:} \@kven}}{}}
				}

\def\maketitle{%
  \pagestyle{empty}%
%%\garamond 
	\fontfamily{\ebgaramond@family}\selectfont % na stronie tytułowej czcionka garamond
%%%%%%%%%%%%%%%%%%%%%%%%%%%%%%%%%%%%%%%%%%%%%%%%%%%%%%%%%%%%%%%%%%%%%%%%%%%%%%	
%% Poniżej, w otoczniu picture, wstawiono tytuł i autora. 
%% Tytuł (z autorem) musi znaleźć się w obszarze 
%% odpowiadającym okienku 110mmx75mm, którego lewy górny róg 
%% jest w położeniu 77mm od lewej i 111mm od górnej  krawędzi strony 
%% (tak wynika z wycięcia na okładce). 
%% Poniższy kod musi być użyty dokładnie w miejscu gdzie jest.
%% Jeśli tytuł nie mieści się w okienku, to należy tak pozmieniać 
%% parametry użytych komend, aby ten przydługi tytuł jednak 
%% upakować do okienka.
%%
%% Sama okładka (kolorowa strona z wycięciem, kiedyś była do pobrania z dydaktyki) 
%% powinna być przycięta o 3mm od każdej z krawędzi.
%% Te 3mm pewnie zostawiono na ewentualne spady czy też specjalną oprawę.
%%%%%%%%%%%%%%%%%%%%%%%%%%%%%%%%%%%%%%%%%%%%%%%%%%%%%%%%%%%%%%%%%%%%%%%%%%%%%%
\newlength{\tmpfboxrule}
\setlength{\tmpfboxrule}{\fboxrule}
\setlength{\fboxsep}{2mm}
\setlength{\fboxrule}{0mm} 
%\setlength{\fboxrule}{0.1mm} %% INFO: Jeśli chcemy zobaczyć ramkę, wystarczy odmarkować tę linijkę
\setlength{\unitlength}{1mm}
\begin{picture}(0,0)
%\put(26,-124){\fbox{% ustawienie do "wyciętego okienka"
\put(20,-124){\fbox{% ustawienie na środku
\parbox[c][71mm][c]{104mm}{\centering%\lineskip=34pt 
{\fontsize{18pt}{20pt}\bfseries\selectfont \@title}\\[5mm]
{\fontsize{18pt}{20pt}\bfseries\selectfont \@titleEN}\\[10mm] % INFO: wstawiono tytuł w języku angielskim, choć w obecnych oficjalnych zaleceniach tego nie ma
%\fontsize{16pt}{18pt}\selectfont AUTOR:\\[2mm]
{\fontsize{16pt}{18pt}\selectfont \@author}}
}
}
\end{picture}
\setlength{\fboxrule}{\tmpfboxrule} 
%%%%%%%%%%%%%%%%%%%%%%%%%%%%%%%%%%%%%%%%%%%%%%%%%%%%%%%%%%%%%%%%%%%%%%%%%%%%%%
%% Reszta strony z nazwą uczelni, wydziału, kierunkiem, specjalnością
%% promotorem, oceną pracy (zakomentowane), miastem i rokiem
	{\vskip 9pt\centering
		{\fontsize{20pt}{22pt}\bfseries\selectfont \@uczelnia}\\[5pt]
		{\fontsize{16pt}{18pt}\bfseries\selectfont \@wydzial}\\[1pt]
		  \hrule
	}
{\vskip 24pt\raggedright\fontsize{14pt}{16pt}\selectfont%
\begin{tabular}{@{}ll}
Kierunek: & {\bfseries \@kierunek}\\
%Specjalność: & {\bfseries \@specjalnosc}\\
\end{tabular}\\[1.3cm]
}
{\vskip 29pt\centering{\fontsize{24pt}{26pt}\selectfont%
{\fontsize{26pt}{28pt}\selectfont P}RACA {\fontsize{26pt}{24pt}\selectfont D}YPLOMOWA\\[7pt]
\ifMaster\selectfont{\fontsize{26pt}{28pt}\selectfont M}AGISTERSKA\\[2.5cm]%
\else \selectfont{\fontsize{26pt}{28pt}\selectfont I}NŻYNIERSKA\\[2.5cm]\fi
}}
	\vfill
{\centering
		{\fontsize{14pt}{16pt}\selectfont Opiekun pracy}\\[2mm] 
		{\fontsize{14pt}{16pt}\bfseries\selectfont \@promotor}\\[10mm]%INFO: tutaj wstawiane ejst nazwisko promotora
%		&{\fontsize{16pt}{18pt}\selectfont OCENA PRACY:}\\[20mm] 
% INFO: linię powyższą zakomentowano, gdyż od czasu pandemii COVID-19 prace mogą być dostarczane bez podpisu promotora
}
\vspace{4cm}\noindent
{\fontsize{12pt}{14pt}\selectfont Słowa kluczowe: \@kvpl}% INFO: na stronę tytułową trafiają tylko słowa kluczowe w języku polskim (w jakim napisana jest praca)
\vspace{1.3cm}
\hrule\vspace*{0.3cm}
{\centering
{\fontsize{14pt}{16pt}\selectfont \@date}\\[0cm]
}
%\ungaramond
\normalfont{}
 \cleardoublepage{}
}
\makeatother

%\AtBeginDocument{\addtocontents{toc}{\protect\thispagestyle{empty}}}

%%%%%%%%%%%%%%%%%%%%%%%%%%%%%%%%%%%%%%%%%%%%%%%%%%%%%%%%%%%%%%%%%%%%%%%%%%%%%%%%%%
%%%%%%%%%%%%%%%%%%%%%%%%%%%%%%%%%%%%%%%%%%%%%%%%%%%%%%%%%%%%%%%%%%%%%%%%%%%%%%%%%%
%   Początek strefy do nanoszenia zmian 
%%%%%%%%%%%%%%%%%%%%%%%%%%%%%%%%%%%%%%%%%%%%%%%%%%%%%%%%%%%%%%%%%%%%%%%%%%%%%%%%%%

%%%%%%%%%%%%%%%%%%%%%%%%%%%%%%%%%%%%%%%%%%%%%%%%%%%%%%%%%%%%%%%%%%%%%%%%%%%%%%%%%%
%%%%%%%%%%%%%%%%%%%%%%%%%%%%%%%%%%%%%%%%%%%%%%%%%%%%%%%%%%%%%%%%%%%%%%%%%%%%%%%%%%
%%
%%  Metadane dokumentu
%%  - tutaj należy wstawić własne dane
%%
%%%%%%%%%%%%%%%%%%%%%%%%%%%%%%%%%%%%%%%%%%%%%%%%%%%%%%%%%%%%%%%%%%%%%%%%%%%%%%%%%%

%%%%%%%%%%%%%%%%%%%%%%%%%%%%%%%%%%%%%%%%%%%%%%%%%%%%%%%%%%%%%%%%%%%%%%%%%%%%%%%%%%
%\Mastertrue % INFO: odkomentuj, jeśli to praca magisterska
\title{Mechanizm multilateracji w rozproszonej sieci sensorów audio} % INFO: tytuł pracy w języku polskim 
\titleShort{Metody multilateracji $\ldots$}  % INFO: krótki tytuł pracy (do zamieszczenia w stopce, sklejony z imieniem i nazwiskiem autora nie powinien zająć więcej niż jedną linijkę)
\titleEN{Multilateration mechanism in distributed net of audio sensors} % INFO: tytuł pracy w języku angielskim
\author{Gabriel Budziński}  % INFO: imię i nazwisko autora
\uczelnia{Politechnika Wrocławska} % INFO: nazwa uczelni
\wydzial{Wydział Informatyki i Telekomunikacji} % INFO: nazwa wydziału
\kierunek{Informatyka algorytmiczna (INA)} % IFO: nazwa kierunku
\specjalnosc{Informatyka algorytmiczna (INA)} % INFO: nazwa specjalności
\promotor{dr inż. Przemysław Błaśkiewicz} % INFO: dane promotora 
\kvpl{multilateracja, sensory audio, ESP8266} % INFO: słowa kluczowe po polsku
\kven{multilateration, WASN, ESP8266} % INFO: słowa kluczowe po angielsku
\date{WROCŁAW, 2024} % INFO: miejscowość, rok złożenia pracy dyplomowej

%%%%%%%%%%%%%%%%%%%%%%%%%%%%%%%%%%%%%%%%%%%%%%%%%%%%%%%%%%%%%%%%%%%%%%%%%%%%%%%%%%
%%
%%  Struktura dokumentu
%%  - tutaj należy wstawić własne rozdziały
%%
%%%%%%%%%%%%%%%%%%%%%%%%%%%%%%%%%%%%%%%%%%%%%%%%%%%%%%%%%%%%%%%%%%%%%%%%%%%%%%%%%%

%%%%%%%%%%%%%%%%%%%%%%%%%%%%%%%%%%%%%%%%%%%%%%%%%%%%%%%%%%%%%%%%%%%%%%%%%%%%%%%%%%
% INFO: Za pomocą polecenia \includeonly{} można dokonać selekcji  
%       tych części (plików z latexowym kodem), które mają być kompilowane. 
%       Przydaje się to szczególnie podczas pracy nad dużymi dokumentami. 
%       Bo im mniej części zostanie wyselekcjonowanych, tym szybsza będzie kompilacja.
%       Proszę nie mylić tej komendy z poleceniem \include{}, którą używa się 
%       do zadeklarowania pełnej struktury dokumentu (plików z latexowym kodem).
%\includeonly{skroty,rozdzial01}  

\begin{document}
% Komendami poniżej można przełączyć odstęp między liniami. Proszę jednak tego nie robić !!!
%\SingleSpacing
%\OnehalfSpacing
%\DoubleSpacing

%\settypeoutlayoutunit{cm} % do debugowania
%\typeoutstandardlayout    % wypisuje na stdout informacje o ustawieniach

%\frontmatter
\pdfbookmark[0]{Tytuł}{Tytul.1}
\maketitle
\clearpage

% Kolejne części dokumentu: streszczenie, spisy, skróty, rozdziały, dodatki
%\chapterstyle{noNumbered}
% STRESZCZENIE (proszę zajrzeć do środka na zakomentowane komendy)
\pdfbookmark[0]{Streszczenie}{streszczenie.1}
%\phantomsection
%\addcontentsline{toc}{chapter}{Streszczenie}
%%% Poniższe zostało niewykorzystane (tj. zrezygnowano z utworzenia nienumerowanego rozdziału na abstrakt)
%%%\begingroup
%%%\setlength\beforechapskip{48pt} % z jakiegoś powodu była maleńka różnica w położeniu nagłówka rozdziału numerowanego i nienumerowanego
%%%\chapter*{\centering Abstrakt}
%%%\endgroup
%%%\label{sec:abstrakt}
%%%Lorem ipsum dolor sit amet eleifend et, congue arcu. Morbi tellus sit amet, massa. Vivamus est id risus. Sed sit amet, libero. Aenean ac ipsum. Mauris vel lectus. 
%%%
%%%Nam id nulla a adipiscing tortor, dictum ut, lobortis urna. Donec non dui. Cras tempus orci ipsum, molestie quis, lacinia varius nunc, rhoncus purus, consectetuer congue risus. 
%\mbox{}\vspace{2cm} % można przesunąć, w zależności od długości streszczenia
\begin{abstract}
    Problem pozycjonowania w przestrzeni na podstawie emitowanego dźwięku obiektu pozycjonowanego wiąże się z wykorzystaniem możliwie zsynchronizowanych w czasie węzłów (mikrofonów) i pomiarze różnic czasu odbioru dźwięku przez czujniki. W pracy zostanie zbudowana sieć (co najmniej 4 sztuki) sensorów audio połączonych bezprzewodowo między sobą i ze stacją główną. Zadaniem sieci będzie wskazanie lokalizacji w przestrzeni punkowego przedmiotu emitującego dźwięk. Oprócz wyboru i implementacji algorytmu multilateracji zaproponowane zostanie rozwiązanie problemu synchronizacji czasu między sensorami, minimalizacji opóźnienia w komunikacji oraz kalibracji systemu.
\end{abstract}
\mykeywords{}

{
    \selectlanguage{english}
    \begin{abstract}
        The problem of positioning in space based on the emitted sound of the positioned object involves the use of as closely synchronized nodes (microphones) as possible in time and measuring the differences in the time of sound reception by sensors. In the work, a network (of at least 4 units) of audio sensors connected wirelessly to each other and to the main station will be built. The network's task will be to indicate the location in space of a point-like object emitting sound. In addition to selecting and implementing the multilateration algorithm, a solution to the problem of time synchronization between sensors, minimizing communication delay, and system calibration will be proposed.
    \end{abstract}
    \mykeywords{}
}

\pagestyle{outer}
\clearpage
% SPIS TREŚCI (zostanie wygenerowany automatycznie)
\pdfbookmark[0]{Spis treści}{spisTresci.1}%
%%\phantomsection
%%\addcontentsline{toc}{chapter}{Spis treści}
\tableofcontents* 
\clearpage
% SPIS RYSUNKÓW (zostanie wygenerowany automatycznie)
\pdfbookmark[0]{Spis rysunków}{spisRysunkow.1} % jeśli chcemy mieć w spisie treści, to zamarkować tę linię, a odmarkować linie poniższe
%%\phantomsection
%%\addcontentsline{toc}{chapter}{Spis rysunków}
\listoffigures*
\clearpage
% SPIS TABEL (zostanie wygenerowany automatycznie)
\pdfbookmark[0]{Spis tabel}{spisTabel.1} %
%%\phantomsection
%%\addcontentsline{toc}{chapter}{Spis tabel}
\listoftables*
\clearpage
% SPIS LISTINGÓW (zostanie wygenerowany automatycznie)
\pdfbookmark[0]{Spis listingów}{spisListingow.1} %
%%\phantomsection
%%\addcontentsline{toc}{chapter}{Spis listingów}
\lstlistoflistings*
\clearpage
% SKRÓTY (to opcjonalna część pracy)
\pdfbookmark[0]{Skróty}{skroty.1}% 
%%\phantomsection
%%\addcontentsline{toc}{chapter}{Skróty}
\chapter*{Skróty}
\label{sec:skroty}
\noindent\vspace{-\topsep-\partopsep-\parsep} % Jeśli zaczyna się od otoczenia description, to otoczenie to ląduje lekko niżej niż wylądowałby zwykły tekst, dlatego wstawiano przesunięcie w pionie
\begin{description}[labelwidth=*]
  \item [OGC] (ang.\ \emph{Open Geospatial Consortium}) %-- jednostka arytmetyczno--logiczna
\end{description}
 
% ROZDZIAŁY (kolejne rozdziały dołączane są z kolejnych plików)
\chapterstyle{default}
\chapter{Sformułowanie problemu}\label{chap:problem}

Weźmy punkt $(x, y, z) \in \mathbb{R}^3$ i oznaczmy go $N$, ponadto weźmy zbiór $n$ punktów $\left\{(x_i, y_i, z_i): i \in [n]\right\} \subset \mathbb{R}^3$, który oznaczmy $\mathcal{O}$, a punkty należące do zbioru odpowiednio $O_i$. Niech $d_i$ będzie odległością $d(N, O_i)$. Znając współrzędne punktów $O_i$ oraz odległości $d_i$ chcemy znaleźć $(x, y, z)$. Można to osiągnąć rozwiązując układ równań

\begin{equation}\label{eq:base}
    \begin{dcases}
        {(x - x_1)}^2 + {(y - y_1)}^2 + {(z - z_1)}^2 = {d_1}^2\\
        {(x - x_2)}^2 + {(y - y_2)}^2 + {(z - z_2)}^2 = {d_2}^2\\
        \mspace{150mu}\vdots\\
        {(x - x_n)}^2 + {(y - y_n)}^2 + {(z - z_n)}^2 = {d_n}^2\,.
    \end{dcases}
\end{equation}

\section{Aktualny stan wiedzy}

W dostępnej literaturze obejmującą temat multilateracji i systemów multilateracyjnych opisane są trzy główne podejścia:

\begin{itemize}
    \item iteracyjne,
    \item algebraiczne,
    \item liniowe.
\end{itemize}

Metoda iteracyjna dąży do minimalizacji błędu przybliżonego rozwiązania układu równań~\ref{eq:base}. Typowymi narzędziami w niej wykorzystywanymi są: metoda najszybszego spadku~\cite{doi:10.1137/0111030}, metoda Newtona oraz szereg Taylora~\cite{4101619}. Praca~\cite{4101619} zasuguje na szczególną uwagę ze względu na skupienie się na zastosowaniu metody do użytków lokalizacyjnych, w przeciwieństwie do ogólnego podejścia do problemu optymalizacyjnego. Największymi problemami metod iteracyjnych są niepewność zbiegania do rozwiązania optymalnego a także złożoność obliczeniowa, która wpływa negatywnie na czas działania.

Metoda algebraiczna zaproponowana w pracy~\cite{4104017}, w której pochylono się na algebraicznym rozwiązaniu układów równań w systemie GPS, ma znaczącą przewagę nad metodami iteracyjnymi otrzymywania dwóch możliwych rozwiązań natychmiastowo, z ktorych można następnie wybrać bardziej odpowiadające. Dodatkowo ta metoda daje możliwość operowania na pomiarach z arbitralnej liczby punktów, w przeciwieństwie do metod y zaproponowanej w pracy~\cite{article}, w której skupiono się na implementacji modelu VHDL pozycjonującego na postawie odczytów z czterech satelitów.

Wyżej wymienione metody, choć są wciąż używane, nie pojawiają się w pracach z ostatnich lat. Większość rozwiązań prezentowanych w ostatnich dekadach opiera się na układzie równań liniowych z dodatkową zmienną ze względu na nieliniowość problemu. Pierwsze tego typu rozwiązanie zaprezentowano w pracy~\cite{301830}, w którym zmienna nieliniowa była niwelowana przez odejmowanie od siebie kolejnych równań. Inny wariant przestawiono w pracy~\cite{1018778}, który zredagowano w czytelniejszej formie w pracy~\cite{norrdine2012algebraic}. Właśnie ta praca była inspiracją do niniejszej pracy.

Opis aktualnego stanu wiedzy zaczerpnięto z niedawnej pracy przeglądowej~\cite{10419087}.


\section{Obrane rozwiązanie}

W pracy wybrano rozwiązanie przedstawione w pracy~\cite{norrdine2012algebraic}. Przekształćmy wyjściowy układ równań~\ref{eq:base} do postaci
\begin{equation}
    \begin{dcases}
        (x^2 + y^2 + z^2) - 2x_1x - 2y_1y - 2z_1z = d_1^2 - x_1^2 - y_1^2 - z_1^2\\
        (x^2 + y^2 + z^2) - 2x_2x - 2y_2y - 2z_2z = d_2^2 - x_2^2 - y_2^2 - z_2^2\\
        \mspace{220mu}\vdots\\
        (x^2 + y^2 + z^2) - 2x_n x - 2y_n y - 2z_n z = d_n^2 - x_n^2 - y_n^2 - z_n^2\,,
    \end{dcases}
\end{equation}
a następnie do reprezentacji macierzowej
\begin{equation}
    \left[
        \begin{matrix}
            1 & -2x_1 & -2y_1 & -2z_1\\
            1 & -2x_2 & -2y_2 & -2z_2\\
            \multicolumn{4}{c}{\vdots}\\
            1 & -2x_n & -2y_n & -2z_n\\
        \end{matrix}
    \right]
    \left[
        \begin{matrix}
            x^2 + y^2 + z^2\\
            x\\
            y\\
            z\\
        \end{matrix}
    \right]
    =
    \left[
        \begin{matrix}
            d_1^2 - x_1^2 - y_1^2 - z_1^2\\
            d_2^2 - x_2^2 - y_2^2 - z_2^2\\
            \vdots\\
            d_n^2 - x_n^2 - y_n^2 - z_n^2\\
        \end{matrix}
    \right]\,,
\end{equation}
którą można przedstawić jako
\begin{equation}
    \mathbf{A} \cdot \mathbf{x} = \mathbf{b}\,,
\label{eq:matrix}
\end{equation}
gdzie
\begin{equation}
    \mathbf{A} =
    \left[
        \begin{matrix}
            1 & -2x_1 & -2y_1 & -2z_1\\
            1 & -2x_2 & -2y_2 & -2z_2\\
            \multicolumn{4}{c}{\vdots}\\
            1 & -2x_n & -2y_n & -2z_n\\
        \end{matrix}
    \right]\,,
\end{equation}
\begin{equation}
    \mathbf{x} =
    \left[
        \begin{matrix}
            x^2 + y^2 + z^2\\
            x\\
            y\\
            z\\
        \end{matrix}
    \right]\,,
\end{equation}
\begin{equation}
    \mathbf{b} =
    \left[
        \begin{matrix}
            d_1^2 - x_1^2 - y_1^2 - z_1^2\\
            d_2^2 - x_2^2 - y_2^2 - z_2^2\\
            \vdots\\
            d_n^2 - x_n^2 - y_n^2 - z_n^2\\
        \end{matrix}
    \right]
\end{equation}

przy założeniu, że $\mathbf{x} \in E$, gdzie $E = \left\{(x_0, x_1, x_2, x_3)^T \in {\mathbb{R}}^4\ |\ x_0 = {x_1}^2 + {x_2}^2 + {x_3}^2\right\}$.

\subsection{Liniowa metoda najmniejszych kwadratów}

Aby rozwiązać powyższe równanie liniowe obrano liniową metodę najmniejszych kwadratów. Z definicji rozwiązaniem równania
\begin{equation}
    \mathbf{A} \cdot \mathbf{x} = \mathbf{b}\,,
\end{equation}
gdzie $\mathbf{A} \in \mathbb{R}^{n\times m}$, $\mathbf{b} \in \mathbb{R}^n$ są znane będzie taki $\mathbf{x} \in \mathbb{R}^m$, który minimalizuje
\begin{equation}
    \sum_{i=1}^{n}{\left|\sum_{j=1}^{m}{A_{ij}x_j} - b_i\right|^2} = ||\mathbf{A}\mathbf{x} - \mathbf{b}||^2
\end{equation}
Taki $\mathbf{x}$ obliczamy przy użyciu powszechnie znanego w kontekście najmniejszych kwadratów \textit{równania normalnego} postaci
\begin{equation}
    \mathbf{A}^T\mathbf{A}\hat{\mathbf{x}} = \mathbf{A}^T\mathbf{b}
\end{equation}
lub po przekształceniu
\begin{equation}\label{eq:lls}
    \hat{\mathbf{x}} = {\left(\mathbf{A}^T\mathbf{A}\right)}^{-1}\mathbf{A}^T\mathbf{b}
\end{equation}

Jeśli żadna trójka odbiorników nie jest współliniowa ($\rank\mathbf{A} = m$) oraz pomiarów jest więcej niż wymiarów $(n > m)$ możemy zagwarantować odwracalność $\mathbf{A}$, a co za tym idzie istnienie $\hat{\mathbf{x}}$.

\subsubsection{Rozwiązanie ogólne}

Uogólniona forma równania macierzowego problemu multilateracji dla przestrzeni $m$-wymiarowej i $n$ odbiorników:
\begin{equation}
    \left[
        \begin{matrix}
            1 & -2\mathbf{x^{(1)}}\\
            1 & -2\mathbf{x^{(2)}}\\
            \vdots & \vdots\\
            1 & -2\mathbf{x^{(n)}}\\
        \end{matrix}
    \right]
    \left[
        \begin{matrix}
            \sum_{i=1}^{m}{x_i}^2\\
            x_1\\
            \vdots\\
            x_m
        \end{matrix}
    \right]
    =
    \left[
        \begin{matrix}
            d_1^2 - \sum_{i=1}^{m}{x_i^{(1)}}^2\\
            d_2^2 - \sum_{i=1}^{m}{x_i^{(2)}}^2\\
            \vdots\\
            d_n^2 - \sum_{i=1}^{m}{x_i^{(n)}}^2\\
        \end{matrix}
    \right]
\end{equation}
z analogicznym ograniczeniem $\mathbf{x} \in E$.

\section{Przykładowy egzemplarz problemu}

\begin{figure}[!h]
    \centering
    \begin{tikzpicture}
        \coordinate (O1) at (0,2);
        \coordinate (O2) at (2,0);
        \coordinate (O3) at (8,3);
        \coordinate (N) at (3,5);
        \filldraw[black] (O1) circle (2pt) node[left]{$O_1$};
        \filldraw[black] (O2) circle (2pt) node[below]{$O_2$};
        \filldraw[black] (O3) circle (2pt) node[right]{$O_3$};
        \filldraw[black] (N) circle (2pt) node[above right]{$N$};
        \draw[black] (O1) -- node[above left]{$d_1$} (N);
        \draw[black] (O2) -- node[left]{$d_2$} (N);
        \draw[black] (O3) -- node[above]{$d_3$} (N);
        \draw[loosely dashed, thin] (O1) -- (O2);
        \draw[loosely dashed, thin] (O1) -- (O3);
        \draw[loosely dashed, thin] (O2) -- (O3);
    \end{tikzpicture}
    \caption{Egzemplarz problemu multilateracji}
\label{fig:example}
\end{figure}

Staramy się ustalić pozycję punktu $N$ na postawie współrzędnych punktów $O_1, O_2, O_3$ oraz odległości $d_1, d_2, d_3$. Wykorzystując równanie~\ref{eq:lls} podstawiamy znane wartości otrzymując $\mathbf{x}$ zawierający szukane współrzędne. 

Jak łatwo zauważyć pierwsza część lewej strony równości zależy jedynie od macierzy $\mathbf{A}$, która nie zawiera zmiennych związanych z węzłem źródłowym, co pozwala, przy założeniu stacjonarności odbiorników, na wcześniejsze i jednorazowe przeprowadzenie kosztownych przekształceń tej macierzy. W tej sytuacji jednorazowe obliczenie rozwiązania wymaga jedynie obliczenia wektora $\mathbf{b}$ oraz iloczynu skalarnego.

\chapter*{Podsumowanie}\label{chap:podsumowanie}
\addcontentsline{toc}{chapter}{Podsumowanie}

Celem pracy niniejszej pracy było zaprojektowanie i zaimplementowanie systemu multilateracyjnego działającego w domenie fal dźwiękowych. W ramach pracy dokonano wyboru metody multilateracji najlepiej nadającej się do tego zastosowania, jak również przeprowadzono szereg testów wykazujących nieprawidłowości działania, a następnie podjęto próby eliminacji tych błędów. Podczas eksperymentów podstawową potrzebę dokładnej synchronizacji zegarów urządzeń w systemie na efekty jego działania. Przeanalizowano też znaczenie wpływu zastosowanej metody synchronizacji zegarów, a także interesujące skalowanie estymowanej odległości wynikającej z fizycznych mechanizmów rozchodzenia się fali dźwiękowej.

Wnioski z przeprowadzonych badań wskazują, że system multilateracyjny oparty o szeroko dostępne moduły mikrofonowe jest zdolny śledzić pozycję nadajnika sygnałów dźwiękowych w podzbiorze przestrzeni dwuwymiarowej obejmowanym ograniczonym zasięgiem tychże mikrofonów w węzłach odbiorczych. Uzyskane wyniki są istotne z punktu widzenia systemów lokalizacji bazujących na falach dźwiękowych, szczególnie w kontekście ich zastosowań w technologii IoT. Wykazano, że dostępne komercyjnie moduły mikrofonowe mogą być wykorzystane do tworzenia efektywnych i stosunkowo tanich systemów lokalizacji.

\color{blue}
wydaje się jednak, że szczyptę dziegciu warto by dodać i rzetelnie powiedzieć, że czasami może być niezły bałagan i dużo zależy od środowiska akustycznego... no i że nie wiadomo, jak to będzie się sprawować na większych odległościach -- może postawiłby Pan jakąś hipotezę?

może słówko jeszcze o wydajności (złożoności obliczeniowej) -- że małe urządzonka, a dają radę i w efekcie nie dość, że mikrofony za dwa złote, to jeszcze procesory za piątaka i wszystko działa (albo: prawie działa).
\color{black}

Dalszymi krokami rozwoju systemu, których nie podjęto ze względu na ograniczony budżet czasowy mogłyby być:

\begin{itemize}
    \item rozszerzenie do i ewaluacja działania w przestrzeni trójwymiarowej,
    \item bardziej wnikliwe testy i badania roli urządzeń przetwarzających sygnał dźwiękowy na cyfrowy; inwestygacja wpływu typu rodzaju mikrofonu, wzmacniacza czy przetwornika ADC (\textit{ang.\ analog to digital}) na estymację odległości,
    \item stworzenie lepszego algorytmu korekcji estymowanej odległości względem rzeczywistej,
    \item ewentualne dostosowywanie algorytmów synchronizacji zegarów do przyszłych zastosowań,
    \item próba minimalizacji wpływu charakterystyki dźwiękowej otoczenia na dokładność wyników.
\end{itemize}

Podsumowując, zaprojektowany i zaimplementowany system multilateracyjny działający w domenie fal dźwiękowych udowodnił swoją skuteczność w śledzeniu pozycji nadajnika w dwuwymiarowej przestrzeni, otwierając tym samym drogę do dalszych badań i udoskonaleń, które mogą znacząco poszerzyć jego zastosowanie w różnych dziedzinach.
% LITERATURA (zostanie wygenerowana automatycznie)
%UWAGA: bibliotekę referencji należy przygotować samemu. Dobrym do tego narzędziem jest JabRef.
%       JabRef oferuje jednak większą liczbę typów rekordów niż obsługuje BibTeX.
%       Proszę nie deklarować rekordów o typach nieobsługiwanych przez BibTeX.
%       Formatowania wykazu literatury i cytowań odbywać się ma zgodnie z zadeklarowanym stylem.
%       Zalecane są style produkujące numeryczne cytowania (w postaci [1], [2,3]).
%       Takim stylem jest np. plabbrv
\bibliographystyle{ieeetr}
%       Aby zapanować nad odstępami w wykazie literatury można posłużyć się poniższą komendą
\setlength{\bibitemsep}{2pt} % - zacieśnia wykaz
%       Pozycja Literatura pojawia się w spisie treści nieco inaczej niż spisy rysunków, tabel itp.
%       Aby zachować właściwe odstępy należy użyć poniższej komendy
\addtocontents{toc}{\addvspace{2pt}} % ustawiamy odstęp w spisie treści przed pozycją Literatura 
%       Nazwę pliku przygotowanej biblioteki wpisuje się bez rozszerzenia .bib
%       (linia poniżej załaduje rekordy z pliku "bibliografia.bib")
\bibliography{bibliografia}
\appendix
\chapter{Dodatek A}
\chapter{Opis załączonej płyty CD/DVD}\label{chap:opis-plyty}
Tutaj jest miejsce na zamieszczenie opisu zawartości załączonej płyty. Opis ten jest redagowany przed załadowaniem pracy do systemy APD USOS, a więc w chwili, gdy nieznana jest jeszcze nazwa, jaką system ten wygeneruje dla załadowanego pliku. Dlatego też redagując treść tego dodatku dobrze jest stosować ogólniki typu: ,,Na płycie zamieszczono dokument \texttt{pdf} z niniejszej tekstem pracy'' -- bez wskazywania nazwy tego pliku. 

Dawniej obowiązywała reguła, by nazywać dokumenty według wzorca \texttt{W04\_[nr albumu]\_[rok kalendarzowy]\_[rodzaj pracy]}, gdzie \texttt{rok kalendarzowy} odnosił się do roku realizacji kursu ,,Praca dyplomowa'', a nie roku obrony. Przykładowo wzorzec nazwy dla pracy dyplomowej inżynierskiej w konkretnym przypadku wyglądał tak: \texttt{W04\_123456\_2015\_praca inżynierska.pdf},  Takie nazwy utrwalane były w systemie składania prac dyplomowych. Obecnie działa to już inaczej.

% Jeśli w pracy pojawiać się ma indeks, należy odkomentować poniższe linie
%%\chapterstyle{noNumbered}
%%\phantomsection % sets an anchor
%%\addcontentsline{toc}{chapter}{Indeks rzeczowy}
%%\printindex

\end{document}
