\chapter{Sformułowanie problemu}\label{chap:problem}

Weźmy punkt $(x, y, z) \in \mathbb{R}^3$ i oznaczmy go $N$, ponadto weźmy zbiór $n$ punktów $\left\{(x_i, y_i, z_i): i \in [n]\right\} \subset \mathbb{R}^3$, który oznaczmy $\mathcal{O}$, a punkty należące do zbioru odpowiednio $O_i$. Niech $d_i$ będzie odległością $d(N, O_i)$. Znając współrzędne punktów $O_i$ oraz odległości $d_i$ chcemy znaleźć $(x, y, z)$. Można to osiągnąć rozwiązując układ równań
\begin{equation}\label{eq:base}
    \begin{dcases}
        {(x - x_1)}^2 + {(y - y_1)}^2 + {(z - z_1)}^2 = {d_1}^2\\
        {(x - x_2)}^2 + {(y - y_2)}^2 + {(z - z_2)}^2 = {d_2}^2\\
        \mspace{150mu}\vdots\\
        {(x - x_n)}^2 + {(y - y_n)}^2 + {(z - z_n)}^2 = {d_n}^2\,.
    \end{dcases}
\end{equation}

\section{Aktualny stan wiedzy}

W dostępnej literaturze obejmującej temat multilateracji i systemów multilateracyjnych opisane są trzy główne podejścia:

\begin{itemize}
    \item iteracyjne,
    \item algebraiczne,
    \item liniowe.
\end{itemize}

Metoda iteracyjna dąży do minimalizacji błędu przybliżonego rozwiązania układu równań~\ref{eq:base}. Typowymi narzędziami w niej wykorzystywanymi są: metoda najszybszego spadku~\cite{doi:10.1137/0111030}, metoda Newtona oraz szereg Taylora~\cite{4101619}. Praca~\cite{4101619} zasuguje na szczególną uwagę ze względu na skupienie się na zastosowaniu metody do celów lokalizacji, w przeciwieństwie do ogólnego podejścia do problemu optymalizacyjnego. Największymi problemami metod iteracyjnych są niepewność zbiegania do rozwiązania optymalnego a także złożoność obliczeniowa, która wpływa negatywnie na czas działania.

Metoda algebraiczna zaproponowana w pracy~\cite{4104017}, w której pochylono się na algebraicznym rozwiązaniu układów równań w systemie GPS, ma znaczącą przewagę nad metodami iteracyjnymi otrzymywania dwóch możliwych rozwiązań natychmiastowo, z ktorych można następnie wybrać bardziej odpowiadające. Dodatkowo ta metoda daje możliwość operowania na pomiarach z arbitralnej liczby punktów, w przeciwieństwie do metody zaproponowanej w pracy~\cite{article}, w której skupiono się na implementacji modelu VHDL pozycjonującego na postawie odczytów z czterech satelitów.

Wyżej wymienione metody, choć są wciąż używane, nie pojawiają się w pracach z ostatnich lat. Większość rozwiązań prezentowanych w ostatnich dekadach opiera się na układzie równań liniowych z dodatkową zmienną ze względu na nieliniowość problemu. Pierwsze tego typu rozwiązanie zaprezentowano w pracy~\cite{301830}, w którym zmienna nieliniowa była niwelowana przez odejmowanie od siebie kolejnych równań. Inny wariant przestawiono w pracy~\cite{1018778}, który zredagowano w czytelniejszej formie w pracy~\cite{norrdine2012algebraic}. Właśnie ta praca była inspiracją do niniejszej pracy. Niedawno opublikowana praca przeglądowa~\cite{10419087} zawiera szeroki opis innych metod wykorzystywanych do rozwiązania omawianego tutaj zagadnienia.

\section{Obrane rozwiązanie}

W pracy wybrano rozwiązanie przedstawione w pracy~\cite{norrdine2012algebraic}. Przekształćmy wyjściowy układ równań~\ref{eq:base} do postaci
\begin{equation}
    \begin{dcases}
        (x^2 + y^2 + z^2) - 2x_1x - 2y_1y - 2z_1z = d_1^2 - x_1^2 - y_1^2 - z_1^2\\
        (x^2 + y^2 + z^2) - 2x_2x - 2y_2y - 2z_2z = d_2^2 - x_2^2 - y_2^2 - z_2^2\\
        \mspace{220mu}\vdots\\
        (x^2 + y^2 + z^2) - 2x_n x - 2y_n y - 2z_n z = d_n^2 - x_n^2 - y_n^2 - z_n^2\,,
    \end{dcases}
\end{equation}
a następnie do reprezentacji macierzowej
\begin{equation}
    \left[
        \begin{matrix}
            1 & -2x_1 & -2y_1 & -2z_1\\
            1 & -2x_2 & -2y_2 & -2z_2\\
            \multicolumn{4}{c}{\vdots}\\
            1 & -2x_n & -2y_n & -2z_n\\
        \end{matrix}
    \right]
    \left[
        \begin{matrix}
            x^2 + y^2 + z^2\\
            x\\
            y\\
            z\\
        \end{matrix}
    \right]
    =
    \left[
        \begin{matrix}
            d_1^2 - x_1^2 - y_1^2 - z_1^2\\
            d_2^2 - x_2^2 - y_2^2 - z_2^2\\
            \vdots\\
            d_n^2 - x_n^2 - y_n^2 - z_n^2\\
        \end{matrix}
    \right]\,,
\end{equation}
którą można przedstawić jako
\begin{equation}
    \mathbf{A} \cdot \mathbf{x} = \mathbf{b}\,,
\label{eq:matrix}
\end{equation}
gdzie
\begin{equation}
    \mathbf{A} =
    \left[
        \begin{matrix}
            1 & -2x_1 & -2y_1 & -2z_1\\
            1 & -2x_2 & -2y_2 & -2z_2\\
            \multicolumn{4}{c}{\vdots}\\
            1 & -2x_n & -2y_n & -2z_n\\
        \end{matrix}
    \right]\,,
\end{equation}
\begin{equation}
    \mathbf{x} =
    \left[
        \begin{matrix}
            x^2 + y^2 + z^2\\
            x\\
            y\\
            z\\
        \end{matrix}
    \right]\,,
\end{equation}
\begin{equation}
    \mathbf{b} =
    \left[
        \begin{matrix}
            d_1^2 - x_1^2 - y_1^2 - z_1^2\\
            d_2^2 - x_2^2 - y_2^2 - z_2^2\\
            \vdots\\
            d_n^2 - x_n^2 - y_n^2 - z_n^2\\
        \end{matrix}
    \right]\,,
\end{equation}
\noindent
przy założeniu, że $\mathbf{x} \in E$, gdzie
\begin{equation}
    E = \left\{(x_0, x_1, x_2, x_3)^T \in {\mathbb{R}}^4\ |\ x_0 = {x_1}^2 + {x_2}^2 + {x_3}^2\right\}
    \label{eq:constraint}
\end{equation}

\subsection{Liniowa metoda najmniejszych kwadratów}

Aby rozwiązać powyższe równanie liniowe obrano liniową metodę najmniejszych kwadratów. Z definicji rozwiązaniem równania
\begin{equation}
    \mathbf{A} \cdot \mathbf{x} = \mathbf{b}\,,
\end{equation}
gdzie $\mathbf{A} \in \mathbb{R}^{n\times m}$, $\mathbf{b} \in \mathbb{R}^n$ są znane, będzie taki $\hat{\mathbf{x}} \in \mathbb{R}^m$, który minimalizuje
\begin{equation}
    \sum_{i=1}^{n}{\left|\sum_{j=1}^{m}{A_{ij}x_j} - b_i\right|^2} = ||\mathbf{A}\hat{\mathbf{x}} - \mathbf{b}||^2
\end{equation}
Taki $\hat{\mathbf{x}}$ obliczamy przy użyciu powszechnie znanego w kontekście najmniejszych kwadratów \textit{równania normalnego} postaci
\begin{equation}
    \mathbf{A}^T\mathbf{A}\hat{\mathbf{x}} = \mathbf{A}^T\mathbf{b}
\end{equation}
lub po przekształceniu
\begin{equation}\label{eq:lls}
    \hat{\mathbf{x}} = {\left(\mathbf{A}^T\mathbf{A}\right)}^{-1}\mathbf{A}^T\mathbf{b}
\end{equation}
Jeśli żadna trójka odbiorników nie jest współliniowa ($\rank\mathbf{A} = m$) oraz pomiarów jest więcej niż wymiarów $(n > m)$ możemy zagwarantować odwracalność $\mathbf{A}$, a co za tym idzie istnienie $\hat{\mathbf{x}}$.

\subsubsection{Forma ogólna}
Zaletą reprezentacji problemu~\ref{eq:matrix} zawartą w obranym rozwiązaniu jest łatwość rozszerzenia obliczeń na dowolną $m$-wymiarową przestrzeń oraz $n$-odbiorników:
\begin{equation}
    \left[
        \begin{matrix}
            1 & -2\mathbf{x^{(1)}}\\
            1 & -2\mathbf{x^{(2)}}\\
            \vdots & \vdots\\
            1 & -2\mathbf{x^{(n)}}\\
        \end{matrix}
    \right]
    \left[
        \begin{matrix}
            \sum_{i=1}^{m}{x_i}^2\\
            x_1\\
            \vdots\\
            x_m
        \end{matrix}
    \right]
    =
    \left[
        \begin{matrix}
            d_1^2 - \sum_{i=1}^{m}{x_i^{(1)}}^2\\
            d_2^2 - \sum_{i=1}^{m}{x_i^{(2)}}^2\\
            \vdots\\
            d_n^2 - \sum_{i=1}^{m}{x_i^{(n)}}^2\\
        \end{matrix}
    \right]
\end{equation}
z analogicznym do~\ref{eq:constraint} ograniczeniem $\mathbf{x} \in E$.

\section{Przykład problemu}

\begin{figure}[!h]
    \centering
    \begin{tikzpicture}
        \coordinate (O1) at (0,0);
        \coordinate (O2) at (6,0);
        \coordinate (O3) at (6,6);
        \coordinate (O4) at (0,6);
        \coordinate (N) at (0.75,4.5);
        \filldraw[black] (O1) circle (2pt) node[left]{$O_1$};
        \filldraw[black] (O2) circle (2pt) node[below]{$O_2$};
        \filldraw[black] (O3) circle (2pt) node[right]{$O_3$};
        \filldraw[black] (N) circle (2pt) node[above left]{$N$};
        \draw[black] (O1) -- node[right]{$d_1$} (N);
        \draw[black] (O2) -- node[left]{$d_2$} (N);
        \draw[black] (O3) -- node[above]{$d_3$} (N);
        \draw[loosely dashed, thin] (O1) -- (O2);
        \draw[loosely dashed, thin] (O2) -- (O3);
        \draw[loosely dashed, thin] (O3) -- (O4);
        \draw[loosely dashed, thin] (O4) -- (O1);
    \end{tikzpicture}
    \caption{Egzemplarz problemu multilateracji}
    \label{fig:example}
\end{figure}

Staramy się ustalić pozycję punktu $N$ na postawie współrzędnych punktów $O_1, O_2, O_3$ oraz odległości $d_1, d_2, d_3$. Wykorzystując równanie~\ref{eq:lls}, podstawiamy znane wartości, otrzymując $\mathbf{x}$ zawierający szukane współrzędne. 

Jak łatwo zauważyć, pierwsza część lewej strony równości zależy jedynie od macierzy $\mathbf{A}$, która nie zawiera zmiennych związanych z węzłem źródłowym, co pozwala, przy założeniu niezmienności pozycji odbiorników, na wcześniejsze i jednorazowe przeprowadzenie kosztownych przekształceń tej macierzy. W tej sytuacji jednorazowe obliczenie rozwiązania wymaga jedynie obliczenia wektora $\mathbf{b}$ oraz iloczynu skalarnego.

Przyjmijmy wartości dla odbiorników $O_1(0;0), O_2(1;0), O_3(1;1)$ oraz załóżmy, że punkt $N$ znajduje się na pozycji $\left(0,125;0,75\right)$. Odległości $d_1, d_2, d_3$ wynoszą odpowiednio $\sqrt{0,578125}, \sqrt{1,328125}$ i $\sqrt{0,828125}$. Wprowadźmy teraz błąd pomiaru poprzez zaokrąglenie długości do $0,05$. Nasze odległości $d_1', d_2', d_3'$ wynoszą $0,75; 1,15$ oraz $0,9$. Dla tych danych macierz $\mathbf{A}$ ma postać
\begin{equation*}
    \mathbf{A} =
    \left[
        \begin{matrix}
            1 & 0 & 0\\
            1 & -2 & 0\\
            1 & -2 & -2\\
        \end{matrix}
    \right]\,,
\end{equation*}
a wektor $\mathbf{b}$

\begin{equation*}
    \mathbf{b} =
    \left[
        \begin{matrix}
            0,5625\\
            0,3225\\
            -1,19\\
        \end{matrix}
    \right]\,.
\end{equation*}
Idąc dalej prawa część równania~\ref{eq:lls} przyjmie postać
\begin{equation*}
    {\left(\mathbf{A}^T\mathbf{A}\right)}^{-1}\mathbf{A}^T\mathbf{b} =
    {\left(
    \left[
        \begin{matrix}
            1 & 1 & 1\\
            0 & -2 & -2\\
            0 & 0 & -2\\
        \end{matrix}
    \right]
    \cdot
    \left[
        \begin{matrix}
            1 & 0 & 0\\
            0,5 & -0,5 & 0\\
            0 & 0,5 & -0,5\\
        \end{matrix}
    \right]
    \right)}^{-1}
    \cdot
    \left[
        \begin{matrix}
            1 & 1 & 1\\
            0 & -2 & -2\\
            0 & 0 & -2\\
        \end{matrix}
    \right]
    \cdot
    \left[
        \begin{matrix}
            0,5625\\
            0,3225\\
            -1,19\\
        \end{matrix}
    \right]\,,
\end{equation*}
co daje nam rozwiązanie aproksymacyjne
\begin{equation*}
    \hat{\mathbf{x}} =
    \left[
    \begin{matrix}
        0,5625\\
        0,12\\
        0,75625\\
    \end{matrix}
    \right]
\end{equation*}
Z niego odczytujemy współrzędne odbiornika $(0,12;0,75625)$, które są bliskie rzeczywistej pozycji $N$.

