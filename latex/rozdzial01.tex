\chapter{Przedstawienie problemu}\label{chap:problem}

Multilateracja jest techniką lokalizacji pozwalającą obliczyć nieznane koordynaty punktu na podstawie odległości od innych, znanych punktów. Weźmy trójwymiarowy egzemplarz naszego problemu (Rys.~\ref{fig:example}), gdzie $N$ - nadajnik, $O_i$ - odbiorniki, $d_i$ - odległości. Przedstawienie problemu zostało zaczerpnięte z pracy~\cite{norrdine2012algebraic}.

\begin{figure}[!h]
    \centering
    \begin{tikzpicture}
        \coordinate (O1) at (0,2);
        \coordinate (O2) at (2,0);
        \coordinate (O3) at (8,3);
        \coordinate (N) at (3,5);
        \filldraw[black] (O1) circle (2pt) node[left]{$O_1$};
        \filldraw[black] (O2) circle (2pt) node[below]{$O_2$};
        \filldraw[black] (O3) circle (2pt) node[right]{$O_3$};
        \filldraw[black] (N) circle (2pt) node[above right]{$N$};
        \draw[black] (O1) -- node[above left]{$d_1$} (N);
        \draw[black] (O2) -- node[left]{$d_2$} (N);
        \draw[black] (O3) -- node[above]{$d_3$} (N);
        \draw[loosely dashed, thin] (O1) -- (O2);
        \draw[loosely dashed, thin] (O1) -- (O3);
        \draw[loosely dashed, thin] (O2) -- (O3);
    \end{tikzpicture}
    \caption{Egzemplarz problemu multilateracji}
\label{fig:example}
\end{figure}

Znalezienie koordynatów $(x,y,z)$ punktu N jest równoważne z rowiązaniem układu równań,
\begin{equation}\label{eq:base}
    \begin{dcases}
        {(x - x_1)}^2 + {(y - y_1)}^2 + {(z - z_1)}^2 = {d_1}^2\\
        {(x - x_2)}^2 + {(y - y_2)}^2 + {(z - z_2)}^2 = {d_2}^2\\
        {(x - x_3)}^2 + {(y - y_3)}^2 + {(z - z_3)}^2 = {d_3}^2\\
    \end{dcases}
\end{equation}

\section{Aktualny stan wiedzy}

W dostępnej literaturze obejmującą temat multilateracji i systemów multilateracyjnych opisanych jest wiele podejść do tego problemu:

\begin{itemize}
    \item układ równań liniowych\cite{murphy1995determination},
    \item metoda najmniejszej sumy kwadratów\cite{murphy1995determination},\cite{norrdine2012algebraic},
    \item nieliniowa metoda najmniejszej sumy kwadratów\cite{murphy1995determination},
    \item rozkład według wartości szczególnych\cite{murphy1995determination}.
\end{itemize}
Na szczególną uwagę zasługuje praca~\cite{murphy1995determination}, w której przeprowadzono eksperymenty porównawcze wyżej wymienionych metod.

Motywacją tej pracy jest brak dostępnych prac obejmujących to zagadnienie w spektrum fal dźwiękowych, w przeciwieństwie do fal elektromagnetycznych, an których opierają się systemy opisane w pozycjach bibliograficznych. Pytania, które zostały postawione:

\subsection{Obrane rozwiązanie}

W pracy wybrano rozwiązanie przedstawione w pracy~\cite{norrdine2012algebraic}. Przekształćmy podstawowy układ równań~\ref{eq:base} do postaci
\begin{equation}
    \begin{dcases}
        (x^2 + y^2 + z^2) - 2x_1x - 2y_1y - 2z_1z = d_1^2 - x_1^2 - y_1^2 - z_1^2\\
        (x^2 + y^2 + z^2) - 2x_2x - 2y_2y - 2z_2z = d_2^2 - x_2^2 - y_2^2 - z_2^2\\
        (x^2 + y^2 + z^2) - 2x_3x - 2y_3y - 2z_3z = d_3^2 - x_3^2 - y_3^2 - z_3^2\\
    \end{dcases}
\end{equation},
a następnie do reprezenacji macierzowej,
\begin{equation}
    \left[
        \begin{matrix}
            1 & -2x_1 & -2y_1 & -2z_1\\
            1 & -2x_2 & -2y_2 & -2z_2\\
            1 & -2x_3 & -2y_3 & -2z_3\\
        \end{matrix}
    \right]
    \left[
        \begin{matrix}
            x^2 + y^2 + z^2\\
            x\\
            y\\
            z\\
        \end{matrix}
    \right]
    =
    \left[
        \begin{matrix}
            d_1^2 - x_1^2 - y_1^2 - z_1^2\\
            d_2^2 - x_2^2 - y_2^2 - z_2^2\\
            d_3^2 - x_3^2 - y_3^2 - z_3^2\\
        \end{matrix}
    \right]
\end{equation},
którą można przedstawić jako
\begin{equation}
    \mathbf{A} \cdot \mathbf{x} = \mathbf{b}
\label{eq:matrix}
\end{equation}
przy założeniu, że $\mathbf{x} \in E$, gdzie $E = \left\{(x_0, x_1, x_2, x_3)^T \in {\mathbb{R}}^4\ |\ x_0 = {x_1}^2 + {x_2}^2 + {x_3}^2\right\}$.

\subsection*{Rozwiązanie ogólne}

Uogólniona forma równania macierzowego problemu multilateracji dla przestrzeni $n$-wymiarowej i $m$ odbiorników:
\begin{equation}
    \left[
        \begin{matrix}
            1 & -2\mathbf{x^{(1)}}\\
            1 & -2\mathbf{x^{(2)}}\\
            \vdots & \vdots\\
            1 & -2\mathbf{x^{(m)}}\\
        \end{matrix}
    \right]
    \left[
        \begin{matrix}
            \sum_{i=1}^{n}{x_i}^2\\
            x_1\\
            \vdots\\
            x_n
        \end{matrix}
    \right]
    =
    \left[
        \begin{matrix}
            d_1^2 - \sum_{i=1}^{n}{x_i^{(1)}}^2\\
            d_2^2 - \sum_{i=1}^{n}{x_i^{(2)}}^2\\
            \vdots\\
            d_m^2 - \sum_{i=1}^{n}{x_i^{(m)}}^2\\
        \end{matrix}
    \right]
\end{equation}
z analogicznym ograniczeniem $\mathbf{x} \in E$. Zachowując oznaczenia z równania~\ref{eq:matrix} rozwiązanie aproksymacyjne $\hat{\mathbf{x}}$ w sensie metody najmniejszych kwadratów ma postać

\begin{equation}\label{eq:lls}
    \hat{\mathbf{x}} = {\left(\mathbf{A}^T\mathbf{A}\right)}^{-1}\mathbf{A}^T\mathbf{b}
\end{equation}

Jak łatwo zauważyć pierwsza część lewej strony równości zależy jedynie od macierzy $\mathbf{A}$, która nie zawiera zmiennych związanych z węzłem źródłowym, co pozwala, przy założeniu stacjonarności odbiorników, na wcześniejsze i jednorazowe przeprowadzenie kosztownych przekształceń tej macierzy. W tej sytuacji jednorazowe obliczenie rozwiązania wymaga jedynie obliczenia wektora $\mathbf{b}$ oraz iloczynu skalarnego.

\begin{itemize}
    \item 
\end{itemize}
