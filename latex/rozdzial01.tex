\chapter{Przedstawienie problemu}

Multilateracja jest techniką lokalizacji pozwalającą obliczyć nieznane koordynaty punktu na podstawie odległości od innych, znanych punktów. Weźmy dwuwymiarowy egzemplarz naszego problemu (Rys.~\ref{fig:example}), gdzie $N$ - nadajnik, $O_i$ - odbiorniki, $d_i$ - odległości

\begin{figure}[!h]
    \centering
    \begin{tikzpicture}
        \coordinate (O1) at (0,0);
        \coordinate (O2) at (3,5);
        \coordinate (O3) at (8,1);
        \coordinate (N) at (4,3);
        \filldraw[black] (O1) circle (2pt) node[left]{$O_1$};
        \filldraw[black] (O2) circle (2pt) node[right]{$O_2$};
        \filldraw[black] (O3) circle (2pt) node[right]{$O_3$};
        \filldraw[black] (N) circle (2pt) node[above right]{$N$};
        \draw[black, thin] (O1) -- node[above left]{$d_1$} (N);
        \draw[black, thin] (O2) -- node[left]{$d_2$} (N);
        \draw[black, thin] (O3) -- node[above]{$d_3$} (N);
    \end{tikzpicture}
    \caption[short]{Egzemplarz problemu multilateracji}
\label{fig:example}
\end{figure}

Znalezienie koordynatów $(x,y)$ punktu N jest równoważne z rowiązaniem układu równań,

\begin{equation}
    \begin{dcases}
        {(x - x_1)}^2 + {(y - y_1)}^2 = {d_1}^2\\
        {(x - x_2)}^2 + {(y - y_2)}^2 = {d_2}^2\\
        {(x - x_3)}^2 + {(y - y_3)}^2 = {d_3}^2\\
    \end{dcases}
\end{equation}

który może zostać przekształcony do postaci

\begin{equation}
    \begin{dcases}
        (x^2 + y^2) - 2x_1x - 2y_1y = d_1^2 - x_1^2 - y_1^2\\
        (x^2 + y^2) - 2x_2x - 2y_2y = d_2^2 - x_2^2 - y_2^2\\
        (x^2 + y^2) - 2x_3x - 2y_3y = d_3^2 - x_3^2 - y_3^2\\
    \end{dcases}
\end{equation}

lub w reprezenacji macierzowej,

\begin{equation}
    \left[
        \begin{matrix}
            1 & -2x_1 & -2y_1\\
            1 & -2x_2 & -2y_2\\
            1 & -2x_3 & -2y_3\\
        \end{matrix}
    \right]
    \left[
        \begin{matrix}
            x^2 + y^2\\
            x\\
            y\\
        \end{matrix}
    \right]
    =
    \left[
        \begin{matrix}
            d_1^2 - x_1^2 - y_1^2\\
            d_2^2 - x_2^2 - y_2^2\\
            d_3^2 - x_3^2 - y_3^2\\
        \end{matrix}
    \right]
\end{equation}

którą można przedstawić jako

\begin{equation}
    \boldsymbol{A} \cdot \boldsymbol{x} = \boldsymbol{b}
\end{equation}

Uogólniona forma równania macierzowego problemu multilateracji dla przestrzeni $n$-wymiarowej i $m$ odbiorników:

\begin{equation}
    \left[
        \begin{matrix}
            1 & -2\boldsymbol{x^{(1)}}\\
            1 & -2\boldsymbol{x^{(2)}}\\
            \vdots & \vdots\\
            1 & -2\boldsymbol{x^{(m)}}\\
        \end{matrix}
    \right]
    \left[
        \begin{matrix}
            \sum_{i=1}^{n}{x_i}^2\\
            x_1\\
            \vdots\\
            x_n
        \end{matrix}
    \right]
    =
    \left[
        \begin{matrix}
            d_1^2 - \sum_{i=1}^{n}{x_i^{(1)}}^2\\
            d_2^2 - \sum_{i=1}^{n}{x_i^{(2)}}^2\\
            \vdots\\
            d_m^2 - \sum_{i=1}^{n}{x_i^{(m)}}^2\\
        \end{matrix}
    \right]
\end{equation}


\section{State of the art}

Napisać coś o~\cite{murphy1995determination},\cite{norrdine2012algebraic}
